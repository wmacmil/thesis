\section{HoTT Grammars} \label{hottgram}

\subsection{Twin Prime Conjecture Revisited} \label{twins}
% \subsection{What is Equality?} \label{eqlty}

% quote peter dybjer at oplss
% all out assault on equality

Mathematicians, and most people generally, have an intuition
for equality, that of an identification between two pieces of information
which intuitively must be the same thing, i.e. $2+2=4$. The philosophically
inclined might ask about identification generally. We showcase different
notions of identifying things in mathematics, logic, and type theory :

\begin{itemize}
\item Equivalence of propositions (logic)
\item Equality of sets (mathematics)
\item Equality of members of sets (mathematics)
\item Isomorphism of structures (algebra)
\item Equality of terms (type theory)
\item Equality of types (type theory)
\item Deformation of paths (homotopy theory)
\item Arrows between arrows (category theory)
\end{itemize}

While there are notions of equality, sameness, or identification outside of
these formal domains, we don't dare take a philosophical stab at these notions
here. Earlier, we saw two notions of equality in type theory, judgmental and
propositional. As was seen, Judgmental equality provides a means of computing,
for instance, that $2+2=4$ is evediently true by appealing to its definition.
Propositional equality, on the other hand, is actually a type. It is defined as
follows in Agda, with an accompanying natural language definition from [cite
hottbook] :

\begin{code}[hide]
{-# OPTIONS --cubical #-}

module hott where

module Id where
\end{code}
\begin{code}
  data _≡'_ {A : Set} : (a b : A) → Set where
    r : (a : A) → a ≡' a
\end{code}
\begin{definition}
  The formation rule says that given a type $A:\UU$ and two elements $a,b:A$, we can form the type $(\id[A]{a}{b}):\UU$ in the same universe.
  The basic way to construct an element of $\id{a}{b}$ is to know that $a$ and $b$ are the same.
  Thus, the introduction rule is a dependent function
  \[\refl{} : \prod_{a:A} (\id[A]{a}{a}) \]
  called \define{reflexivity},
  which says that every element of $A$ is equal to itself (in a specified way).  We regard $\refl{a}$ as being the
  constant path %path\indexdef{path!constant}\indexsee{loop!constant}{path, constant}
  at the point $a$.
\end{definition}


The astute might ask, what
does it mean to ``construct an element of $\id{a}{b}$''? For the mathematician
use to thinking in terms of sets $\{\id{a}{b} \mid a,b \in \mathbb{N} \}$ isn't
a well-defined notion. Due to its use of the axiom of extensionality, the set
theoretic notion of equality is, no suprise, extensional.  This means that sets
are identified when they have the same elements, and equality is therefore
external to the notion of set. To inhabit a type means to provide evidence for
that inhabitation. The reflexivity constructor is therefore a means of
providing evidence of an equality. This evidence approach is disctinctly
constructive, and a big reason why classical and constructive mathematics,
especially when treated in an intuitionistic type theory suitable for a
programming language implementation, are such different beasts.

In Martin-Löf Type Theory, there are two fundamental notions of equality,
propositional and definitional.  While propositional equality is inductively
defined (as above) as a type which may have possibly more than one inhabitant,
definitional equality, denoted $-\equiv -$ and perhaps more aptly named
computational equality, is familiarly what most people think of as equality.
Namely, two terms which compute to the same canonical form are computationally
equal. In intensional type theory, propositional equality is a weaker notion
than computational equality : all computationally equal terms are
propositionally equal. If there is a rule allowing one to transform
propositional equalities into definitional ones, one enters into the space of
extensional type theory. 

Prior to the homotopical interpretation of identity types, debates about
extensional and intensional type theories centred around two features or bugs :
extensional type theory sacrificed decideable type checking, while intensional
type theories required extra beauracracy when dealing with equality in proofs.
One approach in intensional type theories treated types as setoids, therefore
leading to so-called ``Setoid Hell''. These debates reflected Martin-Löf's
flip-flopping on the issue. His seminal Constructive Mathematics and Computer
Programming [cite], took an extensional view, and was soon betrayed by lectures
he gave soon thereafter in Padova in 1980. Martin-Löf was a born again
intensional type theorist. These Padova lectures were later published in the
"Bibliopolis Book" [cite], and went on to inspire the European (and Gothenburg
in particular) approach to implementing proof assitants, whereas the
extensionalists were primarily eminating from Robert Constable's group at
Cornell.

This tension has now been at least partially resolved, or at the very least
clarified, by an insight Voevodsky was apparently most proud of : the
introduction of h-levels. We'll delegate these details for a later section, it
is mentioned here to indicate that extensional type theory was really ``set
theory'' in disguise, in that it collapses the higher path structure of
identity types. The work over the past 10 years has elucidated the intensional
and extensional positions. HoTT, by allowing higher paths, is unashamedly
intentional, and admits a collapse into the extensional universe if so desired.

\subsection{Why HoTT grammars?}

There were abundant incentives motivating us to use HoTT as a case study for
grammatical transformations of mathematics.  These include :

\begin{itemize}[noitemsep]
\item HoTT is a new foundation of mathematics
\item Much expertise revolves around Göteborg, and Agda 
\item The HoTT book is a canonical reference
\item Most of the work is being formalized concurrent to its being written 
\item Independent of set theory
\item Univalence allows us to compare notions of equality
\item Higher inductive types allow for ``more natural" higher dimensional constructions
\item Cubical Type Theory (CTT) allowing for a constructive interpretation of univalence
\item It is interdisciplinary and is transforming the boundaries of disciplines
\end{itemize}

Other projects, like lean [cite avigad] are seeking to do classical mathematics
in a way that's relatively familiar to classical mathematicians, and due to its
extensive libraries, from some vantages at least, it would give easier
ways of accessing classical results, and possibly appeal more directly to
mathematicians [cite buzzard]. The recent formalization of \emph{perfectoid
  spaces} [cite buzzard], for example, validate that Lean may soon offer
mathematicians an toolbox worthy them spending their time on.

Homotopy Type theory and its offspring have greatly centered around the Agda
community, is perhaps less practically oriented with more of a foundational
emphasis than other type theories (and the proof assistants it has inspired).
Nonetheless, HoTT gives a direct means of interpretation on both a proof assistant,
on pen and paper, and perhaps most importantly appeals to the semantic intuition
of the topologist. These reasons, for us, makes it the most theoretically
appealing from the natural language perspective, because it's simultaneously
trying to emphasize the multisorted nature of the Trinitarian subjects and their
various cross-interpretations, with syntax being seen as one main obstacle in
getting disparate researchers to coalesce. 

The two grammars we explore HoTT we explore, emphasize the duality of
programming and writing which HoTT is itself attempting to accentuate. Prior to
the writing of the grammars, much of chapter 2 of the HoTT book was formalized
in Agda with a fixation of syntactic overlap. This wasn't entirely possible, for
we found certain proofs could be proven other ways more easily in Agda. The
exercise showed how important the creation of a corpus can be, for no other
reason than to analyze the language in an unambiguous setting. While the goal of
both ``informalizing" and possibly generating Agda code was originally proposed,
it turned out that the myriad tools and skills needed to get there required
navigating many subtle issues that have been outlined above. We hope these final
two grammars and a brief comparison of them will give the reader a new lens with
which to study HoTT.
