\documentclass[11pt, a4paper]{article}

\usepackage{mlt-thesis-2015}

% With Xetex/Luatex this shouldn't be used
%\usepackage[utf8]{inputenc}

\usepackage{csquotes}

\usepackage[english]{babel}
\usepackage{graphicx}
\usepackage{setspace}

\usepackage{tikz-cd}

% from here

\usepackage{fontspec}
\usepackage{fullpage}
\usepackage{hyperref}
\usepackage{agda}

\usepackage{unicode-math}

%\usepackage{amssymb,amsmath,amsthm,stmaryrd,mathrsfs,wasysym}
\usepackage{stmaryrd}

%\usepackage{enumitem,mathtools,xspace}
\usepackage{amsfonts}
\usepackage{mathtools}
\usepackage{xspace}


\usepackage{enumitem}


\setmainfont{DejaVu Serif}
\setsansfont{DejaVu Sans}
\setmonofont{DejaVu Sans Mono}

% \setmonofont{Fira Mono}
% \setsansfont{Noto Sans}

\usepackage{newunicodechar}

\newunicodechar{ℓ}{\ensuremath{\mathnormal\ell}}
\newunicodechar{→}{\ensuremath{\mathnormal\rightarrow}}

\newtheorem{definition}{Definition}
\newtheorem{lem}{Lemma}
\newtheorem{proof}{Proof}
\newtheorem{thm}{Theorem}

\newcommand{\jdeq}{\equiv}      % An equality judgment
\newcommand{\refl}[1]{\ensuremath{\mathsf{refl}_{#1}}\xspace}
\newcommand{\define}[1]{\textbf{#1}}
\newcommand{\defeq}{\vcentcolon\equiv}  % A judgmental equality currently being defined

%\newcommand{\jdeq}{\equiv}      % An equality judgment
%\let\judgeq\jdeq


\newcommand{\ind}[1]{\mathsf{ind}_{#1}}
\newcommand{\indid}[1]{\ind{=_{#1}}} % (Martin-Lof) path induction principle for identity types

\newcommand{\blank}{\mathord{\hspace{1pt}\text{--}\hspace{1pt}}}

\newcommand{\opp}[1]{\mathord{{#1}^{-1}}}
\let\rev\opp

\newcommand{\id}[3][]{\ensuremath{#2 =_{#1} #3}\xspace}



\newcommand{\UU}{\ensuremath{\mathcal{U}}\xspace}
\let\bbU\UU
\let\type\UU


%\newcommand{\ct}{%
  %\mathchoice{\mathbin{\raisebox{0.5ex}{$\displaystyle\centerdot$}}}%
             %{\mathbin{\raisebox{0.5ex}{$\centerdot$}}}%
             %{\mathbin{\raisebox{0.25ex}{$\scriptstyle\,\centerdot\,$}}}%
             %{\mathbin{\raisebox{0.1ex}{$\scriptscriptstyle\,\centerdot\,$}}}
%}

% til here

\title{A grammar of proof}
% \subtitle{Subtitle} case study in formal & nl proof translation
\author{Warrick Macmillan}

\begin{document}

%% ============================================================================
%% Title page
%% ============================================================================
\begin{titlepage}

\maketitle

\vfill

\begingroup
\renewcommand*{\arraystretch}{1.2}
\begin{tabular}{l@{\hskip 20mm}l}
\hline
Master's Thesis: & 30 credits\\
Programme: & Master’s Programme in Language Technology\\
Level: & Advanced level \\
Semester and year: & Fall, 2021\\
Supervisor & Aarne Ranta\\
Examiner & (name of the examiner)\\
Report number & (number will be provided by the administrators) \\
Keywords &  Grammatical Framework, Natural Language Generation,\\
\end{tabular}
\endgroup

\thispagestyle{empty}
\end{titlepage}

%% ============================================================================
%% Abstract
%% ============================================================================
\newpage
\singlespacing
\section*{Abstract}

Brief summary of research question, background, method, results\ldots

\thispagestyle{empty}

%% ============================================================================
%% Preface
%% ============================================================================
\newpage
\section*{Preface}

Acknowledgements, etc.

\thispagestyle{empty}

%% ============================================================================
%% Contents
%% ============================================================================
\newpage

\begin{spacing}{0.0}
\tableofcontents
\end{spacing}

\thispagestyle{empty}

%% ============================================================================
%% Introduction
%% ============================================================================
\newpage
\setcounter{page}{1}

% for it to be lagda
\begin{code}[hide]
{-# OPTIONS --cubical #-}

module roadmap where
\end{code}


1428

\section{Introduction}
\label{sec:intro}

The central concern of this thesis is the syntax of mathematics, programming
languages, and their respective mutual influence, as conceived and practiced by
mathematicians and computer scientists.  From one vantage point, the role of
syntax in mathematics may be regarded as a 2nd order concern, a topic for
discussion during a Fika, an artifact of ad hoc development by the working
mathematician whose real goals are producing genuine mathematical knowledge.
For the programmers and computer scientists, syntax may be regarding as a
matter of taste, with friendly debates recurring regarding the use of
semicolons, brackets, and white space.  Yet, when viewed through the lens of
the propositions-as-types paradigm, these discussions intersect in new and
interesting ways.  When one introduces a third paradigm through which to
analyze the use of syntax in mathematics and programming, namely linguistics, I
propose what some may regard as superficial detail, indeed becomes a central
paradigm raising many interesting and important questions. 


\subsection{Beyond Computational Trinitarianism}

\begin{displayquote}

The doctrine of computational trinitarianism holds that computation manifests
itself in three forms: proofs of propositions, programs of a type, and mappings
between structures. These three aspects give rise to three sects of worship:
Logic, which gives primacy to proofs and propositions; Languages, which gives
primacy to programs and types; Categories, which gives primacy to mappings and
structures.\cite{harperTrinity}
\end{displayquote}

We begin this discussion of the three relationships between three respective
fields, mathematics, computer science, and logic. The aptly named 
trinity, shown in \autoref{fig:M1}, are related via both \emph{formal} and \emph{informal}
methods. The propositions as types paradigm, for example, is a heuristic. Yet
it also offers many examples of successful ideas translating between the domains.
Alternatively, the interpretation of a Type Theory(TT) into a category theory is
incredibly \emph{formal}.


\begin{figure}[H]
\centering
\begin{tikzcd}
                                                                            &  &  & Logic \arrow[llldddd, "Denotational\ Semantics" description] \arrow[rrrdddd, "Include\ Terms" description] &  &  &                                                                                                       \\
                                                                            &  &  &                                                                                                            &  &  &                                                                                                       \\
                                                                            &  &  &                                                                                                            &  &  &                                                                                                       \\
                                                                            &  &  &                                                                                                            &  &  &                                                                                                       \\
Math \arrow[rrruuuu, "Embedded\ in\ FOL", bend left] \arrow[rrrrrr, "ITP"'] &  &  &                                                                                                            &  &  & CS \arrow[llllll, "Denotational\ Semantics", bend left] \arrow[llluuuu, "Remove\ Terms"', bend right]
\end{tikzcd}
\caption{The Holy Trinity} \label{fig:M1}
\end{figure}

We hope this thesis will help clarify another possible dimension in this
diagram, that of Linguistics, and call it the ``holy tetrahedron". The different
vertices also resemble
religions in their own right, with communities convinced that they have a
canonical perspective on foundations and the essence of mathematics. Questioning the holy trinity is an act of a heresy, and
it is the goal of this thesis to be a bit heretical by including a much less well understood 
perspective which provides additional challenges and
insights into the trinity.

\begin{figure}[H]
\centering
\begin{tikzcd}
     &  &  & Logic                                                                                                                     &  &  &            \\
     &  &  &                                                                                                                           &  &  &            \\
     &  &  & Linguistics \arrow[uu, "Montague\ Semantics"'] \arrow[llldd, "Distributional\ Semantics"'] \arrow[rrrdd, "TT\ Semantics"] &  &  &            \\
     &  &  &                                                                                                                           &  &  &            \\
Math &  &  &                                                                                                                           &  &  & CS\ (MLTT)
\end{tikzcd}
\caption{Formal Semantics} \label{fig:M2}
\end{figure}

One may see how the trinity give rise to \emph{formal} semantic interpretations
of natural language in \autoref{fig:M2}. Semantics is just one possible
linguistic phenomenon worth investigating in these domains, and could be
replaced by other linguistic paradigms. This thesis is alternatively concerned
with syntax.

Finally, as in \autoref{fig:M3}, we can ask : how does the trinity embed into
natural language? These are the most \emph{informal} arrows of tetrahedron, or
at least one reading of it. One can analyze mathematics using linguistic
methods, or try to give a natural language justification of Intuitionistic Type
Theory (ITT) using Martin-Löf's meaning explanations.

\begin{figure}[H]
\centering
\begin{tikzcd}
                                                &  &  & Logic \arrow[dd, "Embedding"] &  &  &                               \\
                                                &  &  &                               &  &  &                               \\
                                                &  &  & Linguistics                   &  &  &                               \\
                                                &  &  &                               &  &  &                               \\
Math \arrow[rrruu, "Language\ Of\ Mathematics"] &  &  &
&  &  & CS\ (MLTT) \arrow[llluu, "Meaning\ Explanations"]
\end{tikzcd}
\caption{Interpretations of Natural Language} \label{fig:M3}
\end{figure}

In this work, we will see that there are multiple GF grammars which model some
subset of each member of the trinity. Constructing these grammars, and asking
how they can be used in applications for mathematicians, logicians, and computer
scientists is an important practical and philosophical question. Therefore we
hope this attempt at giving the language of mathematics, in particular how
propositions and proofs are expressed and thought about in that language, a
stronger foundation.

\subsection{What is a Homomorophism?}

To get a feel for the syntactic paradigm we explore in this thesis, let us look at a basic mathematical
example: that of a group homomorphism as expressed in by a variety of somewhat
randomly sampled authors.  

% Wikipedia Defn:

\begin{definition}
In mathematics, given two groups, $(G, \ast)$ and $(H, \cdot)$, a group homomorphism from $(G, \ast)$ to $(H, \cdot)$ is a function $h : G \to H$ such that for all $u$ and $v$ in $G$ it holds that
  $$h(u \ast v) = h ( u ) \cdot h ( v )$$ 
\end{definition}

% http://math.mit.edu/~jwellens/Group%20Theory%20Forum.pdf

\begin{definition}
Let $G = (G,\cdot)$ and $G' = (G',\ast)$ be groups, and let $\phi : G \to G'$ be a map between them. We call $\phi$ a \textbf{homomorphism} if for every pair of elements $g, h \in G$, we have 
% \begin{center}
  $$\phi(g \ast h) = \phi ( g ) \cdot \phi ( h )$$ 
% \end{center}
\end{definition}

% http://www.maths.gla.ac.uk/~mwemyss/teaching/3alg1-7.pdf

\begin{definition}\label{def:def3}
Let $G$, $H$, be groups.  A map $\phi : G \to H$ is called a \emph{group homomorphism} if
  $$\phi(xy) = \phi ( x ) \phi ( y )$ for all $x, y \in G$$ 
(Note that $xy$ on the left is formed using the group operation in $G$, whilst the product $\phi ( x ) \phi ( y )$ is formed using the group operation $H$.)
\end{definition}

% NLab:

\begin{definition}\label{def:def4}
Classically, a group is a monoid in which every element has an inverse (necessarily unique).
\end{definition}

We inquire the reader to pay attention to nuance and difference in presentation
that is normally ignored or taken for granted by the fluent mathematician, ask
which definitions feel better, and how the reader herself might present the
definition differently.

If one want to distill the meaning of each of these presentations, there is a
significant amount of subliminal interpretation happening very much analogous to
our innate lingusitic ussage. The inverse and identity are discarded, even
though they are necessary data when defning a group. The order of presentation
of information is inconsistent, as well as the choice to use symbolic or natural
language information. In Definition~\ref{def:def3}, the group operation is used
implicitly, and its clarification a side remark.

Details aside, these all mean the same thing - don't they?  This thesis seeks to provide an
abstract framework to determine whether two lingusitically nuanced presenations
mean the same thing via their syntactic transformations. Obviously these
meanings  are not resolvable in any kind of absolute sense, but at least from a
translational sense. These syntactic transformations come in two flavors : parsing and
linearization, and are natively handled by a Logical Framework (LF) for
specifying grammars : Grammatical Framework (GF).

We now show yet another definition of a group homomorphism formalized in the
Agda programming language:

\begin{code}[hide]%
\>[0]\AgdaComment{--\{-\# OPTIONS --cubical \#-\}}\<%
\\
\>[0]\AgdaSymbol{\{-\#}\AgdaSpace{}%
\AgdaKeyword{OPTIONS}\AgdaSpace{}%
\AgdaPragma{--cubical}\AgdaSpace{}%
\AgdaPragma{--no-import-sorts}\AgdaSpace{}%
\AgdaPragma{--safe}\AgdaSpace{}%
\AgdaSymbol{\#-\}}\<%
\\
%
\\[\AgdaEmptyExtraSkip]%
\>[0]\AgdaKeyword{module}\AgdaSpace{}%
\AgdaModule{monoid}\AgdaSpace{}%
\AgdaKeyword{where}\<%
\\
%
\\[\AgdaEmptyExtraSkip]%
\>[0]\AgdaKeyword{module}\AgdaSpace{}%
\AgdaModule{Namespace1}\AgdaSpace{}%
\AgdaKeyword{where}\<%
\\
%
\\[\AgdaEmptyExtraSkip]%
\>[0][@{}l@{\AgdaIndent{0}}]%
\>[2]\AgdaKeyword{open}\AgdaSpace{}%
\AgdaKeyword{import}\AgdaSpace{}%
\AgdaModule{Cubical.Foundations.Prelude}\<%
\\
%
\>[2]\AgdaKeyword{open}\AgdaSpace{}%
\AgdaKeyword{import}\AgdaSpace{}%
\AgdaModule{Cubical.Foundations.Equiv}\<%
\\
%
\>[2]\AgdaKeyword{open}\AgdaSpace{}%
\AgdaKeyword{import}\AgdaSpace{}%
\AgdaModule{Cubical.Foundations.Structure}\<%
\\
%
\>[2]\AgdaKeyword{open}\AgdaSpace{}%
\AgdaKeyword{import}\AgdaSpace{}%
\AgdaModule{Cubical.Algebra.Group.Base}\<%
\\
%
\>[2]\AgdaKeyword{open}\AgdaSpace{}%
\AgdaKeyword{import}\AgdaSpace{}%
\AgdaModule{Cubical.Data.Sigma}\<%
\\
%
\\[\AgdaEmptyExtraSkip]%
%
\>[2]\AgdaKeyword{private}\<%
\\
\>[2][@{}l@{\AgdaIndent{0}}]%
\>[4]\AgdaKeyword{variable}\<%
\\
\>[4][@{}l@{\AgdaIndent{0}}]%
\>[6]\AgdaGeneralizable{ℓ}\AgdaSpace{}%
\AgdaGeneralizable{ℓ'}\AgdaSpace{}%
\AgdaGeneralizable{ℓ''}\AgdaSpace{}%
\AgdaGeneralizable{ℓ'''}\AgdaSpace{}%
\AgdaSymbol{:}\AgdaSpace{}%
\AgdaPostulate{Level}\<%
\end{code}
\begin{code}%
%
\>[2]\AgdaFunction{isGroupHom}\AgdaSpace{}%
\AgdaSymbol{:}\AgdaSpace{}%
\AgdaSymbol{(}\AgdaBound{G}\AgdaSpace{}%
\AgdaSymbol{:}\AgdaSpace{}%
\AgdaFunction{Group}\AgdaSpace{}%
\AgdaSymbol{\{}\AgdaGeneralizable{ℓ}\AgdaSymbol{\})}\AgdaSpace{}%
\AgdaSymbol{(}\AgdaBound{H}\AgdaSpace{}%
\AgdaSymbol{:}\AgdaSpace{}%
\AgdaFunction{Group}\AgdaSpace{}%
\AgdaSymbol{\{}\AgdaGeneralizable{ℓ'}\AgdaSymbol{\})}\AgdaSpace{}%
\AgdaSymbol{(}\AgdaBound{f}\AgdaSpace{}%
\AgdaSymbol{:}\AgdaSpace{}%
\AgdaOperator{\AgdaFunction{⟨}}\AgdaSpace{}%
\AgdaBound{G}\AgdaSpace{}%
\AgdaOperator{\AgdaFunction{⟩}}\AgdaSpace{}%
\AgdaSymbol{→}\AgdaSpace{}%
\AgdaOperator{\AgdaFunction{⟨}}\AgdaSpace{}%
\AgdaBound{H}\AgdaSpace{}%
\AgdaOperator{\AgdaFunction{⟩}}\AgdaSymbol{)}\AgdaSpace{}%
\AgdaSymbol{→}\AgdaSpace{}%
\AgdaPrimitive{Type}\AgdaSpace{}%
\AgdaSymbol{\AgdaUnderscore{}}\<%
\\
%
\>[2]\AgdaFunction{isGroupHom}\AgdaSpace{}%
\AgdaBound{G}\AgdaSpace{}%
\AgdaBound{H}\AgdaSpace{}%
\AgdaBound{f}\AgdaSpace{}%
\AgdaSymbol{=}\AgdaSpace{}%
\AgdaSymbol{(}\AgdaBound{x}\AgdaSpace{}%
\AgdaBound{y}\AgdaSpace{}%
\AgdaSymbol{:}\AgdaSpace{}%
\AgdaOperator{\AgdaFunction{⟨}}\AgdaSpace{}%
\AgdaBound{G}\AgdaSpace{}%
\AgdaOperator{\AgdaFunction{⟩}}\AgdaSymbol{)}\AgdaSpace{}%
\AgdaSymbol{→}\AgdaSpace{}%
\AgdaBound{f}\AgdaSpace{}%
\AgdaSymbol{(}\AgdaBound{x}\AgdaSpace{}%
\AgdaOperator{\AgdaFunction{G.+}}\AgdaSpace{}%
\AgdaBound{y}\AgdaSymbol{)}\AgdaSpace{}%
\AgdaOperator{\AgdaFunction{≡}}\AgdaSpace{}%
\AgdaSymbol{(}\AgdaBound{f}\AgdaSpace{}%
\AgdaBound{x}\AgdaSpace{}%
\AgdaOperator{\AgdaFunction{H.+}}\AgdaSpace{}%
\AgdaBound{f}\AgdaSpace{}%
\AgdaBound{y}\AgdaSymbol{)}\AgdaSpace{}%
\AgdaKeyword{where}\<%
\\
\>[2][@{}l@{\AgdaIndent{0}}]%
\>[4]\AgdaKeyword{module}\AgdaSpace{}%
\AgdaModule{G}\AgdaSpace{}%
\AgdaSymbol{=}\AgdaSpace{}%
\AgdaModule{GroupStr}\AgdaSpace{}%
\AgdaSymbol{(}\AgdaField{snd}\AgdaSpace{}%
\AgdaBound{G}\AgdaSymbol{)}\<%
\\
%
\>[4]\AgdaKeyword{module}\AgdaSpace{}%
\AgdaModule{H}\AgdaSpace{}%
\AgdaSymbol{=}\AgdaSpace{}%
\AgdaModule{GroupStr}\AgdaSpace{}%
\AgdaSymbol{(}\AgdaField{snd}\AgdaSpace{}%
\AgdaBound{H}\AgdaSymbol{)}\<%
\\
%
\\[\AgdaEmptyExtraSkip]%
%
\>[2]\AgdaKeyword{record}\AgdaSpace{}%
\AgdaRecord{GroupHom}\AgdaSpace{}%
\AgdaSymbol{(}\AgdaBound{G}\AgdaSpace{}%
\AgdaSymbol{:}\AgdaSpace{}%
\AgdaFunction{Group}\AgdaSpace{}%
\AgdaSymbol{\{}\AgdaGeneralizable{ℓ}\AgdaSymbol{\})}\AgdaSpace{}%
\AgdaSymbol{(}\AgdaBound{H}\AgdaSpace{}%
\AgdaSymbol{:}\AgdaSpace{}%
\AgdaFunction{Group}\AgdaSpace{}%
\AgdaSymbol{\{}\AgdaGeneralizable{ℓ'}\AgdaSymbol{\})}\AgdaSpace{}%
\AgdaSymbol{:}\AgdaSpace{}%
\AgdaPrimitive{Type}\AgdaSpace{}%
\AgdaSymbol{(}\AgdaPrimitive{ℓ-max}\AgdaSpace{}%
\AgdaBound{ℓ}\AgdaSpace{}%
\AgdaBound{ℓ'}\AgdaSymbol{)}\AgdaSpace{}%
\AgdaKeyword{where}\<%
\\
\>[2][@{}l@{\AgdaIndent{0}}]%
\>[4]\AgdaKeyword{constructor}\AgdaSpace{}%
\AgdaInductiveConstructor{grouphom}\<%
\\
%
\\[\AgdaEmptyExtraSkip]%
%
\>[4]\AgdaKeyword{field}\<%
\\
\>[4][@{}l@{\AgdaIndent{0}}]%
\>[6]\AgdaField{fun}\AgdaSpace{}%
\AgdaSymbol{:}\AgdaSpace{}%
\AgdaOperator{\AgdaFunction{⟨}}\AgdaSpace{}%
\AgdaBound{G}\AgdaSpace{}%
\AgdaOperator{\AgdaFunction{⟩}}\AgdaSpace{}%
\AgdaSymbol{→}\AgdaSpace{}%
\AgdaOperator{\AgdaFunction{⟨}}\AgdaSpace{}%
\AgdaBound{H}\AgdaSpace{}%
\AgdaOperator{\AgdaFunction{⟩}}\<%
\\
%
\>[6]\AgdaField{isHom}\AgdaSpace{}%
\AgdaSymbol{:}\AgdaSpace{}%
\AgdaFunction{isGroupHom}\AgdaSpace{}%
\AgdaBound{G}\AgdaSpace{}%
\AgdaBound{H}\AgdaSpace{}%
\AgdaField{fun}\<%
\end{code}
This actually \emph{was} the Cubical Agda implementation of a group homomorphism
sometime around the end of 2020. We see that, while a mathematician might be
able to infer the meaning of some of the syntax, the use of levels, the
distinguising bewteen isGroupHom and GroupHom itself, and many other details
might obscure what's going on.

We provide the current, as of May 2021, definition via Cubical Agda. One may
witness a significant number of differences from the previous version :
concrete syntax differenes via changes in camel case, new uses of Group vs
GroupStr, as well as, most significantly, the identity and inverse preservation
data not appearing as corollaries, but part of the definition. Additionally, we
had to refactor the commented lines to those shown below to be compatible with
our outdated version of cubical.  These changes would not just be interesting
to look at from the author of the libraries's perspective, but also
syntactically.
\begin{code}%
%
\>[2]\AgdaKeyword{record}\AgdaSpace{}%
\AgdaRecord{IsGroupHom}\AgdaSpace{}%
\AgdaSymbol{\{}\AgdaBound{A}\AgdaSpace{}%
\AgdaSymbol{:}\AgdaSpace{}%
\AgdaPrimitive{Type}\AgdaSpace{}%
\AgdaGeneralizable{ℓ}\AgdaSymbol{\}}\AgdaSpace{}%
\AgdaSymbol{\{}\AgdaBound{B}\AgdaSpace{}%
\AgdaSymbol{:}\AgdaSpace{}%
\AgdaPrimitive{Type}\AgdaSpace{}%
\AgdaGeneralizable{ℓ'}\AgdaSymbol{\}}\<%
\\
\>[2][@{}l@{\AgdaIndent{0}}]%
\>[4]\AgdaSymbol{(}\AgdaBound{M}\AgdaSpace{}%
\AgdaSymbol{:}\AgdaSpace{}%
\AgdaRecord{GroupStr}\AgdaSpace{}%
\AgdaBound{A}\AgdaSymbol{)}\AgdaSpace{}%
\AgdaSymbol{(}\AgdaBound{f}\AgdaSpace{}%
\AgdaSymbol{:}\AgdaSpace{}%
\AgdaBound{A}\AgdaSpace{}%
\AgdaSymbol{→}\AgdaSpace{}%
\AgdaBound{B}\AgdaSymbol{)}\AgdaSpace{}%
\AgdaSymbol{(}\AgdaBound{N}\AgdaSpace{}%
\AgdaSymbol{:}\AgdaSpace{}%
\AgdaRecord{GroupStr}\AgdaSpace{}%
\AgdaBound{B}\AgdaSymbol{)}\<%
\\
%
\>[4]\AgdaSymbol{:}\AgdaSpace{}%
\AgdaPrimitive{Type}\AgdaSpace{}%
\AgdaSymbol{(}\AgdaPrimitive{ℓ-max}\AgdaSpace{}%
\AgdaBound{ℓ}\AgdaSpace{}%
\AgdaBound{ℓ'}\AgdaSymbol{)}\<%
\\
%
\>[4]\AgdaKeyword{where}\<%
\\
%
\\[\AgdaEmptyExtraSkip]%
%
\>[4]\AgdaComment{-- Shorter qualified names}\<%
\\
%
\>[4]\AgdaKeyword{private}\<%
\\
\>[4][@{}l@{\AgdaIndent{0}}]%
\>[6]\AgdaKeyword{module}\AgdaSpace{}%
\AgdaModule{M}\AgdaSpace{}%
\AgdaSymbol{=}\AgdaSpace{}%
\AgdaModule{GroupStr}\AgdaSpace{}%
\AgdaBound{M}\<%
\\
%
\>[6]\AgdaKeyword{module}\AgdaSpace{}%
\AgdaModule{N}\AgdaSpace{}%
\AgdaSymbol{=}\AgdaSpace{}%
\AgdaModule{GroupStr}\AgdaSpace{}%
\AgdaBound{N}\<%
\\
%
\\[\AgdaEmptyExtraSkip]%
%
\>[4]\AgdaKeyword{field}\<%
\\
\>[4][@{}l@{\AgdaIndent{0}}]%
\>[6]\AgdaField{pres·}\AgdaSpace{}%
\AgdaSymbol{:}\AgdaSpace{}%
\AgdaSymbol{(}\AgdaBound{x}\AgdaSpace{}%
\AgdaBound{y}\AgdaSpace{}%
\AgdaSymbol{:}\AgdaSpace{}%
\AgdaBound{A}\AgdaSymbol{)}\AgdaSpace{}%
\AgdaSymbol{→}\AgdaSpace{}%
\AgdaBound{f}\AgdaSpace{}%
\AgdaSymbol{(}\AgdaOperator{\AgdaFunction{M.\AgdaUnderscore{}+\AgdaUnderscore{}}}\AgdaSpace{}%
\AgdaBound{x}\AgdaSpace{}%
\AgdaBound{y}\AgdaSymbol{)}\AgdaSpace{}%
\AgdaOperator{\AgdaFunction{≡}}\AgdaSpace{}%
\AgdaSymbol{(}\AgdaOperator{\AgdaFunction{N.\AgdaUnderscore{}+\AgdaUnderscore{}}}\AgdaSpace{}%
\AgdaSymbol{(}\AgdaBound{f}\AgdaSpace{}%
\AgdaBound{x}\AgdaSymbol{)}\AgdaSpace{}%
\AgdaSymbol{(}\AgdaBound{f}\AgdaSpace{}%
\AgdaBound{y}\AgdaSymbol{))}\<%
\\
%
\>[6]\AgdaField{pres1}\AgdaSpace{}%
\AgdaSymbol{:}\AgdaSpace{}%
\AgdaBound{f}\AgdaSpace{}%
\AgdaFunction{M.0g}\AgdaSpace{}%
\AgdaOperator{\AgdaFunction{≡}}\AgdaSpace{}%
\AgdaFunction{N.0g}\<%
\\
%
\>[6]\AgdaField{presinv}\AgdaSpace{}%
\AgdaSymbol{:}\AgdaSpace{}%
\AgdaSymbol{(}\AgdaBound{x}\AgdaSpace{}%
\AgdaSymbol{:}\AgdaSpace{}%
\AgdaBound{A}\AgdaSymbol{)}\AgdaSpace{}%
\AgdaSymbol{→}\AgdaSpace{}%
\AgdaBound{f}\AgdaSpace{}%
\AgdaSymbol{(}\AgdaOperator{\AgdaFunction{M.-\AgdaUnderscore{}}}\AgdaSpace{}%
\AgdaBound{x}\AgdaSymbol{)}\AgdaSpace{}%
\AgdaOperator{\AgdaFunction{≡}}\AgdaSpace{}%
\AgdaOperator{\AgdaFunction{N.-\AgdaUnderscore{}}}\AgdaSpace{}%
\AgdaSymbol{(}\AgdaBound{f}\AgdaSpace{}%
\AgdaBound{x}\AgdaSymbol{)}\<%
\\
%
\>[6]\AgdaComment{-- pres· : (x y : A) → f (x M.· y) ≡ f x N.· f y}\<%
\\
%
\>[6]\AgdaComment{-- pres1 : f M.1g ≡ N.1g}\<%
\\
%
\>[6]\AgdaComment{-- presinv : (x : A) → f (M.inv x) ≡ N.inv (f x)}\<%
\\
%
\\[\AgdaEmptyExtraSkip]%
%
\>[2]\AgdaFunction{GroupHom'}\AgdaSpace{}%
\AgdaSymbol{:}\AgdaSpace{}%
\AgdaSymbol{(}\AgdaBound{G}\AgdaSpace{}%
\AgdaSymbol{:}\AgdaSpace{}%
\AgdaFunction{Group}\AgdaSpace{}%
\AgdaSymbol{\{}\AgdaGeneralizable{ℓ}\AgdaSymbol{\})}\AgdaSpace{}%
\AgdaSymbol{(}\AgdaBound{H}\AgdaSpace{}%
\AgdaSymbol{:}\AgdaSpace{}%
\AgdaFunction{Group}\AgdaSpace{}%
\AgdaSymbol{\{}\AgdaGeneralizable{ℓ'}\AgdaSymbol{\})}\AgdaSpace{}%
\AgdaSymbol{→}\AgdaSpace{}%
\AgdaPrimitive{Type}\AgdaSpace{}%
\AgdaSymbol{(}\AgdaPrimitive{ℓ-max}\AgdaSpace{}%
\AgdaGeneralizable{ℓ}\AgdaSpace{}%
\AgdaGeneralizable{ℓ'}\AgdaSymbol{)}\<%
\\
%
\>[2]\AgdaComment{-- GroupHom' : (G : Group ℓ) (H : Group ℓ') → Type (ℓ-max ℓ ℓ')}\<%
\\
%
\>[2]\AgdaFunction{GroupHom'}\AgdaSpace{}%
\AgdaBound{G}\AgdaSpace{}%
\AgdaBound{H}\AgdaSpace{}%
\AgdaSymbol{=}\AgdaSpace{}%
\AgdaFunction{Σ[}\AgdaSpace{}%
\AgdaBound{f}\AgdaSpace{}%
\AgdaFunction{∈}\AgdaSpace{}%
\AgdaSymbol{(}\AgdaBound{G}\AgdaSpace{}%
\AgdaSymbol{.}\AgdaField{fst}\AgdaSpace{}%
\AgdaSymbol{→}\AgdaSpace{}%
\AgdaBound{H}\AgdaSpace{}%
\AgdaSymbol{.}\AgdaField{fst}\AgdaSymbol{)}\AgdaSpace{}%
\AgdaFunction{]}\AgdaSpace{}%
\AgdaRecord{IsGroupHom}\AgdaSpace{}%
\AgdaSymbol{(}\AgdaBound{G}\AgdaSpace{}%
\AgdaSymbol{.}\AgdaField{snd}\AgdaSymbol{)}\AgdaSpace{}%
\AgdaBound{f}\AgdaSpace{}%
\AgdaSymbol{(}\AgdaBound{H}\AgdaSpace{}%
\AgdaSymbol{.}\AgdaField{snd}\AgdaSymbol{)}\<%
\end{code}

While the last two definitions may carry degree of comprehension to a programmer
or mathematician not exposed to Agda, it is certainly comprehensible to a
computer : that is, it typechecks on a computer where Cubical Agda is installed.
While GF is designed for multilingual syntactic transformations and is targeted
for natural language translation, it's underlying theory is largely based on
ideas from the compiler communities. A cousin of the BNF Converter (BNFC), GF is
fully capable of parsing programming languages like Agda! And while the Agda
definitions are just another concrete syntactic presentation of a group
homomorphism, they are distinct from the natural language presentations above in
that the colors indicate it has indeed type checked.

While this example may not exemplify the power of Agda's type-checker, it is of
considerable interest to many. The type-checker has merely assured us that
\term{GroupHom(')} are well-formed types - not that we have a canonical representation
of a group homomorphism. The type-checker is much more useful than is
immediately evident: it delegates the work of verifying that a proof is correct,
that is, the work of judging whether a term has a type, to the computer. While
it's of practical concern is immediate to any exploited grad student grading
papers late on a Sunday night, its theoretical concern has led to many recent
developments in modern mathematics. Thomas Hales solution to the Kepler
Conjecture was seen as unverifiable by those reviewing it, and this led to Hales
outsourcing the verification to Interactive Theorem Provers (ITPs) HOL Light and
Isabelle. This computer delegated verification phase led to many minor
corrections in the original proof which were never spotted due to human
oversight.

Fields medalist Vladimir Voevodsky had the experience of being told one day
his proof of the Milnor conjecture was fatally flawed. Although the leak in the
proof was patched, this experience of temporarily believing much of his life's
work invalidated led him to investigate proof assintants as a tool for future
thought. Indeed, this proof verification error was a key event that led to the
Univalent Foundations
Project~\cite{theunivalentfoundationsprogram-homotopytypetheory-2013}.

While Agda and other programming languages are capable of encoding definitions,
theorems, and proofs, they have so far seen little adoption. In some cases they
have been treated with suspicion and scorn by many mathematicians. This isn't
entirely unfounded : it's a lot of work to learn how to use Agda or Coq,
software updates may cause proofs to break, and the inevitable imperfections we
humans are prone to instilled in these tools . Besides, Martin-Löf Type Theory,
the constructive foundational project which underlies these proof assistants, is
often misunderstood by those who dogmatically accept the law of the excluded
middle as the word of God.

It should be noted, the constructivist rejects neither the law of the excluded
middle, nor ZFC. She merely observes them, and admits their handiness in certain
citations. Excluded middle is indeed a helpful tool as many mathematicians
may attest. The contention is that it should be avoided whenever possible -
proofs which don't rely on it, or it's corallary of proof by contradction, are
much more ameanable to formalization in systems with decideable type checking.
And ZFC, while serving the mathematicians of the early 20th century, is 
lacking when it comes to the higher dimensional structure of n-categories and
infinity groupoids.

What these theorem provers give the mathematician is confidence that her work
is correct, and even more importantly, that the work which she takes for
granted and references in her work is also correct. The task before us is then
one of religious conversion. And one doesn't undertake a conversion by simply
by preaching. Foundational details aside, this thesis is meant to provide a
blueprint for the syntactic reformation that must take place.  

We don't insist a mathematician relinquish the beautiful language she has
come to love in expressing her ideas.  Rather, it asks her to make a
hypothetical compromise
for the time being, and use a Controlled Natural Language (CNL) to develop her
work. In exchange she'll get the confidence that Agda provides. Not only that,
she'll be able to search through a library, to see who else has possibly
already postulated and proved her conjecture. A version of this grandiose vision is 
explored in The Formal Abstracts Project \cite{halesCNL}, and it should
practically motivate work.  

Practicalities aside, this work also attempts to offer a nuanced philosophical
perspective on the matter by exploring why translation of mathematical language,
despite it's seemingly structured form, is difficult. We note that the natural
language definitions of monoid differ in form, but also in pragmatic content.
How one expresses formalities in natural language is incredibly diverse, and
Definition~\ref{def:def4} as compared with the prior homomorphism definitions is
particularly poignant in demonstrating this. These differ very much in nature to
the Agda definitions - especially pragmatically. The differences between the Cubical
Agda definitions may be loosely called pragmatic, in the sense that the choice
of definitions may have downstream effects on readability, maintainability, modularity, and other
considerations when trying to write good code, in a burgeoning area known as proof engineering.

A pragmatic treatment of the language of mathematics is the golden egg if one
wishes to articulate the nuance in how the notions proposition, proof, and
judgment are understood by humans. Nonetheless, this problem is just now seeing
attention. We hope that the treatment of syntax in this thesis, while a long
ways away from giving a pragmatic account of mathematics, will help pave the way
there.


\section{Philosophical Perspectives}

\begin{displayquote}

...when it comes to understanding the power of mathematical language to guide our
thought and help us reason well, formal mathematical languages like the ones
used by interactive proof assistants provide informative models of informal
mathematical language. The formal languages underlying foundational frameworks
such as set theory and type theory were designed to provide an account of the
correct rules of mathematical reasoning, and, as Gödel observed, they do a
remarkably good job. But correctness isn’t everything: we want our mathematical
languages to enable us to reason efficiently and effectively as well. To that
end, we need not just accounts as to what makes a mathematical argument correct,
but also accounts of the structural features of our theorizing that help us
manage mathematical complexity.\cite{avigad2015mathematics}

\end{displayquote}

\subsection{Linguistic and Programming Language Abstractions}

The key development of this thesis is to explore the formal and informal
distinction of presenting mathematics as understood by mathematicians and computer
scientists by means of rule-based, syntax oriented machine translation.

Computational linguistics, particularly those in the tradition of type
theoretical semantics\cite{ranta1994type}, gives one a way of comparing natural
and programming languages. Type theoretical semantics it is concerned with the
semantics of natural language in the logical tradition of Montague, who
synthesized work in the shadows of Chomsky \cite{Chomsky57} and Frege
\cite{frege79}. This work ended up inspiring the GF system, a side effect of
which was to realize that machine translation was possible as a side effect of
this abstracted view of natural language semantics. Indeed, one such description
of GF is that it is a compiler tool applied to domain specific machine
translation. We may compare the ``compiler view" of PLs and the ``linguistics view"
of NLs, and interpolate this comparison to other general phenomenon in the
respective domains.

We will reference these programming language and linguistic abstraction ladders,
and after viewing \autoref{fig:N1}, the reader should examine this
comparison with her own knowledge and expertise in mind. These respective
ladders are perhaps the most important lens one should keep in mind while
reading this thesis. Importantly, we should observe that the PL dimension, the
left diagram, represents synthetic processes, those which we design, make
decisions about, and describe formally. Alternatively, the NL abstractions on
the right represent analytic observations. They are therefore are subject to
different, in some ways orthogonal, constraints.

The linguistic abstractions are subject to empirical observations and
constraints, and this diagram only serves as an atlas for the different
abstractions and relations between these abstractions, which may be subject to
modifications depending on the linguist or philosopher investigating such
matters. The PL abstractions as represented, while also an approximations,
serves as an actual high altitude blueprint for the design of programming
languages. While the devil is in the details and this view is greatly
simplified, the representation of PL design is unlikely to create angst in the
computer science communities. The linguistic abstractions are at the
intersection of many fascinating debates between linguists, and there is
certainly nothing close to any type of consensus among linguists which
linguistic abstractions, as well as their hierarchical arrangement, are more
practically useful, theoretically compelling, or empirically testable.


\begin{figure}
\centering
\begin{tikzcd}
Strings \ar[d,"Lexical\ Analysis"] \ar[dd,bend right=+90.0, swap,"Front\ End"]
&[5m]
\\ Lexemes \ar[d,"Parsing"] &[5em]
\\ ASTs \ar[d,"Type\ Checker"] &[5m]
\\ Typed\ ASTs
  \ar[dd, bend left, "Code\ Generator"] 
  \ar[dd, bend right, swap, "Interpreter"] &[5m]
\\ ...
\\ Target\ Language
\end{tikzcd}
\hspace{1cm}
\begin{tikzcd}
  Phonemes \arrow[d, "Morhphophonological
  \\ Anaylsis" description]
  \\ Morphemes \arrow[d, "Parse"]
  \\ \{\ Syntactic\ Representation\ \} \arrow[d, "Montague"', bend right=49]
    \arrow[d, "Ranta", bend left=49] \arrow[d, "..." description]
  \\ {\{\ STLC,\ ...\ ,\ DTT\ \}} \arrow[d, "?" description]
  \\ {\{\ Nearal Encoding\ ,\ ...\ Internal\ Language\ \}} \arrow[d, "?" description]
  \\ Phonemes
\end{tikzcd}
\caption{PL (left) and NL (right) Abstraction Ladders} \label{fig:N1}
\end{figure}


There are also many relevant concerns not addressed in either abstraction chain
that are necessary to give a more comprehsive snapshot. For instance, we may
consider intrinsic and extrensic abstractions that diverge from the idealized
picture. In PL extrensic domain, we can inquire about 

\begin{itemize}[noitemsep]

\item systems with multiple interactive programming language 
\item how the programming languages behave with respect to given programs
\item embedding programming languages into one another

\end{itemize}

Alternatively, intrinsic to a given PL, there picture is also not so clear.
Agda, for example, requires the evaluation of terms during typechecking. It is
implemented with 4.5 different stages between the syntax written by the
programmers and the ``fully reflected Abstract Syntax Tree (AST)" \cite{andreasEmail}. But this
example is perhaps an outlier, because Agda's type-checker is so powerful that
the design, implemenation, and use of Agda revolves around it,
(which, ironically, is already called during the parsing phase). It is not
anticipated that floating point computation, for instance, would ever be
considered when implementing new features of Agda, at least not for the
foreseeable future. Indeed, the ways Agda represents ASTs were an obstacle
encountered doing this work, because deciding which parsing stage one should connect
to the Portable Grammar Format (PGF) embedding is nontrivial.

% \begin{displayquote}
% \begin{enumerate}
% \item unicode: before parsing
% \item Concrete: after happy parsing Parser/Parser.y that does a little desugaring already---[but ideally shouldn't].  Expressions are not parsed yet.
% \item Concrete.Definitions: the "nice" syntax after the nicifier, preparing for the scope checker
% \item Abstract: after scope checking, mostly desugared, expressions parsed
% \item Internal: after type checking, fully desugared
% \item Reflected: quoted from Internal
% \begin{end}
% \end{displayquote}


\begin{figure}
\centering
\begin{tikzcd}
Strings \ar[r,"Lexical\ Analysis"] \ar[rr,bend right,"GF\ Parser"'] &[10em] Lexemes
\ar[r,"Parsing"] &[10em] ASTs \ar[ll,bend right, "GF\ Linearization"] 
\end{tikzcd}
\caption{GF in a nutshell} \label{fig:N2}
\end{figure}

Let's zoom in a little and observe the so-called front-end part of the compiler.
Displayed in \autoref{fig:N2} is the highest possible overview of GF. This is a
deceptively simple depiction of such a powerful and intricate system. What makes
GF so compelling is its ability to translate between inductively defined
languages that type theorists specify and relatively expressive fragments of
natural languages, via the composition of GF's parsing and linearization
capabilities. It is in some sense the attempt to overlay the abstraction ladders
at the syntactic level and semantic led to this development.

For natural language, some intrinsic properties might take place, if one
chooses, at the neurological level, where one somehow can contrast the internal
language (i-language) with the mechanism of externalization (generally speech) as proposed by
Chomsky \cite{Chomsky1995}. Extrinsic to the linguistic abstractions depicted, pragmatics is
absent.
.
The point is to recognize their are stark differences between natural languages
and programming languages which are even more apparent when one gets to certain
abstractions. Classifying both programming languages as
languages is best read as an incomplete (and even sometimes contradictory)
metaphor, due to perceived similarities (of which their are ample).

Nonetheless, the point of this thesis is to take a crack at that exact question
: how can one compare programming and natural languages, in the sense that a
natural language, when restricted to a small enough (and presumably
well-behaved) domain, behaves as a programming language. Simultaneously, we
probe the topic of Natural Language Generation (NLG). Given a
logic or type system with some theory inside (say arithmetic over the naturals),
how do we not just find a natural language representation which interprets our
expressions, but also does so in a way that is linguistically coherent in a
sense that a competent speaker can make sense of it in a facile way.

The specific linguistic domain we focus on, that of mathematics, is a particular
sweet spot at the intersection of these natural and formal language spaces. It
should be noted that this problem, that of translating between \emph{formal} and
\emph{informal} mathematics as stated, is both vague and difficult. It is
difficult in both the practical sense, that it may be either of infeasible
complexity or even perhaps undecidable, but it is also difficult in the
philosophical sense. One may entertain the prospect of syntactically translated
mathematics may a priori may deflate its effectiveness or meaningfulness. Like
all collective human endeavors, mathematics is a historical construction - that
is, its conventions, notations, understanding, methodologies, and means of
publication and distribution have all been in a constant flux. There is no
consensus on what mathematics is, how it is to be done, and most relevant for
this treatise, how it is to be expressed.

Historically, mathematics has been filtered of natural language artifacts,
culminating in some sense with Frege's development of a formal proof. A
mathematician often never sees a formal proof as it is treated in Logic and Type
Theory. We hope this work helps with a new foundational mentality, whereby we
try to bring natural language back into mathematics in a controlled way, or at
least to bridge the gap between our technologies, specifically injecting ITPs
into a mathematicians toolbox.


We present a sketch of the difference of this so-called formal/informal
distinction. Mathematics, that is mathematical constructions like numbers and
geometrical figures, arose out of ad-hoc needs as humans cultures grew and
evolved over the millennia. Indeed, just like many of the most interesting human
developments of which there is a sparsely documented record until relatively
recently, it is likely to remain a mystery what the long historical arc of
mathematics could have looked like in the context of human evolution. And while
mathematical intuitions precede mathematical constructions (the spherical planet
precedes the human use of a ruler compass construction to generate a circle), we
should take it as a starting point that mathematics arises naturally out of our
linguistic capacity. This may very well not be the case, or at least not
universally so, but it is impossible to imagine humans developing mathematical
constructions elaborating anything particularly general without linguistic
faculties. Despite whatever empirical or philosophical dispute one takes with
this linguistic view of mathematical abilities, we seek to make a first order
approximation of our linguistic view for the sake of this work. The discussion around
mathematics relation to linguistics generally, regardless of the stance
one takes, should benefit from this work.

\subsection{Formalization and Informalization}

Formalization is the process of taking an informal piece of natural language
mathematics, embedding it in into a theorem prover, constructing a model,
and working with types instead of sets. This often requires significant amounts of
work. We note some interesting artifacts about a piece of mathematics
being formalized:

\begin{itemize}

\item it may be formalized differently by two different people in many different ways
\item it may have to be modified, to include hidden lemmas, to correct of an
  error, or other bureaucratic obstacles
\item it may not type check, and only be presumed hypothetically to be 'a
  correct formalization' given evidence 

\end{itemize}

Informalization, on the other hand is a process of taking a piece formal syntax, and turning it into a natural
language utterance, along with commentary motivating and or relating it to other
mathematics. It is a clarification of the meaning of a piece of
code, suppressing certain details and sometimes
redundantly reiterating other details. In figure \autoref{fig:N3} we offer a few
dimensions of comparison.

\begin{figure}
\centering
\begin{tabular}{|c|c|c|} \hline
  Category & Formal Proof & Informal Proof \\ \hline
  Audience & Agda (and Human) & Human \\ \hline
  Translation & Compiler & Human \\ \hline
  Objectivity & Objective & Subjective \\ \hline % not always true
  Historical & 20th Century & <= Euclid \\ \hline
  Orientation & Syntax & Semantics \\ \hline
  Inferability & Complete & Domain Expertise Necessary \\ \hline
  Verification & PL Designer & Human \\ \hline
  Ambiguity & Unambiguous & Ambiguous \\ \hline

\end{tabular}
\caption{Informal and Formal Proofs} \label{fig:N3}
\end{figure}

Mathematicians working in either direction know this is a respectable task,
often leading to new methods, abstractions, and research altogether. And just as
any type of machine translation, rule-based or statistical, on Virginia Woolf
novel or Emily Dickinson poem from English to Mandarin would be 
absurd, so-to would the pretense that the methods we explore here using GF
could actually match the competence of mathematicians translating work between a
computer a book. Despite the futility of surpassing a mathematician at proof
translation, it shouldn't deter those so inspired to try.

\subsection{Syntactic Completeness and Semantic Adequacy}

The GF pipeline, that of bidirectional translation through an intermediary
abstract syntax representation, has two fundamental criteria that must be
assessed for one to judge the success of an approach over both formalization and
informalization.

The first criterion mentioned above, which we'll call \emph{syntactic
  completeness}, says that a term either type-checks, or some natural language
form can be deterministically transformed to a term that does type-check.

It asks the following : given an utterance or natural language expression that a
mathematician might understand, does the GF grammar emit a well-formed syntactic
expression in the target logic or programming language? The saying ``grammars
leak", can be transposed to say (NL) ``proofs leak" in that they are certain to
contain omissions.

This problem of syntactically complete mathematics is certain to be infeasible
in many cases - a mathematician might not be able to reconstruct the unstated
syntactic details of a proof in an discipline outside her expertise, it is at
worthy pursuit to ask why it is so difficult! Additionally, certain inferable
details may also detract from the natural language reading rather than assist.
Perhaps most importantly, one does not know a priori that the generated
expression in the logic has its intended meaning, other than through some meta
verification procedure.

Conversely, given a well formed syntactic expression in, for instance, Agda, one
can ask if the resulting English expression generated by GF is
\emph{semantically adequate}.

This notion of semantic adequacy is also delicate, as mathematicians themselves
may dispute, for instance, the proof of a given proposition or the correct
definition of some notion. However, if it is doubtful that there would be many
mathematicians who would not understand some standard theorem statement and
proof in an arbitrary introductory analysis text, even if one may dispute it's
presentation, clarity, pedagogy, or take issue with other details. Whether one
asks that semantic adequacy means some kind of sociological consensus among
those with relevant expertise, or a more relaxed criterion that some expert
herself understands the argument, a dubious perspective in scientific circles,
semantic adequacy should appease at least one and potentially more
mathematicians.

An example of a syntactically complete but semantically
inadequate statement which one of our grammars parses is ``is it the case that
the sum of 3 and the sum of 4 and 10 is prime and 9999 is odd". Alternatively,
most mathematical proofs in most mathematical papers are most likely
syntactically incomplete - anyone interested in formalizing a piece of
mathematics from some arbitrary (even simple) resource will learn this the hard
way.

\begin{figure}[H]
\centering
\begin{tikzcd}
Syntactically\ Complete \ar[r,"Informalization"] &[10em]
Semantically\ Coherent \ar[l,bend right, "Formalization"] 
\end{tikzcd}
\caption{Formal and Informal Mathematics} \label{fig:N4}
\end{figure}

We introduce these terms, syntactic completeness and semantic adequacy to
highlight perspectives and insights that seems to underlie the biggest
differences between informal and formal mathematics, as is show in
\autoref{fig:N4}. We claim that mathematics, as done via a theorem prover, is a
syntax oriented endeavor, whereas mathematics, as practiced by mathematicians,
prioritizes semantic understanding. Developing a system which is able to
formalize and informalize utterances which preserve syntactic completeness and
semantic adequacy, respectively, is probably infeasible. Even introducing
objective criteria to really judge these definitions is likely to be infeasible.

This perspective represents an observation and is not intended to judge whether
the syntactic or semantic perspective on mathematics is better - there is a
dialectical phenomena between the two. Let's highlight some advantages both
provide, and try to distinguish more precisely what a syntactic and semantic
perspective may be. 

When the Agda user builds her proof, she is outsourcing much of the bookkeeping
to the type-checker. This isn't purely a mechanical process though, she often
does have to think, how her definitions will interact with downstream programs,
as well as whether they are even sensible to begin with (i.e. does this have a
proof). The syntactic side is expressly clear from the readers perspective as
well. If Agda proofs were semantically coherent, one would only need to look at
code, with perhaps a few occasional remarks about various intentions and
conclusions, to understand the mathematics being expressed. Yet, papers are
often written exclusively in Latex, where Agda proofs have to be reverse
engineered, preserving only semantic details and forsaking syntactic nuance.


Oftentimes the code is kept in the appendix so as to provide a complete
syntactic blueprint. But the act of writing an Agda proof and reading it is
often orthogonal, as the term shadows the application of typing rules
which enable its construction. The construction of the proof is
entirely engaged with the types, whereas the human witness of a large term is
either lost as to why it fulfills the typing judgment, she has to reexamine
parts of the proof reasoning in her head or perhaps, try to rebuild 
interactively with Agda's help.

Even in cases where Agda code is included in a paper, it is most often the types
which are emphasized and produced. Complex proof terms are seldom to be read on
their own terms. The natural language description and commentary is still
largely necessary to convey whatever results, regardless if the Agda code is
self-contained. And while literate Agda is some type of bridge, it is still the
commentary which in some sense unfolds the code and ultimately makes the Agda
code legible.

This is particularly pronounced in the Coq programming language, where proof
terms are built using Ltac, which can be seen as some kind of imperative
syntactic metaprogramming over the core language, Gallina. The user rarely sees
the internal proof tree that one becomes familiar with in Agda. The tactics are
not typed, often feel very adhoc, and tacticals, sequences of tactics, may carry
very little semantic value (or even possibly muddy one's understanding when
reading proofs with unknown tactics). Indeed, since Ltac isn't itself typed, it
often descends into the sorrows of so-called untyped languages (which are really
uni-typed), and there are recent attempts to change this \cite{mtac2}
\cite{ltac2}. From our perspective, the use of tactics is an additional
syntactic obfuscation of what a proof should look like from the mathematicians
perspective - and it is important to attempt to remedy this is. Alecytron is one
impressive development in giving Coq proofs more readability through a
interactive back-end which shows the proof state, and offers other semantically
appealing models like interactive graphics \cite{coqAlec}. This kind of system
could and should inspire other proof assistants to allow for experimentation
with syntactic alternative to linear code.

Tactics obviously have their uses, and sometimes enhance high level proof
understanding, as tactics like \emph{ring} or \emph{omega} often save the reader overhead
of parsing pedantic and uninformative details. For certain proofs,
especially those involving many hundreds of cases, the metaprogramming
facilities actually give one exclusive advantages not offered to the classical
mathematician using pen and paper. Nonetheless, the dependent type theorist's
dream that all mathematicians begin using theorem provers in their everyday work
is largely just a dream, and with relatively little mainstream adoption by
mathematicians, the future is all but clear.

Mathematicians may indeed like some of the facilities theorem provers provide,
but ultimately, they may not see that as the "essence" of what they are doing.
What is this essence? We will try to shine a small light on perhaps the most
fundamental question in mathematics.

\subsection{What is a proof?}

\begin{displayquote}

A proof is what makes a judgment evident \cite{mlMeanings}.

\end{displayquote}

The proofs of Agda, and any programming language supporting proof development,
are \emph{formal proofs}. Formal proofs have no holes, and while there may very
well be bugs in the underlying technologies supporting these proofs, formal
proofs are seen as some kind of immutable form of data. One could say they
provide \emph{objective evidence} for judgments, which themselves are objective
entities when encoded on a computer. What we call formal proofs might provide a
science fiction writer an interesting thought experiment as regards
communicating mathematics with an alien species incapable of understanding our
language otherwise. Formal proofs, however, certainly don't appease all
mathematicians writing for other mathematicians.

Mathematics, and the act of proving theorems, according to Brouwer is a social
process. And because social processes between humans involve our linguistic
faculties, a we hope to elucidates what a proof with a simplified description.
Suppose we have two humans, $h_1$ and $h_2$. If $h_1$ claims to have a proof
$p_1$, and elaborates it to $p_2$ who claims she can either verify $p_1$ or
reproduce and re-articulate it via $p_1'$, such that $h1$ and $h2$ agree that
$p1$ and $p_1'$ are equivalent, then they have discovered some mathematics. In
fact, in this guise mathematics, can be viewed as a science, even if in fact it
is constructed instead of discovered.

An apt comparison is to see the mathematician is architect, whereas the computer
scientist responsible for formalizing the mathematics is an engineer. The
mathematics is the building which, like all human endeavors, is created via
resources and labor of many people. The role of the architect is to envision the
facade, the exterior layer directly perceived by others, giving a building its
character, purpose, and function. The engineer is on the other hand, tasked with
assuring the building gets built, doesn't collapse, and functions with many
implicit features which the user of the building may not notice : the running
water, insulation, and electricity. Whereas the architect is responsible for the
building's \emph{specification}, the engineer is tasked with its
\emph{implementation}.

We claim informal proofs are specifications and formal proofs are
implementations. Additionally, via the propositions-as-types interpretation, one
may see a logic as a specification and a PL as an implementation of a given
logic, often with multiple ways of assigning terms to a given type. Therefore,
one may see the mathematician ambiently developing a theorem in classical first
order logic as providing a specification of a proposition in that language,
whereas a given implementation of that theorem in Agda could be viewed as a
model construction of some NL fragment, where truth in the model would
correspond to termination of type-checking. Alternatively, during the
informalization process, two different authors may suppress different details,
or phrase a given utterance entirely differently, possibly leading to two
different, but possibly similar proofs. Extrapolating our analogy, the same two
architects given the same engineering plans could produce two entirely different
looking and functioning buildings. Oftentimes though, it is the architect who
has the vision, and the engineers who end up implementing the architects art.

We also briefly explore the difference between the mathematician and the
physicist. The physicist will often say under her breath to a class,
``don't tell anyone in the math department I'm doing this" when swapping an
integral and a sum or other loose but effective tricks in her blackboard
calculations. While there is an implicit assumption that there are theorems in
analysis which may justify these calculations, it is not the physicist's objective
to be as rigorous as the mathematician. This is because the physicist is not
using the mathematics as a syntactic mechanism to reflect the semantic domain of
particles, energy, and other physical processes which the mathematics in physics
serves to describe. The mathematician using Agda, needing to make syntactically
complete arguments, needs to be obsessed with the details - whereas the
``pen and paper" mathematician would need be reluctant to carry out all the
excruciating syntactic details for sake of semantic clarity.

There isn't a natural notion of equivalence between informal and formal proofs,
but rather, loosely, some kind of adjunction between these two sets. We note the
fact that the ``acceptable" Natural language utterances aren't inductively
defined. This precludes us from actually constructing a canonical mathematical
model of formal/informal relationship, but we most certainly believe that if the
GF perspective of translation is used, there can at least be an approximation of
what a model may look like. It is our contention that the linguist interested in
the language of mathematics should perhaps be seen as a scientist, whose point
is to contribute basic ideas and insights from which the architects and
engineers can use to inform their designs.


Mathematicians seek model independence in their results (i.e., they don't need a
direct encoding of Fermat's last theorem in set theory in order to trust its
validity). This is one possible reason why there is so much reluctance to adopt
proof assistant, because the implementation of a result in Coq, Agda, or HOL4
may lead to many permutations of the same result, each presumably representing
the same piece of knowledge. It's also noted a proof doesn't obey the same
universality that it does when it's on paper or verbalized - that Agda 2.6.2,
and its standard library, when updated in the future, may ``break proofs", as
was seen in the introduction. While this is a unanimous problem with all
software, we believe the GF approach offers at least a vision of not only
linguistic, but also foundation agnosticism with respect to mathematics.

This thesis examines not just a practical problem, but touches many deep issues
in some space in the intersection of the foundations, of mathematics, logic,
computer science, and their relations studied via linguistic formalisms. These
subjects, and their various relations, are the subject of countless hours of
work and consideration by many great minds. We barely scratches the surface of a
few of these developments, but it nonetheless, it is hoped, provides a
nontrivial perspective at many important issues.

Recapitulating much of what was said, we hope that the following questions may
have a new perspective :

\begin{itemize}[noitemsep]
\item What are mathematical objects?
\item How do their encodings in different foundational formalisms affect their
  interpretations?
\item How does is mathematics develop as a social process?
\item How does what mathematics is and how it is done rely on given technologies
  of a given historical era?
\end{itemize}

While various branches of linguistics have seen rapid evolution due to, in large
part, their adoption of mathematical tools, the dual application of linguistic
tools to mathematics is quite sparse and open terrain. We hope the reader can
walk away with an new appreciation to some of these questions and topics after
reading this. These nuances we will not explore here, but shall be further
elaborated in the future and and more importantly, hopefully inspire other
readers to respond accordingly.

Although not given in specific detail, the view of what mathematics is, in both
a philosophical and mathematical sense, as well as from the view of what a
foundational perspective, requires deep consideration in its relation to
linguistics. And while this work is perhaps just a finer grain of sandpaper on
an incomplete and primordial marble sculpture, it is hoped that the sculptor's
own reflection is a little bit more clear after we polish it here.

\subsection{What is a proof revisited}

\begin{displayquote}
Though philosophical discussion of visual thinking in mathematics has
concentrated on its role in proof, visual thinking may be more valuable for
discovery than proof \cite{sep-epistemology-visual-thinking}
\end{displayquote}

As an addendum to asking such a presumably simple question in the previous
section, we briefly address the one particular oversimplification which was
made. We briefly touch on what isn't just syntactic about mathematics, namely
so-called ``Proofs without Words" \cite{proofWW} and other diagrammatic and visual
reasoning tools generally. Because our work focuses on syntax, and is not
generalized to other mathematical tools, we hope one considers this as well when
pondering the language of mathematics.

The role of visualization in programming, logic, and mathematics generally offers
an abundance of contrast to syntactically oriented alphanumeric alphabets, i.e.
strings of symbols, which we discuss here. Although the trees in GF are visual,
they are of intermediary form between strings in different languages, and
therefore the type of syntax we're discussing here is strings, we hope a brief
exploration of alternatives for concrete syntax will be fruitful. Targeting
latex via GF for instance, is a small step in this direction.

Graphical Programming languages facilitating diagrammatic programming are one
instance of a nonlinear syntax which would prove tricky but possible to
implement via GF.  Additionally, Globular, which allows one to
carry out higher categorical constructions via globular sets is an interesting
case study for a graphical programming language which is designed for 
theorem proving \cite{Bar2016GlobularAO}.
Additionally, Alecytron  supports basic data structure
visualization, like red-black trees which carry semantic content less easy in a
string based-setting \cite{coqAlec}.

Visualization are ubiquitous in contemporary mathematics, whether it be
analytic functions, knots, diagram chases in category theory, and a myriad of
other visual tools which both assist understanding and inform our syntactic
descriptions. We find these languages appealing because of their focus on a different kind of
internal semantic sensation. The diagrammatic languages for monoidal categories, for example, also allow
interpretations of formal proofs via topological deformations, and they have
given semantic representations to various graphical languages like circuit
diagrams and petri nets \cite{fong2016algebra}.

We also note that, while programming languages whose visual syntax evaluates to
strings, means that all diagrams can in some sense be encoded in more
traditional syntax, this is only for the computers sake - the human may consume
the diagram as an abstract entity other than a string. There are often words to
describe, but not to give visual intuition to many of the mathematical ideas we
grasp. There are also, famously blind mathematicians who work in topology,
geometry, and analysis \cite{2002TheWO}. Bernard Morin, blinded at a young age,
was a topologist who discovered the first eversion of a sphere by using clay
models which were then diagrammatically transcribed by a colleague on the board.
This is a remarkable use of mathematical tools PL researchers cannot  yet
handle, and warrants careful consideration of what the
boundaries of proof assistants are capable of in terms of giving mathematicians
more tangible constructions.

For if there is one message one should take away from this thesis, it is that
there needs to be a coming to terms in the mathematics and TT communities, of
the difference between \emph{a proof}, both formal and informal, and the
\emph{the understanding of a proof}. The first is a mathematical judgment where
one supplies evidence, via the form of a term that Agda can type-check and
verify. A NL proof can be reviewed by a human. The understanding of a proof,
however, is not done by anything but a human. And this internal understanding
and processing of mathematical information, what I'll tongue-and-cheek call
i-mathematics, with its externalization facilities being our main concerns in
this thesis, requires much more work by future scholars.

\section{Previous Work}

There is a story that at some point, Göran Sundholm and Per Martin-Löf were
sitting at a dinner table, dicsussing various questions of interest to the
respective scholars, and Sundholm presented Martin-Löf with the problem of
Donkey Sentences in natural language semantics, those analogous 'Every man who
owns a donkey beats it'. This had been puzzling to those in the Montague
tradition, whereby higher order logic didn't provide facile ways of interpreting
these sentences. Martin-Löf apparently then, using his dependent type
constructors, provided an interpretation of the donkey sentence on the back of
the napkin. This is perhaps the genesis of dependent type theory in natural
language semantics. The research program was thereafter taken up by Martin-Löf's
student Aarne Ranta \cite{ranta1994type}, bled into the development of GF, and
has now in some sense led to this current work.

The prior exploration of these interleaving subjects is vast, and we can only
sample the available literature here. Indeed, there are so many approaches that
this work should be seen in a small (but important) case in the context of a
deep and broad literature \cite{surveyLang}. Acquiring expertise in such a
breadth of work is outside the scope of this thesis. Our approach, using
GF ASTs as a basis language for Mathematics and the logic the mathematical
objects are described in, is both distinct but has many roots and
interconnections with the remaining literature. The success of finding a
suitable language for mathematics will obviously require a comparative analysis
of the strengths and weaknesses in the goals in such a vast bibliography. 
 How the GF approach compares with this long merits careful consideration and
 future work.

It will function of our purpose, constrained by the limited scope of this work,
to focus on a few important resources.

\subsection{Ranta}

The initial considerations of Ranta were both oriented towards the language of
mathematics \cite{ranta93}, as well as purely linguistic concerns
\cite{ranta1994type}. In the treatise, Ranta explored not just the many avenues
to describe NL semantic phenomena with Dependent Types, but, after concentrating
on a linguistic analysis, he also proposed a primitive way of parsing and
sugaring these dependently typed interpretations of utterances into the strings
themselves - introducing the common nouns as types idea which has been since
seen great interest from both type theoretic and linguistic communities
\cite{luoCNs}. Therefore, if we interpret the set of men and the set of donkeys
as types, e.g. we judge $\vdash man \; {:} {\rm type}$ and $\vdash donkey \; {:}
{\rm type}$ where ${\rm type}$ really denotes a universe, and ditransitive verbs
``owns'' and ``beats'' as predicates, or dependent types over the CN types, i.e.
$\vdash owns \; {:} man \rightarrow donkey \rightarrow {\rm type}$ we can
interpret the sentence ``every man who owns a donkey beats it'' in DTT via the
following judgment :

\[\Pi z : (\Sigma x : man. \; \Sigma y : donkey. \; owns(x,y)). \; beats(\pi_1z,\pi_1(\pi_2z))\]

We note that the natural language quantifiers, which were largely the subject of
Montague's original investigations \cite{Montague1973}, find a natural
interpretation as the dependent product and sum types, $\Pi$ and $\Sigma$,
respectively. As type theory is constructive, and requires explicit witnesses
for claims, we admit the behavior following semantic interpretation : given a
man $m$, a donkey $d$ and evidence $m-owns-d$ that the man owns the donkey, we
can supply, via the term of the above type applied to our own tripple
$(m,d,m-owns-d)$ , evidence that the man beats the donkey, $beats(m,d)$ via
$pi_1$ and $pi_2$, the projections, or $\Sigma$ eliminators.

In the final chapter of \cite{ranta1994type}, $Sugaring and Parsing$, Ranta
explores the explicit relation, and of translation between the above logical
form and the string, where he presents a GF predecessor in the Alfa proof
assistant, itself a predecessor of Agda. To accomplish this translation he
introduces an intermediary , a functional phrase structure tree, which later
becomes the basis for GFs abstract syntax.  What is referred to as ``sugaring''
later changes to ``linearization''.

Soon thereafter, GF became a fully realized vision, with better and more
expressive parsing algorithms \cite{ljunglof2004expressivity} developed in
Göteborg allowed for sugaring that can largely accommodate morphological
features of the target natural language \cite{rantaForsberg}, the translation
between the functional phrase structure (ASTs) and strings \cite{ranta_2004}.

Interestingly, the functions that were called $ambiguation : MLTT \to \{Phrase\
Structure\}$ and $interpretation : \{Phrase Structure\} \to MLTT$ were absorbed
into GF by providing dependently typed ASTs, which allows GF not just to parse
syntactic strings, but only parse semantically well formed, or meaningful
strings. Although this feature was in some sense the genesis that allowed GF to
implement the lingusitic ideas from the book \cite{rantaTT}, it has remained
relatively limited in terms of actual GF programmers using it in their day to
day work. Nonetheless, it was intriguing enough to investigate briefly during
the course of this work as one can implement a programming language grammar that
only accepts well typed programs, at least as far as they can be encoded via
GF's dependent types \cite{warrickHarper}. Although GF isn't designed with
TypeChecking in mind explicity, it would be very interesting to apply GF
dependent types in the more advanced programming languages to filter parses of
meaningless strings.

While the semantics of natural language in MLTT is relevant historically, it is
not the focus of this thesis. Its relevance comes from the fact that all these
ideas were circulating in the same circles - that is, Ranta's writings on the
language of mathematics, his approach to NL semantics, and their confluence
among other things, with the development of GF. This led to the development of a
natural language layer to Alfa \cite{alfaGF}, which in some sense can be seen as
a direct predecessor to this work. In some sense, the scope of work seeks to
recapitulate what was already done in 1998 - but this was prior to both GF's
completion, and Alfa's hard fork to Agda.

\subsubsection{Prior GF Formalizations}

Prior to the grammars explored thin this thesis, Ranta produced two main results
\cite{rantaLog} \cite{aarneHott}. These are incredibly important precedents in
this approach to proof translation, and serve as important comparative work for
which this work responds. In \cite{rantaLog}, Ranta designed a grammar which
allowed for predicate logic with domain specific lexicon supporting mathematical
theories on top of the logic like geometry or arithmetic. The the syntax was
both meant to be relatively complete, so that typical logical utterances of
interest could be accommodated, as well as relatively non-trivial linguistic
nuance including lists of terms, predicates, and propositions, in-situ and
bounded quantification, and multiple forms of constructing more syntactically
nuanced predicates. The syntactic nuance captured in this work was by means of an
extended grammar, via a Portable Grammar Format (PGF), on top of the minimal,
core logical formalism.

One could translate from the core and extended via a denotational semantics
approach. The tree representing the \emph{syntactically complete} phrase ``for
all natural numbers x, x is even or x is odd" would be evaluated to a tree which
linearizes to the \emph{semantically adequate} phrase ``every natural number is
even or odd". In the opposite direction, the desugaring of a logically informal
statement into something linguistically idiomatic is also accomplished. In some
sense, this grammar serves as a case study for what this thesis is trying to do.
However, we note that the core logic only supports propositions without proofs -
it is not a type theory with terms. This means that we are being slightly
abusive to our terms, as the formal/informal translation is taking place is at
the PGF level. The GF translation between concrete syntaxes supports multiple
NLs, but the syntactic completeness has no mechanism of verification via Agda's
type checker. Additionally, the domain of arithmetic is an important case study,
but scaling this grammar (or any other, for that matter) to allow for
\emph{semantic adequacy} of real mathematics is still far away, or as Ranta
concedes, ``it seems that text generation involves undecidable optimization
problems that have no ultimate automatic solution." It would be interesting to
further extend this grammar with both terms and an Agda-like concrete syntax. 

In 2014, Ranta gave an unpublished talk at the Stockholm Mathematics seminar
\cite{aarneHott}. Fortunately the code is available, although many of the design
choices aren't documented in the grammar. This project aimed to provide a
translation like the one desired in our current work, but it took a real piece
of mathematics text as the main influence on the design of the Abstract syntax.

This work took a page of text from Peter Aczel's book which more or less goes
over standard HoTT definitions and theorems, and allows the translation of the
latex to a pidgin logical language. The central motivation of this grammar was
to capture, entirely ``real" natural language mathematics, i.e. that which was
written for the mathematician. Therefore, it isn't reminiscent of the slender
abstract syntax the type theorist adores, and sacrificed ``syntactic
completeness" for ``semantic adequacy". This means that the abstract syntax is
much larger and very expressive, but it no longer becomes easy to reason about
and additionally quite ad-hoc. Another defect is that this grammar
overgenerates, so producing a unique parse from the PL side will become tricky.
Nonetheless, this means that it's presumably possible to carve a subset of the
GF HoTT abstract file to accommodate an Agda program, but one encounters rocks as soon
as one begins to dig. For example, in \autoref{fig:M1} is some rendered latex
taken verbatim from Ranta's test code.

With some of hours of tinkering on the pidgin logic concrete syntax and some
reverse engineering with help from the GF shell, one is able to get these
definitions in \autoref{fig:M2}, which are intended to share the same syntactic
space as cubicalTT. We note the first definition of ``contractability" actually
runs in cubicalTT up to renaming a lexical items, and it is clear that the
translation from that to Agda should be a benign task. However, the
\emph{equivalence} syntax is stuck with the artifact from the bloated abstract
syntax for the of the anaphoric use of ``it", which may presumably be fixed with
a few hours more of tinkering, but becomes even more complicated when not just
defining new types, but actually writing real mathematical proofs, or relatively
large terms. To extend this grammar to accommodate a chapter worth of material,
let alone a book, will not just require extending the lexicon, but encountering
other syntactic phenomena that will further be difficult to compress when
defining Agda's concrete syntax.

\begin{figure}

 \textbf{Definition}:
 A type $A$ is contractible, if there is $a : A$, called the center of contraction, such that for all $x : A$, $\equalH {a}{x}$.

 \textbf{Definition}:
 A map $f : \arrowH {A}{B}$ is an equivalence, if for all $y : B$, its fiber, $\comprehensionH {x}{A}{\equalH {\appH {f}{x}}{y}}$, is contractible.
 We write $\equivalenceH {A}{B}$, if there is an equivalence $\arrowH {A}{B}$.
\caption{Rendered Latex} \label{fig:M1}


\begin{verbatim}
isContr ( A : Set ) : Set = ( a : A ) ( * ) ( ( x : A ) -> Id ( a ) ( x ) )

Equivalence ( f : A -> B ) : Set = 
  ( y : B ) -> ( isContr ( fiber it ) ) ; ; ; 
  fiber it : Set = ( x : A ) ( * ) ( Id ( f ( x ) ) ( y ) )
\end{verbatim}
\caption{Pidgin cubicalTT} \label{fig:M2}


\begin{code}[hide]%
\>[0]\AgdaSymbol{\{-\#}\AgdaSpace{}%
\AgdaKeyword{OPTIONS}\AgdaSpace{}%
\AgdaPragma{--omega-in-omega}\AgdaSpace{}%
\AgdaPragma{--type-in-type}\AgdaSpace{}%
\AgdaSymbol{\#-\}}\<%
\\
%
\\[\AgdaEmptyExtraSkip]%
\>[0]\AgdaKeyword{module}\AgdaSpace{}%
\AgdaModule{ContrClean}\AgdaSpace{}%
\AgdaKeyword{where}\<%
\\
%
\\[\AgdaEmptyExtraSkip]%
\>[0]\AgdaKeyword{open}\AgdaSpace{}%
\AgdaKeyword{import}\AgdaSpace{}%
\AgdaModule{Agda.Builtin.Sigma}\AgdaSpace{}%
\AgdaKeyword{public}\<%
\\
%
\\[\AgdaEmptyExtraSkip]%
\>[0]\AgdaKeyword{variable}\<%
\\
\>[0][@{}l@{\AgdaIndent{0}}]%
\>[2]\AgdaGeneralizable{A}\AgdaSpace{}%
\AgdaGeneralizable{B}\AgdaSpace{}%
\AgdaSymbol{:}\AgdaSpace{}%
\AgdaPrimitive{Set}\<%
\\
%
\\[\AgdaEmptyExtraSkip]%
\>[0]\AgdaKeyword{data}\AgdaSpace{}%
\AgdaOperator{\AgdaDatatype{\AgdaUnderscore{}≡\AgdaUnderscore{}}}\AgdaSpace{}%
\AgdaSymbol{\{}\AgdaBound{A}\AgdaSpace{}%
\AgdaSymbol{:}\AgdaSpace{}%
\AgdaPrimitive{Set}\AgdaSymbol{\}}\AgdaSpace{}%
\AgdaSymbol{(}\AgdaBound{a}\AgdaSpace{}%
\AgdaSymbol{:}\AgdaSpace{}%
\AgdaBound{A}\AgdaSymbol{)}\AgdaSpace{}%
\AgdaSymbol{:}\AgdaSpace{}%
\AgdaBound{A}\AgdaSpace{}%
\AgdaSymbol{→}\AgdaSpace{}%
\AgdaPrimitive{Set}\AgdaSpace{}%
\AgdaKeyword{where}\<%
\\
\>[0][@{}l@{\AgdaIndent{0}}]%
\>[2]\AgdaInductiveConstructor{r}\AgdaSpace{}%
\AgdaSymbol{:}\AgdaSpace{}%
\AgdaBound{a}\AgdaSpace{}%
\AgdaOperator{\AgdaDatatype{≡}}\AgdaSpace{}%
\AgdaBound{a}\<%
\\
%
\\[\AgdaEmptyExtraSkip]%
\>[0]\AgdaKeyword{infix}\AgdaSpace{}%
\AgdaNumber{20}\AgdaSpace{}%
\AgdaOperator{\AgdaDatatype{\AgdaUnderscore{}≡\AgdaUnderscore{}}}\<%
\\
%
\\[\AgdaEmptyExtraSkip]%
\>[0]\AgdaFunction{id}\AgdaSpace{}%
\AgdaSymbol{:}\AgdaSpace{}%
\AgdaGeneralizable{A}\AgdaSpace{}%
\AgdaSymbol{→}\AgdaSpace{}%
\AgdaGeneralizable{A}\<%
\\
\>[0]\AgdaFunction{id}\AgdaSpace{}%
\AgdaSymbol{=}\AgdaSpace{}%
\AgdaSymbol{λ}\AgdaSpace{}%
\AgdaBound{z}\AgdaSpace{}%
\AgdaSymbol{→}\AgdaSpace{}%
\AgdaBound{z}\<%
\\
%
\\[\AgdaEmptyExtraSkip]%
\>[0]\<%
\end{code}

\begin{code}%
\>[0]\<%
\\
\>[0]\AgdaFunction{isContr}\AgdaSpace{}%
\AgdaSymbol{:}\AgdaSpace{}%
\AgdaSymbol{(}\AgdaBound{A}\AgdaSpace{}%
\AgdaSymbol{:}\AgdaSpace{}%
\AgdaPrimitive{Set}\AgdaSymbol{)}\AgdaSpace{}%
\AgdaSymbol{→}\AgdaSpace{}%
\AgdaPrimitive{Set}\<%
\\
\>[0]\AgdaFunction{isContr}\AgdaSpace{}%
\AgdaBound{A}\AgdaSpace{}%
\AgdaSymbol{=}%
\>[13]\AgdaRecord{Σ}\AgdaSpace{}%
\AgdaBound{A}\AgdaSpace{}%
\AgdaSymbol{λ}\AgdaSpace{}%
\AgdaBound{a}\AgdaSpace{}%
\AgdaSymbol{→}\AgdaSpace{}%
\AgdaSymbol{(}\AgdaBound{x}\AgdaSpace{}%
\AgdaSymbol{:}\AgdaSpace{}%
\AgdaBound{A}\AgdaSymbol{)}\AgdaSpace{}%
\AgdaSymbol{→}\AgdaSpace{}%
\AgdaSymbol{(}\AgdaBound{a}\AgdaSpace{}%
\AgdaOperator{\AgdaDatatype{≡}}\AgdaSpace{}%
\AgdaBound{x}\AgdaSymbol{)}\<%
\\
%
\\[\AgdaEmptyExtraSkip]%
\>[0]\AgdaFunction{Equivalence}\AgdaSpace{}%
\AgdaSymbol{:}\AgdaSpace{}%
\AgdaSymbol{(}\AgdaBound{A}\AgdaSpace{}%
\AgdaBound{B}\AgdaSpace{}%
\AgdaSymbol{:}\AgdaSpace{}%
\AgdaPrimitive{Set}\AgdaSymbol{)}\AgdaSpace{}%
\AgdaSymbol{→}\AgdaSpace{}%
\AgdaSymbol{(}\AgdaBound{f}\AgdaSpace{}%
\AgdaSymbol{:}\AgdaSpace{}%
\AgdaBound{A}\AgdaSpace{}%
\AgdaSymbol{→}\AgdaSpace{}%
\AgdaBound{B}\AgdaSymbol{)}\AgdaSpace{}%
\AgdaSymbol{→}\AgdaSpace{}%
\AgdaPrimitive{Set}\<%
\\
\>[0]\AgdaFunction{Equivalence}\AgdaSpace{}%
\AgdaBound{A}\AgdaSpace{}%
\AgdaBound{B}\AgdaSpace{}%
\AgdaBound{f}\AgdaSpace{}%
\AgdaSymbol{=}\AgdaSpace{}%
\AgdaSymbol{∀}\AgdaSpace{}%
\AgdaSymbol{(}\AgdaBound{y}\AgdaSpace{}%
\AgdaSymbol{:}\AgdaSpace{}%
\AgdaBound{B}\AgdaSymbol{)}\AgdaSpace{}%
\AgdaSymbol{→}\AgdaSpace{}%
\AgdaFunction{isContr}\AgdaSpace{}%
\AgdaSymbol{(}\AgdaFunction{fiber'}\AgdaSpace{}%
\AgdaBound{y}\AgdaSymbol{)}\<%
\\
\>[0][@{}l@{\AgdaIndent{0}}]%
\>[2]\AgdaKeyword{where}\<%
\\
\>[2][@{}l@{\AgdaIndent{0}}]%
\>[4]\AgdaFunction{fiber'}\AgdaSpace{}%
\AgdaSymbol{:}\AgdaSpace{}%
\AgdaSymbol{(}\AgdaBound{y}\AgdaSpace{}%
\AgdaSymbol{:}\AgdaSpace{}%
\AgdaBound{B}\AgdaSymbol{)}\AgdaSpace{}%
\AgdaSymbol{→}\AgdaSpace{}%
\AgdaPrimitive{Set}\<%
\\
%
\>[4]\AgdaFunction{fiber'}\AgdaSpace{}%
\AgdaBound{y}\AgdaSpace{}%
\AgdaSymbol{=}\AgdaSpace{}%
\AgdaRecord{Σ}\AgdaSpace{}%
\AgdaBound{A}\AgdaSpace{}%
\AgdaSymbol{(λ}\AgdaSpace{}%
\AgdaBound{x}\AgdaSpace{}%
\AgdaSymbol{→}\AgdaSpace{}%
\AgdaBound{y}\AgdaSpace{}%
\AgdaOperator{\AgdaDatatype{≡}}\AgdaSpace{}%
\AgdaBound{f}\AgdaSpace{}%
\AgdaBound{x}\AgdaSymbol{)}\<%
\\
\>[0]\<%
\end{code}

% \begin{code}[hide]%
\>[0]\<%
\\
\>[0]\AgdaComment{-- \{-\# OPTIONS --omega-in-omega --type-in-type \#-\}}\<%
\\
%
\\[\AgdaEmptyExtraSkip]%
\>[0]\AgdaKeyword{module}\AgdaSpace{}%
\AgdaModule{ex}\AgdaSpace{}%
\AgdaKeyword{where}\<%
\\
%
\\[\AgdaEmptyExtraSkip]%
\>[0]\AgdaKeyword{data}\AgdaSpace{}%
\AgdaDatatype{aℕ}\AgdaSpace{}%
\AgdaSymbol{:}\AgdaSpace{}%
\AgdaPrimitive{Set}\AgdaSpace{}%
\AgdaKeyword{where}\<%
\\
\>[0][@{}l@{\AgdaIndent{0}}]%
\>[2]\AgdaInductiveConstructor{zero'}\AgdaSpace{}%
\AgdaSymbol{:}\AgdaSpace{}%
\AgdaDatatype{aℕ}\<%
\\
%
\\[\AgdaEmptyExtraSkip]%
\>[0]\AgdaKeyword{variable}\<%
\\
\>[0][@{}l@{\AgdaIndent{0}}]%
\>[2]\AgdaGeneralizable{A}\AgdaSpace{}%
\AgdaSymbol{:}\AgdaSpace{}%
\AgdaPrimitive{Set}\<%
\\
%
\>[2]\AgdaGeneralizable{D}\AgdaSpace{}%
\AgdaSymbol{:}\AgdaSpace{}%
\AgdaPrimitive{Set}\<%
\\
%
\>[2]\AgdaGeneralizable{stuff}\AgdaSpace{}%
\AgdaSymbol{:}\AgdaSpace{}%
\AgdaPrimitive{Set}\<%
\\
%
\\[\AgdaEmptyExtraSkip]%
\>[0]\AgdaFunction{definition-body}\AgdaSpace{}%
\AgdaSymbol{=}\AgdaSpace{}%
\AgdaDatatype{aℕ}\<%
\\
%
\\[\AgdaEmptyExtraSkip]%
\>[0]\AgdaFunction{T}\AgdaSpace{}%
\AgdaSymbol{=}\AgdaSpace{}%
\AgdaDatatype{aℕ}\AgdaSpace{}%
\AgdaSymbol{→}\AgdaSpace{}%
\AgdaDatatype{aℕ}\<%
\\
\>[0]\AgdaFunction{L}\AgdaSpace{}%
\AgdaSymbol{=}\AgdaSpace{}%
\AgdaDatatype{aℕ}\<%
\\
\>[0]\AgdaFunction{E}\AgdaSpace{}%
\AgdaSymbol{=}\AgdaSpace{}%
\AgdaDatatype{aℕ}\<%
\\
\>[0]\AgdaFunction{C}\AgdaSpace{}%
\AgdaSymbol{=}\AgdaSpace{}%
\AgdaDatatype{aℕ}\<%
\\
%
\\[\AgdaEmptyExtraSkip]%
\>[0]\AgdaFunction{proof}\AgdaSpace{}%
\AgdaSymbol{:}\AgdaSpace{}%
\AgdaFunction{L}\<%
\\
\>[0]\AgdaFunction{proof}\AgdaSpace{}%
\AgdaSymbol{=}\AgdaSpace{}%
\AgdaInductiveConstructor{zero'}\<%
\\
%
\\[\AgdaEmptyExtraSkip]%
\>[0]\AgdaFunction{corollaryStuff}\AgdaSpace{}%
\AgdaSymbol{=}\AgdaSpace{}%
\AgdaDatatype{aℕ}\<%
\\
%
\\[\AgdaEmptyExtraSkip]%
\>[0]\AgdaFunction{proofNeedingLemma}\AgdaSpace{}%
\AgdaSymbol{:}\AgdaSpace{}%
\AgdaDatatype{aℕ}\AgdaSpace{}%
\AgdaSymbol{→}\AgdaSpace{}%
\AgdaDatatype{aℕ}\AgdaSpace{}%
\AgdaSymbol{→}\AgdaSpace{}%
\AgdaDatatype{aℕ}\<%
\\
\>[0]\AgdaFunction{proofNeedingLemma}\AgdaSpace{}%
\AgdaBound{x}\AgdaSpace{}%
\AgdaSymbol{=}\AgdaSpace{}%
\AgdaSymbol{λ}\AgdaSpace{}%
\AgdaBound{x₁}\AgdaSpace{}%
\AgdaSymbol{→}\AgdaSpace{}%
\AgdaInductiveConstructor{zero'}\<%
\\
\>[0]\<%
\end{code}

\subsubsection{Agda Programming}

Listed is the syntax Agda uses for judgements: \term{T} : \term{Set} means
\term{T} is a type, \term{t} : \term{T} means a term \term{t} has type \term{T},
and \term{t} = \term{t'} means \term{t} is defined to be judgmentally equal to
\term{t'}. Once one has made this equality judgement, Agda can normalize the
definitionally equal terms to the same normal form. Let's compare these Agda
judgements to those keywords ubiquitous in mathematics:

\begin{figure}
\centering
\begin{minipage}[t]{.3\textwidth}
\vspace{2cm}
\begin{itemize}
\item Axiom
\item Definition
\item Lemma
\item Theorem
\item Proof
\item Corollary
\item Example
\end{itemize}
\end{minipage}%
\begin{minipage}[t]{.55\textwidth}
\begin{code}%
\>[0]\AgdaKeyword{postulate}%
\>[12]\AgdaComment{-- Axiom}\<%
\\
\>[0][@{}l@{\AgdaIndent{0}}]%
\>[2]\AgdaPostulate{axiom}\AgdaSpace{}%
\AgdaSymbol{:}\AgdaSpace{}%
\AgdaGeneralizable{A}\<%
\\
%
\\[\AgdaEmptyExtraSkip]%
\>[0]\AgdaFunction{definition}\AgdaSpace{}%
\AgdaSymbol{:}\AgdaSpace{}%
\AgdaGeneralizable{stuff}\AgdaSpace{}%
\AgdaSymbol{→}\AgdaSpace{}%
\AgdaPrimitive{Set}\AgdaSpace{}%
\AgdaComment{--Definition}\<%
\\
\>[0]\AgdaFunction{definition}\AgdaSpace{}%
\AgdaBound{s}\AgdaSpace{}%
\AgdaSymbol{=}\AgdaSpace{}%
\AgdaFunction{definition-body}\<%
\\
%
\\[\AgdaEmptyExtraSkip]%
\>[0]\AgdaFunction{theorem}\AgdaSpace{}%
\AgdaSymbol{:}\AgdaSpace{}%
\AgdaFunction{T}%
\>[16]\AgdaComment{-- Theorem Statement}\<%
\\
\>[0]\AgdaFunction{theorem}\AgdaSpace{}%
\AgdaSymbol{=}\AgdaSpace{}%
\AgdaFunction{proofNeedingLemma}\AgdaSpace{}%
\AgdaFunction{lemma}\AgdaSpace{}%
\AgdaComment{-- Proof}\<%
\\
\>[0][@{}l@{\AgdaIndent{0}}]%
\>[2]\AgdaKeyword{where}\<%
\\
\>[2][@{}l@{\AgdaIndent{0}}]%
\>[4]\AgdaFunction{lemma}\AgdaSpace{}%
\AgdaSymbol{:}\AgdaSpace{}%
\AgdaFunction{L}%
\>[18]\AgdaComment{-- Lemma Statement}\<%
\\
%
\>[4]\AgdaFunction{lemma}\AgdaSpace{}%
\AgdaSymbol{=}\AgdaSpace{}%
\AgdaFunction{proof}\<%
\\
%
\\[\AgdaEmptyExtraSkip]%
\>[0]\AgdaFunction{corollary}\AgdaSpace{}%
\AgdaSymbol{:}\AgdaSpace{}%
\AgdaFunction{corollaryStuff}\AgdaSpace{}%
\AgdaSymbol{→}\AgdaSpace{}%
\AgdaFunction{C}\<%
\\
\>[0]\AgdaFunction{corollary}\AgdaSpace{}%
\AgdaBound{coro-term}\AgdaSpace{}%
\AgdaSymbol{=}\AgdaSpace{}%
\AgdaFunction{theorem}\AgdaSpace{}%
\AgdaBound{coro-term}\<%
\\
%
\\[\AgdaEmptyExtraSkip]%
\>[0]\AgdaFunction{example}\AgdaSpace{}%
\AgdaSymbol{:}\AgdaSpace{}%
\AgdaFunction{E}%
\>[16]\AgdaComment{-- Example Statement}\<%
\\
\>[0]\AgdaFunction{example}\AgdaSpace{}%
\AgdaSymbol{=}\AgdaSpace{}%
\AgdaFunction{proof}\<%
\end{code}
\end{minipage}
\caption{Mathematical Assertions and Agda Judgements} \label{fig:O1}
\end{figure}

Formation rules are given by the first line of the data declaration, followed
by some number of constructors which correspond to the introduction forms of the
type being defined. Therefore, to define a type for Booleans, $𝔹$, we present
these rules both in the proof theoretic and Agda syntax. We note that the
context $\Gamma$ is not present in Agda.

\begin{minipage}[t]{.4\textwidth}
\vspace{3mm}
\[
  \begin{prooftree}
    \infer1[]{ \vdash 𝔹 : {\rm type}}
  \end{prooftree}
\]
\[
  \begin{prooftree}
    \infer1[]{ \Gamma \vdash true : 𝔹  }
  \end{prooftree}
  \quad \quad
  \begin{prooftree}
    \infer1[]{ \Gamma \vdash false : 𝔹  }
  \end{prooftree}
\]
\end{minipage}
\begin{minipage}[t]{.3\textwidth}
\begin{code}%
\>[0]\AgdaKeyword{data}\AgdaSpace{}%
\AgdaDatatype{𝔹}\AgdaSpace{}%
\AgdaSymbol{:}\AgdaSpace{}%
\AgdaPrimitive{Set}\AgdaSpace{}%
\AgdaKeyword{where}\AgdaSpace{}%
\AgdaComment{-- formation rule}\<%
\\
\>[0][@{}l@{\AgdaIndent{0}}]%
\>[2]\AgdaInductiveConstructor{true}%
\>[8]\AgdaSymbol{:}\AgdaSpace{}%
\AgdaDatatype{𝔹}\AgdaSpace{}%
\AgdaComment{-- introduction rule}\<%
\\
%
\>[2]\AgdaInductiveConstructor{false}\AgdaSpace{}%
\AgdaSymbol{:}\AgdaSpace{}%
\AgdaDatatype{𝔹}\<%
\end{code}
\end{minipage}

The elimination forms are deriveable from the introduction rules, and the
computation rules can then be extracted by via the harmonious relationship
between the introduction and elmination forms \cite{pfenningHar}. Agda's pattern
matching is equivalent to the deriveable dependently typed elimination forms
\cite{coqPat}, and one can simply pattern match on a boolean, producing multiple
lines for each constructor of the variable's type, to extract the classic
recursion principle for Booleans. The \term{if then else} statement shown below
is really just the boolean elimination form. It is not standard to include the
premises of the eqaulity rules.

\begin{minipage}[t]{.4\textwidth}
\[
  \begin{prooftree}
    \hypo{̌\Gamma \vdash A : {\rm type} }
    \hypo{\Gamma \vdash b : 𝔹 }
    \hypo{\Gamma \vdash a1 : A}
    \hypo{\Gamma \vdash a2 : A }
    \infer4[]{\Gamma \vdash boolrec\{a1;a2\}(b) : A }
  \end{prooftree}
\]
$$\Gamma \vdash boolrec\{a1;a2\}(true) \equiv a1 : A$$
$$\Gamma \vdash boolrec\{a1;a2\}(false) \equiv a2 : A$$
\end{minipage}
\hfill
\begin{minipage}[t]{.5\textwidth}
\begin{code}%
\>[0]\AgdaOperator{\AgdaFunction{if\AgdaUnderscore{}then\AgdaUnderscore{}else\AgdaUnderscore{}}}\AgdaSpace{}%
\AgdaSymbol{:}\<%
\\
\>[0][@{}l@{\AgdaIndent{0}}]%
\>[2]\AgdaSymbol{\{}\AgdaBound{A}\AgdaSpace{}%
\AgdaSymbol{:}\AgdaSpace{}%
\AgdaPrimitive{Set}\AgdaSymbol{\}}\AgdaSpace{}%
\AgdaSymbol{→}\AgdaSpace{}%
\AgdaDatatype{𝔹}\AgdaSpace{}%
\AgdaSymbol{→}\AgdaSpace{}%
\AgdaBound{A}\AgdaSpace{}%
\AgdaSymbol{→}\AgdaSpace{}%
\AgdaBound{A}\AgdaSpace{}%
\AgdaSymbol{→}\AgdaSpace{}%
\AgdaBound{A}\<%
\\
\>[0]\AgdaOperator{\AgdaFunction{if}}\AgdaSpace{}%
\AgdaInductiveConstructor{true}\AgdaSpace{}%
\AgdaOperator{\AgdaFunction{then}}\AgdaSpace{}%
\AgdaBound{a1}\AgdaSpace{}%
\AgdaOperator{\AgdaFunction{else}}\AgdaSpace{}%
\AgdaBound{a2}\AgdaSpace{}%
\AgdaSymbol{=}\AgdaSpace{}%
\AgdaBound{a1}\<%
\\
\>[0]\AgdaOperator{\AgdaFunction{if}}\AgdaSpace{}%
\AgdaInductiveConstructor{false}\AgdaSpace{}%
\AgdaOperator{\AgdaFunction{then}}\AgdaSpace{}%
\AgdaBound{a1}\AgdaSpace{}%
\AgdaOperator{\AgdaFunction{else}}\AgdaSpace{}%
\AgdaBound{a2}\AgdaSpace{}%
\AgdaSymbol{=}\AgdaSpace{}%
\AgdaBound{a2}\<%
\end{code}
\end{minipage}

When using Agda one is interactively building a proof via holes. There is an
Agda Emacs mode which enables this. Glossing over many details, we show sample
code in the proof development state prior to pattern matching on \codeword{b}.
We have a hole, \codeword{{ b }0}, and the proof state is displayed to the
right. It shows both the current context with \codeword{A, b, a1, a2}, the goal
which is something of type \codeword{A}, and what we have, \codeword{B},
represents the type of the variable in the hole.

\hfill
\begin{minipage}[t]{.4\textwidth}
\begin{verbatim}
if_then_else_ :
  {A : Set} → B → A → A → A
if b then a1 else a2 = { b }0
\end{verbatim}
\end{minipage}
\hfill
\begin{minipage}[t]{.5\textwidth}
\begin{verbatim}
Goal: A
Have: B
———————————————
a2 : A
a1 : A
b  : B
A  : Set   (not in scope)
\end{verbatim}
\end{minipage}

The interactivity is performed via emacs commands, and every time one updates
the hole with a new term, we can immediately view the next goal with an updated
context. The underscore in \term{if_then_else_} denotes the placement of the
arguements, as Agda allows mixfix operations. Agda allows for more nuanced
syntacic features like unicode. This is interesting from the \emph{concrete
syntax} perspective as the arguement placement and symbolic expressiveness makes
Agda's syntax feel more familiar to the mathematician. We also observe the use
of parametric polymorphism, namely, that we can extract a member of some
arbtitrary type \term{A} from a boolean value given two members of \term{A}.

This polymorphism allows one to implement simple programs like boolean negation,
\term{~}, and more interestingly, \term{functionalNegation}, where one can use
functions as arguements. \term{functionalNegation} is a functional, or higher
order functions, which take functions as arguements and return functions. We
also notice in \term{functionalNegation} that one can work directly with a
built-in $\lambda$ to ensure the correct return type.

\begin{code}%
\>[0]\AgdaFunction{\textasciitilde{}}\AgdaSpace{}%
\AgdaSymbol{:}\AgdaSpace{}%
\AgdaDatatype{𝔹}\AgdaSpace{}%
\AgdaSymbol{→}\AgdaSpace{}%
\AgdaDatatype{𝔹}\<%
\\
\>[0]\AgdaFunction{\textasciitilde{}}\AgdaSpace{}%
\AgdaBound{b}\AgdaSpace{}%
\AgdaSymbol{=}\AgdaSpace{}%
\AgdaOperator{\AgdaFunction{if}}\AgdaSpace{}%
\AgdaBound{b}\AgdaSpace{}%
\AgdaOperator{\AgdaFunction{then}}\AgdaSpace{}%
\AgdaInductiveConstructor{false}\AgdaSpace{}%
\AgdaOperator{\AgdaFunction{else}}\AgdaSpace{}%
\AgdaInductiveConstructor{true}\<%
\\
%
\\[\AgdaEmptyExtraSkip]%
\>[0]\AgdaFunction{functionalNegation}\AgdaSpace{}%
\AgdaSymbol{:}\AgdaSpace{}%
\AgdaDatatype{𝔹}\AgdaSpace{}%
\AgdaSymbol{→}\AgdaSpace{}%
\AgdaSymbol{(}\AgdaDatatype{𝔹}\AgdaSpace{}%
\AgdaSymbol{→}\AgdaSpace{}%
\AgdaDatatype{𝔹}\AgdaSymbol{)}\AgdaSpace{}%
\AgdaSymbol{→}\AgdaSpace{}%
\AgdaSymbol{(}\AgdaDatatype{𝔹}\AgdaSpace{}%
\AgdaSymbol{→}\AgdaSpace{}%
\AgdaDatatype{𝔹}\AgdaSymbol{)}\<%
\\
\>[0]\AgdaFunction{functionalNegation}\AgdaSpace{}%
\AgdaBound{b}\AgdaSpace{}%
\AgdaBound{f}\AgdaSpace{}%
\AgdaSymbol{=}\AgdaSpace{}%
\AgdaOperator{\AgdaFunction{if}}\AgdaSpace{}%
\AgdaBound{b}\AgdaSpace{}%
\AgdaOperator{\AgdaFunction{then}}\AgdaSpace{}%
\AgdaBound{f}\AgdaSpace{}%
\AgdaOperator{\AgdaFunction{else}}\AgdaSpace{}%
\AgdaSymbol{λ}\AgdaSpace{}%
\AgdaBound{b'}\AgdaSpace{}%
\AgdaSymbol{→}\AgdaSpace{}%
\AgdaBound{f}\AgdaSpace{}%
\AgdaSymbol{(}\AgdaFunction{\textasciitilde{}}\AgdaSpace{}%
\AgdaBound{b'}\AgdaSymbol{)}\<%
\end{code}

This simple example leads us to one of the domains our subsequent grammars will
describe, like arithmetic (see \ref{npf}). We show how to inductively define
natural numbers in Agda, with the formation and introduction rules included
beside for contrast.

\begin{minipage}[t]{.4\textwidth}
\vspace{3mm}
\[
  \begin{prooftree}
    \infer1[]{ \vdash ℕ : {\rm type}}
  \end{prooftree}
\]
\[
  \begin{prooftree}
    \infer1[]{ \Gamma \vdash 0 : ℕ  }
  \end{prooftree}
  \quad \quad
  \begin{prooftree}
    \hypo{\Gamma \vdash n : ℕ}
    \infer1[]{ \Gamma \vdash (suc\ n) : ℕ  }
  \end{prooftree}
\]
\end{minipage}
\begin{minipage}[t]{.3\textwidth}
\begin{code}%
\>[0]\AgdaKeyword{data}\AgdaSpace{}%
\AgdaDatatype{ℕ}\AgdaSpace{}%
\AgdaSymbol{:}\AgdaSpace{}%
\AgdaPrimitive{Set}\AgdaSpace{}%
\AgdaKeyword{where}\<%
\\
\>[0][@{}l@{\AgdaIndent{0}}]%
\>[2]\AgdaInductiveConstructor{zero}\AgdaSpace{}%
\AgdaSymbol{:}\AgdaSpace{}%
\AgdaDatatype{ℕ}\<%
\\
%
\>[2]\AgdaInductiveConstructor{suc}%
\>[7]\AgdaSymbol{:}\AgdaSpace{}%
\AgdaDatatype{ℕ}\AgdaSpace{}%
\AgdaSymbol{→}\AgdaSpace{}%
\AgdaDatatype{ℕ}\<%
\end{code}
\end{minipage}

This is a recursive type, whereby pattern matching over $ℕ$ allows one to use an
induction hypothesis over the subtree and gurantee termination when making
recurive calls on the function being defined. We can define a recursion
principle for $ℕ$, which gives one the power to build iterators.
Again, we include the elimination and equality rules for syntactic
juxtaposition.

\[
  \begin{prooftree}
    \hypo{̌\Gamma \vdash X : {\rm type} }
    \hypo{\Gamma \vdash n : ℕ }
    \hypo{\Gamma \vdash e₀ : X}
    \hypo{\Gamma, x : ℕ, y : X \vdash e₁ : X }
    \infer4[]{\Gamma \vdash natrec\{e\;x.y.e₁\}(n) : X }
  \end{prooftree}
\]
$$\Gamma \vdash natrec\{e₀;x.y.e₁\}(n) \equiv e₀ : X$$
$$\Gamma \vdash natrec\{e₀;x.y.e₁\}(suc\ n) \equiv e₁[x := n,y := natrec\{e₀;x.y.e₁\}(n)] : X$$
\begin{code}%
\>[0]\AgdaFunction{natrec}\AgdaSpace{}%
\AgdaSymbol{:}\AgdaSpace{}%
\AgdaSymbol{\{}\AgdaBound{X}\AgdaSpace{}%
\AgdaSymbol{:}\AgdaSpace{}%
\AgdaPrimitive{Set}\AgdaSymbol{\}}\AgdaSpace{}%
\AgdaSymbol{→}\AgdaSpace{}%
\AgdaDatatype{ℕ}\AgdaSpace{}%
\AgdaSymbol{→}\AgdaSpace{}%
\AgdaBound{X}\AgdaSpace{}%
\AgdaSymbol{→}\AgdaSpace{}%
\AgdaSymbol{(}\AgdaDatatype{ℕ}\AgdaSpace{}%
\AgdaSymbol{→}\AgdaSpace{}%
\AgdaBound{X}\AgdaSpace{}%
\AgdaSymbol{→}\AgdaSpace{}%
\AgdaBound{X}\AgdaSymbol{)}\AgdaSpace{}%
\AgdaSymbol{→}\AgdaSpace{}%
\AgdaBound{X}\<%
\\
\>[0]\AgdaFunction{natrec}\AgdaSpace{}%
\AgdaInductiveConstructor{zero}\AgdaSpace{}%
\AgdaBound{e₀}\AgdaSpace{}%
\AgdaBound{e₁}\AgdaSpace{}%
\AgdaSymbol{=}\AgdaSpace{}%
\AgdaBound{e₀}\<%
\\
\>[0]\AgdaFunction{natrec}\AgdaSpace{}%
\AgdaSymbol{(}\AgdaInductiveConstructor{suc}\AgdaSpace{}%
\AgdaBound{n}\AgdaSymbol{)}\AgdaSpace{}%
\AgdaBound{e₀}\AgdaSpace{}%
\AgdaBound{e₁}\AgdaSpace{}%
\AgdaSymbol{=}\AgdaSpace{}%
\AgdaBound{e₁}\AgdaSpace{}%
\AgdaBound{n}\AgdaSpace{}%
\AgdaSymbol{(}\AgdaFunction{natrec}\AgdaSpace{}%
\AgdaBound{n}\AgdaSpace{}%
\AgdaBound{e₀}\AgdaSpace{}%
\AgdaBound{e₁}\AgdaSymbol{)}\<%
\end{code}

Since we are in a dependently typed setting, however, we prove theorems as well
as write programs. Therefore, we can see this recursion principle as a special
case of the induction principle \term{natind}, which represents the by induction
for the natural numbers. One may notice that while the types are different, the
programs \term{natrec} and \term{natind} are actually the same, up to
α-equivalence. One can therefore, as a corollary, actually just include the type
infomation and Agda can infer the speciliazation for you, as seen in
\term{natrec'} below.

\[
  \begin{prooftree}
    \hypo{̌\Gamma, x : ℕ \vdash X : {\rm type} }
    \hypo{\Gamma \vdash n : ℕ }
    \hypo{\Gamma \vdash e₀ : X[x := 0] }
    \hypo{\Gamma, y : ℕ, z : X[x := y] \vdash e₁ : X[x := suc\ y]}
    \infer4[]{\Gamma \vdash natind\{e₀,\;x.y.e₁\}(n) : X[x := n]}
  \end{prooftree}
\]
$$\Gamma \vdash natind\{e₀;x.y.e₁\}(n) \equiv e₀ : X[x := 0]$$
$$\Gamma \vdash natind\{e₀;x.y.e₁\}(suc\ n) \equiv e₁[x := n,y := natind\{e₀;x.y.e₁\}(n)] : X[x := suc\ n]$$
\begin{code}%
\>[0]\AgdaFunction{natind}\AgdaSpace{}%
\AgdaSymbol{:}\AgdaSpace{}%
\AgdaSymbol{\{}\AgdaBound{X}\AgdaSpace{}%
\AgdaSymbol{:}\AgdaSpace{}%
\AgdaDatatype{ℕ}\AgdaSpace{}%
\AgdaSymbol{→}\AgdaSpace{}%
\AgdaPrimitive{Set}\AgdaSymbol{\}}\AgdaSpace{}%
\AgdaSymbol{→}\AgdaSpace{}%
\AgdaSymbol{(}\AgdaBound{n}\AgdaSpace{}%
\AgdaSymbol{:}\AgdaSpace{}%
\AgdaDatatype{ℕ}\AgdaSymbol{)}\AgdaSpace{}%
\AgdaSymbol{→}\AgdaSpace{}%
\AgdaBound{X}\AgdaSpace{}%
\AgdaInductiveConstructor{zero}\AgdaSpace{}%
\AgdaSymbol{→}\AgdaSpace{}%
\AgdaSymbol{((}\AgdaBound{n}\AgdaSpace{}%
\AgdaSymbol{:}\AgdaSpace{}%
\AgdaDatatype{ℕ}\AgdaSymbol{)}\AgdaSpace{}%
\AgdaSymbol{→}\AgdaSpace{}%
\AgdaBound{X}\AgdaSpace{}%
\AgdaBound{n}\AgdaSpace{}%
\AgdaSymbol{→}\AgdaSpace{}%
\AgdaBound{X}\AgdaSpace{}%
\AgdaSymbol{(}\AgdaInductiveConstructor{suc}\AgdaSpace{}%
\AgdaBound{n}\AgdaSymbol{))}\AgdaSpace{}%
\AgdaSymbol{→}\AgdaSpace{}%
\AgdaBound{X}\AgdaSpace{}%
\AgdaBound{n}\<%
\\
\>[0]\AgdaFunction{natind}\AgdaSpace{}%
\AgdaInductiveConstructor{zero}\AgdaSpace{}%
\AgdaBound{base}\AgdaSpace{}%
\AgdaBound{step}\AgdaSpace{}%
\AgdaSymbol{=}\AgdaSpace{}%
\AgdaBound{base}\<%
\\
\>[0]\AgdaFunction{natind}\AgdaSpace{}%
\AgdaSymbol{(}\AgdaInductiveConstructor{suc}\AgdaSpace{}%
\AgdaBound{n}\AgdaSymbol{)}\AgdaSpace{}%
\AgdaBound{base}\AgdaSpace{}%
\AgdaBound{step}\AgdaSpace{}%
\AgdaSymbol{=}\AgdaSpace{}%
\AgdaBound{step}\AgdaSpace{}%
\AgdaBound{n}\AgdaSpace{}%
\AgdaSymbol{(}\AgdaFunction{natind}\AgdaSpace{}%
\AgdaBound{n}\AgdaSpace{}%
\AgdaBound{base}\AgdaSpace{}%
\AgdaBound{step}\AgdaSymbol{)}\<%
\\
%
\\[\AgdaEmptyExtraSkip]%
\>[0]\AgdaFunction{natrec'}\AgdaSpace{}%
\AgdaSymbol{:}\AgdaSpace{}%
\AgdaSymbol{\{}\AgdaBound{X}\AgdaSpace{}%
\AgdaSymbol{:}\AgdaSpace{}%
\AgdaPrimitive{Set}\AgdaSymbol{\}}\AgdaSpace{}%
\AgdaSymbol{→}\AgdaSpace{}%
\AgdaDatatype{ℕ}\AgdaSpace{}%
\AgdaSymbol{→}\AgdaSpace{}%
\AgdaBound{X}\AgdaSpace{}%
\AgdaSymbol{→}\AgdaSpace{}%
\AgdaSymbol{(}\AgdaDatatype{ℕ}\AgdaSpace{}%
\AgdaSymbol{→}\AgdaSpace{}%
\AgdaBound{X}\AgdaSpace{}%
\AgdaSymbol{→}\AgdaSpace{}%
\AgdaBound{X}\AgdaSymbol{)}\AgdaSpace{}%
\AgdaSymbol{→}\AgdaSpace{}%
\AgdaBound{X}\<%
\\
\>[0]\AgdaFunction{natrec'}\AgdaSpace{}%
\AgdaSymbol{=}\AgdaSpace{}%
\AgdaFunction{natind}\<%
\end{code}
We will defer the details of using induction and recursion principles for later
when we actually give examples of pidgin proofs some of our grammars can
handle.

% \begin{code}[hide]%
\>[0]\<%
\\
\>[0]\AgdaKeyword{module}\AgdaSpace{}%
\AgdaModule{primitives}\AgdaSpace{}%
\AgdaKeyword{where}\<%
\\
\>[0]\<%
\end{code}

Formation rules, are given by the data declaration, followed by some number of
constructors which correspond to the 


A proof the proof-theoretic this looks like the following


\begin{prooftree}
  \hypo{ \Gamma, A &\vdash B }
  \infer1[abs]{ \Gamma &\vdash A\to B }
  \hypo{ \Gamma \vdash A }
  \infer2[app]{ \Gamma \vdash B }
\end{prooftree}


\begin{code}%
\>[0]\<%
\\
\>[0]\AgdaKeyword{data}\AgdaSpace{}%
\AgdaDatatype{𝔹}\AgdaSpace{}%
\AgdaSymbol{:}\AgdaSpace{}%
\AgdaPrimitive{Set}\AgdaSpace{}%
\AgdaKeyword{where}\<%
\\
\>[0][@{}l@{\AgdaIndent{0}}]%
\>[2]\AgdaInductiveConstructor{true}\AgdaSpace{}%
\AgdaSymbol{:}\AgdaSpace{}%
\AgdaDatatype{𝔹}\<%
\\
%
\>[2]\AgdaInductiveConstructor{false}\AgdaSpace{}%
\AgdaSymbol{:}\AgdaSpace{}%
\AgdaDatatype{𝔹}\<%
\\
\>[0]\<%
\end{code}


-- $ \frac{\Gamma, x : A \vdash b : B} {\Gamma \vdash \lambda x. b : A \rightarrow
-- B} $

\begin{code}%
\>[0]\<%
\\
\>[0]\AgdaComment{-- if\AgdaUnderscore{}then\AgdaUnderscore{}else\AgdaUnderscore{} : \{A : Set\} → 𝔹 → A → A → A}\<%
\\
\>[0]\AgdaComment{-- if true then a1 else a2 = a1}\<%
\\
\>[0]\AgdaComment{-- if false then a1 else a2 = a2}\<%
\\
\>[0]\<%
\end{code}

-- data ℕ : Type where
--   zero : ℕ
--   suc  : ℕ → ℕ

-- data List (A : Type) : Type where
  

-- data Vector : 



-- \begin{code}%
\>[0]\<%
\\
\>[0]\AgdaComment{-- Type : Set₁}\<%
\\
\>[0]\AgdaComment{-- Type = Set}\<%
\\
%
\\[\AgdaEmptyExtraSkip]%
\>[0]\AgdaComment{-- \textbackslash{}end\{code\}}\<%


\caption{Agda} \label{fig:M3}
\end{figure}

Additionally, we give the Agda code in \autoref{fig:M3}, so-as to see what the
end result of such a program would be. The astute reader will also notice a
semantic in the pidgin rendering error relative to the Agda implementation.
\codeword{fiber} has the type \codeword{it : Set} instead of something like
\codeword{(y : B) : Set}, and the y variable is unbound in the \codeword{fiber}
expression. This demonstrates that to design a grammar prioritizing
\emph{semantic adequacy} and subsequently trying to incorporate \emph{syntactic
completeness} becomes a very difficult problem. Depending on the application of
the grammar, the emphasis on this axis is most assuredly a choice one should
consider up front.

While both these grammars have their strengths and weaknesses, one shall see
shortly that the approach in this thesis, taking an actual programming language
parser in Backus-Naur Form Converter (BNFC), GFifying it, and trying to use the
abstract syntax to model natural language, gives in some sense a dual challenge,
where the abstract syntax remains simple, but its linearizations become
must increase in complexity.

% Draw a figure with these as axes, the names of the grammars, and a z axis for
% size of lexicon, and basically how can we optimize all three

\subsection{Mohan Ganesalingam}


\begin{displayquote}

there is a considerable gap between what mathematicians claim is true and what
they believe, and this mismatch causes a number of serious linguistic problems

\end{displayquote}

Perhaps the most substantial analysis of the linguistic perspective on
written mathematics comes from Ganesalingam \cite{ganesalingam2013language}.
Not only does he pick up and reexamine much of Ranta's early work, but he
develops a whole theory for how to understand with the language mathematics from
a formal point of view, additionally working with many questions about the
foundation of mathematics. His model which is developed early in the treatise
and is referenced 
throughout uses Discourse Representation Theory \cite{kamp2011discourse}, to
capture anaphoric use of variables. While he is interested in analyzing
language, or goal is to translate, because the meaning of an expression is
contained in its set of formalizations, so GF is more of just a tool in the
pipeline rather than an actual infrastructure through which to dissect the various
nuances of human speech.


"
meaningful statements in some underlying logic. If it was pointed out that
a particular sentence had no translation into such a logic, a mathematician
would genuinely feel that they had been insufficiently precise. (The actual
translation into logic is never performed, because it is exceptionally laborious;
"

"
mathematics has a normative notion of what its content should look like; there is no
analogue in natural languages.
"

1.2.3 full adaptivity


"
From a linguistic perspective, the formal mode is more novel and interesting
because it is restricted enough to describe completely, both in terms of syntax
and semantics. By contrast, the informal mode seems as hard to describe as
general natural language. We will therefore look only at mathematics in the
formal mode.
"


Section 2

"
The primary function of symbolic mathematics is to abbreviate material
that would be too cumbersome to state with text alone. Thus a sentence
"

"
Because symbolic material functions primarily in an abbreviative capacity,
symbolic mathematics tends to occur inside textual mathematics rather than
vice versa. Thus mathematical texts are largely composed out of textual
"

"
adaptivity
a phenomenon that is much more remarkable than the use of symbols. Math-
ematical language expands as more mathematics is encountered. The kind of
"

"
Thus definitions always contain enough information to fully specify the
semantics of the material being defined.
"

"
As a result, textual mathematics predominantly
uses the third person singular and third person plural, to denote individual
"

"
verbs, typically to refer to the mutual intent of the author and reader.
Working mathematicians treat mathematical objects as if they were Pla-
tonic ideals, timeless objects existing independently of the physical world. The
"

"
The limited variation in person and tense means that inflectional morphol-
gy plays only a small part in mathematical language. The only morphological
"

"The syntax of textual mathematics also exhibits relatively limited variation."


this means that textual mathematics can be effectively captured by a context-
free grammar (in the sense of (Jurafsky and Martin, 2009, p. 387)).

In contrast to the morphology and syntax of textual mathematics, its
lexicon is remarkably varied. As we have noted above, the mechanism of

no tense or events
no intesnionality
no modality

"
usual. To a first approximation, mathematics does not exhibit any pragmatic
phenomena: the meaning of a mathematical sentence is merely its composition-
ally determined semantic content, and nothing more. In order to state this point
"


"
Due to the absence of pragmatic phenomena, phenomena which are some-
times analysed as semantic and sometimes analysed as pragmatic can be
treated as being purely semantic where they occur in mathematics, i.e. they
can be analysed in purely truth-conditional terms. This applies particularly
to presuppositions, which play an important role in mathematics. Because
"

"
Thus, in some intuitive sense, syntax is dependent on the types of expres-
sions in a way that does not occur in existing formal languages. As we will
show in §3.2 and Chapter 4, this notion of type is too semantic itself to be
formalised in syntactic terms. In other words, the type of an expression is too
closely related to what that expression refers to for purely syntactic notions
of type to be applicable.
"

"
most ambiguity in mathematics is not noticed by mathematicians, just as the extensive ambi-
guity in natural languages is “simply not noticed by the majority of language
"

"
in an extremely compact manner. In essence, they serve as a mathematical
alternative to anaphor. They cannot be eliminated precisely because anaphor
"

reanalysis

Chapter 7 pg 181

be equal as distinct. In both cases, a disparity between the way we think
about mathematical objects and the way they are formally defined causes
our linguistic theories to make incorrect predictions. In order to obtain the
correct predictions about language, we need to make sure that the formal
situation matches actual usage.

mal proofs are provided; and the cycle repeats. Thus informal mathematics
changes over the centuries. 187

% my idea
engineers are motivated by needs and desires (emirical, descriptive,
practial) , whereas mathematicians are much more idealistic in their pursuits,
almost comparable to a stances on religious scripture

mathematics developing over time is natural, the more and deeper we dig into the
ground, the more we develop refinements of what kind of tools we are using,
develop better iterations of the same tools (or possibly entirely new ones) as
well as knowledge about the ground in which we are digging (these are adjoining)

in some sense the library of babel problem, whereby we dont just discover
predefined ideas by randomly sampling bags of words, but we have to work with
hard labor, sweat, and tears, to imbue the sentences of mathematics with meaning
that makes them descriptive, that there is some kind of internal, but
distributed, mental process which is mirror whats on paper (and the 'reality' it
describes)

relate this to HoTT as a perfect 'case study' in the foundations of mathematics


\subsection{other authors}

The question 

We note that 

NaProche

he Naproche project (Natural language Proof Checking) studies the semi-formal language of mathematics from a linguistic, philosophical and mathematical perspective. A central goal of Naproche is to develop a controlled natural language (CNL) for mathematical texts and adapted proof checking software which checks texts written in the CNL for syntactical and mathematical correctness. 

Uses Automated Theorem Prover (ATP) backend rather than ITP


Mizar

Coq (coquand), 

Boxer
Boxer system [29] is able to parse any English sentence and translate it into a formula
of predicate calculus. Combined with theorem proving and model checking, Boxer can
moreover solve problems in open-domain textual entailment by formal reasoning [30].
The problem that remains with Boxer is that the quality of formalization and rea-


\section{Preliminaries}

We give brief but relevant overviews of the background ideas and tools that went
into the generation of this thesis. 

% from blog post
\subsection{Martin-Löf Type Theory}
\subsubsection{Judgments}

A central contribution of Per Martin-Löf in the development of type theory was
the recognition of the centrality of judgments in logic. Many mathematicians
aren't familiar with the spectrum of judgments available, and merely believe
they are concerned with *the* notion of truth, namely *the truth* of a
mathematical proposition or theorem. There are many judgments one can make which
most mathematicians aren't aware of or at least never mention. These include,
for instance,

\begin{itemize}[noitemsep]

\item $A$ is a proposition
\item $A$ is possible
\item $A$ is probable

\end{itemize}

These judgments are understood not in the object language in which we state our
propositions, possibilities, or probabilities, but as assertions in the
metalanguage which require evidence for us to know and believe them. Most
mathematicians may reach for their wallets if I come in and give a talk saying
it is possible that the Riemann Hypothesis is true, partially because they
already know that, and partially because it doesn't seem particularly
interesting to say that something is possible, in the same way that a physicist
may flinch if you say alchemy is possible. Most mathematicians, however, would
agree that $P = NP$ is possible but isn't probable.

For the logician these judgments may well be interesting because their may be
logics in which the discussion of possibility or probability is even more
interesting than the discussion of truth. And for the type theorist, interested
in designing and building programming languages over many various logics, these
judgments become a prime focus. The role of the type-checker in a programming
language is to present evidence for, or decide the validity of the judgments.
The four main judgments of type theory are :

\begin{itemize}[noitemsep]
\item $T$ is a type
\item $T$ and $T'$ are equal types
\item $t$ is a term of type $T$
\item $t$ and $t'$ are equal terms of type $T$
\end{itemize}


We succinctly present these in a mathematical notation where Frege's turnstile,
$\vdash$, denotes a judgment :

\begin{itemize}[noitemsep]
\item $\vdash T \; {\rm type}$
\item $\vdash T = T'$
\item $\vdash t:T$
\item $\vdash t = t':T$
\end{itemize}

These judgments become much more interesting when we add the ability for them to
be interpreted in a some context with judgment hypotheses. Given a series of
judgments $J_1,...,J_n$, denoted $\Gamma$, where $J_i$ can depend on previously
listed $J's$, we can make judgment $J$ under the hypotheses, e.g. $J_1,...,J_n
\vdash J$. Often these hypotheses $J_i$, alternatively called *antecedents*,
denote variables which may occur freely in the *consequent* judgment $J$. For
instance, the antecedent, $x : \mathbb{R}$ occurs freely in the syntactic
expression $\sin x$, a which is given meaning in the judgment $\vdash \sin x { :
} \mathbb{R}$. We write our hypothetical judgement as follows :

$x : \mathbb{R} \vdash \sin x : \mathbb{R}$

One reason why hypothetical judgments are so interesting is we can devise rules
which allow us to translate from the metalanguage to the object language using
lambda expressions. These play the role of a function in mathematics and
implication in logic. This comes out in the following introduction rule :

$ \frac{\Gamma, x : A \vdash b : B} {\Gamma \vdash \lambda x. b : A \rightarrow
B} $

Using this rule, we now see a typical judgment, typical in a field like from
real analysis,

$\vdash \lambda x. \sin x : \R \rightarrow \R$

Equality :

Mathematicians denote this judgement
\begin{align*} f {:} \mathbb{R} &\rightarrow \mathbb{R}\\ x &\mapsto \sin ( x )
\end{align*}




\subsection{Agda}

\subsubsection{Overview}

Agda is an attempt to faithfully formalize Martin-Löf's intensional type theory
\cite{ml1984} into a functional programming language . One can think of
Martin-Löf's original work as a specification of a foundational system, and Agda
as one possible implementation.

Through an interactive environment, Agda allows one to iteratively apply rules
and develop constructive mathematics. It's current incarnation, Agda2 (but just
called Agda), was preceded by ALF, Cayenne, and Alfa, and Agda1. In addition to
the core MLTT, Agda incorporates dependent records, inductive definitions with
all types of bells and whistles, pattern matching, a versatile module system,
and a myriad of other features which are of interest generally but not relevant
to this work.

We will only look at what can in some sense be seen as the kernel of Agda.
Developing a full-blown GF grammar to incorporate more advanced Agda features
would require efforts beyond the scope of this work. And while there are still
many reasons one may wish to use other programming languages, there is a sense
of purity one gets when writing Agda code. There are many good resources for
learning Agda \cite{Bove2009} \cite{stump} \cite{ulf} \cite{wadler} so we'll
only give a cursory overview of what is relevant for this thesis, with a
particular emphasis on the syntax.

\begin{code}[hide]%
\>[0]\<%
\\
\>[0]\AgdaComment{-- \{-\# OPTIONS --omega-in-omega --type-in-type \#-\}}\<%
\\
%
\\[\AgdaEmptyExtraSkip]%
\>[0]\AgdaKeyword{module}\AgdaSpace{}%
\AgdaModule{ex}\AgdaSpace{}%
\AgdaKeyword{where}\<%
\\
%
\\[\AgdaEmptyExtraSkip]%
\>[0]\AgdaKeyword{data}\AgdaSpace{}%
\AgdaDatatype{aℕ}\AgdaSpace{}%
\AgdaSymbol{:}\AgdaSpace{}%
\AgdaPrimitive{Set}\AgdaSpace{}%
\AgdaKeyword{where}\<%
\\
\>[0][@{}l@{\AgdaIndent{0}}]%
\>[2]\AgdaInductiveConstructor{zero'}\AgdaSpace{}%
\AgdaSymbol{:}\AgdaSpace{}%
\AgdaDatatype{aℕ}\<%
\\
%
\\[\AgdaEmptyExtraSkip]%
\>[0]\AgdaKeyword{variable}\<%
\\
\>[0][@{}l@{\AgdaIndent{0}}]%
\>[2]\AgdaGeneralizable{A}\AgdaSpace{}%
\AgdaSymbol{:}\AgdaSpace{}%
\AgdaPrimitive{Set}\<%
\\
%
\>[2]\AgdaGeneralizable{D}\AgdaSpace{}%
\AgdaSymbol{:}\AgdaSpace{}%
\AgdaPrimitive{Set}\<%
\\
%
\>[2]\AgdaGeneralizable{stuff}\AgdaSpace{}%
\AgdaSymbol{:}\AgdaSpace{}%
\AgdaPrimitive{Set}\<%
\\
%
\\[\AgdaEmptyExtraSkip]%
\>[0]\AgdaFunction{definition-body}\AgdaSpace{}%
\AgdaSymbol{=}\AgdaSpace{}%
\AgdaDatatype{aℕ}\<%
\\
%
\\[\AgdaEmptyExtraSkip]%
\>[0]\AgdaFunction{T}\AgdaSpace{}%
\AgdaSymbol{=}\AgdaSpace{}%
\AgdaDatatype{aℕ}\AgdaSpace{}%
\AgdaSymbol{→}\AgdaSpace{}%
\AgdaDatatype{aℕ}\<%
\\
\>[0]\AgdaFunction{L}\AgdaSpace{}%
\AgdaSymbol{=}\AgdaSpace{}%
\AgdaDatatype{aℕ}\<%
\\
\>[0]\AgdaFunction{E}\AgdaSpace{}%
\AgdaSymbol{=}\AgdaSpace{}%
\AgdaDatatype{aℕ}\<%
\\
\>[0]\AgdaFunction{C}\AgdaSpace{}%
\AgdaSymbol{=}\AgdaSpace{}%
\AgdaDatatype{aℕ}\<%
\\
%
\\[\AgdaEmptyExtraSkip]%
\>[0]\AgdaFunction{proof}\AgdaSpace{}%
\AgdaSymbol{:}\AgdaSpace{}%
\AgdaFunction{L}\<%
\\
\>[0]\AgdaFunction{proof}\AgdaSpace{}%
\AgdaSymbol{=}\AgdaSpace{}%
\AgdaInductiveConstructor{zero'}\<%
\\
%
\\[\AgdaEmptyExtraSkip]%
\>[0]\AgdaFunction{corollaryStuff}\AgdaSpace{}%
\AgdaSymbol{=}\AgdaSpace{}%
\AgdaDatatype{aℕ}\<%
\\
%
\\[\AgdaEmptyExtraSkip]%
\>[0]\AgdaFunction{proofNeedingLemma}\AgdaSpace{}%
\AgdaSymbol{:}\AgdaSpace{}%
\AgdaDatatype{aℕ}\AgdaSpace{}%
\AgdaSymbol{→}\AgdaSpace{}%
\AgdaDatatype{aℕ}\AgdaSpace{}%
\AgdaSymbol{→}\AgdaSpace{}%
\AgdaDatatype{aℕ}\<%
\\
\>[0]\AgdaFunction{proofNeedingLemma}\AgdaSpace{}%
\AgdaBound{x}\AgdaSpace{}%
\AgdaSymbol{=}\AgdaSpace{}%
\AgdaSymbol{λ}\AgdaSpace{}%
\AgdaBound{x₁}\AgdaSpace{}%
\AgdaSymbol{→}\AgdaSpace{}%
\AgdaInductiveConstructor{zero'}\<%
\\
\>[0]\<%
\end{code}

\subsubsection{Agda Programming}

Listed is the syntax Agda uses for judgements: \term{T} : \term{Set} means
\term{T} is a type, \term{t} : \term{T} means a term \term{t} has type \term{T},
and \term{t} = \term{t'} means \term{t} is defined to be judgmentally equal to
\term{t'}. Once one has made this equality judgement, Agda can normalize the
definitionally equal terms to the same normal form. Let's compare these Agda
judgements to those keywords ubiquitous in mathematics:

\begin{figure}
\centering
\begin{minipage}[t]{.3\textwidth}
\vspace{2cm}
\begin{itemize}
\item Axiom
\item Definition
\item Lemma
\item Theorem
\item Proof
\item Corollary
\item Example
\end{itemize}
\end{minipage}%
\begin{minipage}[t]{.55\textwidth}
\begin{code}%
\>[0]\AgdaKeyword{postulate}%
\>[12]\AgdaComment{-- Axiom}\<%
\\
\>[0][@{}l@{\AgdaIndent{0}}]%
\>[2]\AgdaPostulate{axiom}\AgdaSpace{}%
\AgdaSymbol{:}\AgdaSpace{}%
\AgdaGeneralizable{A}\<%
\\
%
\\[\AgdaEmptyExtraSkip]%
\>[0]\AgdaFunction{definition}\AgdaSpace{}%
\AgdaSymbol{:}\AgdaSpace{}%
\AgdaGeneralizable{stuff}\AgdaSpace{}%
\AgdaSymbol{→}\AgdaSpace{}%
\AgdaPrimitive{Set}\AgdaSpace{}%
\AgdaComment{--Definition}\<%
\\
\>[0]\AgdaFunction{definition}\AgdaSpace{}%
\AgdaBound{s}\AgdaSpace{}%
\AgdaSymbol{=}\AgdaSpace{}%
\AgdaFunction{definition-body}\<%
\\
%
\\[\AgdaEmptyExtraSkip]%
\>[0]\AgdaFunction{theorem}\AgdaSpace{}%
\AgdaSymbol{:}\AgdaSpace{}%
\AgdaFunction{T}%
\>[16]\AgdaComment{-- Theorem Statement}\<%
\\
\>[0]\AgdaFunction{theorem}\AgdaSpace{}%
\AgdaSymbol{=}\AgdaSpace{}%
\AgdaFunction{proofNeedingLemma}\AgdaSpace{}%
\AgdaFunction{lemma}\AgdaSpace{}%
\AgdaComment{-- Proof}\<%
\\
\>[0][@{}l@{\AgdaIndent{0}}]%
\>[2]\AgdaKeyword{where}\<%
\\
\>[2][@{}l@{\AgdaIndent{0}}]%
\>[4]\AgdaFunction{lemma}\AgdaSpace{}%
\AgdaSymbol{:}\AgdaSpace{}%
\AgdaFunction{L}%
\>[18]\AgdaComment{-- Lemma Statement}\<%
\\
%
\>[4]\AgdaFunction{lemma}\AgdaSpace{}%
\AgdaSymbol{=}\AgdaSpace{}%
\AgdaFunction{proof}\<%
\\
%
\\[\AgdaEmptyExtraSkip]%
\>[0]\AgdaFunction{corollary}\AgdaSpace{}%
\AgdaSymbol{:}\AgdaSpace{}%
\AgdaFunction{corollaryStuff}\AgdaSpace{}%
\AgdaSymbol{→}\AgdaSpace{}%
\AgdaFunction{C}\<%
\\
\>[0]\AgdaFunction{corollary}\AgdaSpace{}%
\AgdaBound{coro-term}\AgdaSpace{}%
\AgdaSymbol{=}\AgdaSpace{}%
\AgdaFunction{theorem}\AgdaSpace{}%
\AgdaBound{coro-term}\<%
\\
%
\\[\AgdaEmptyExtraSkip]%
\>[0]\AgdaFunction{example}\AgdaSpace{}%
\AgdaSymbol{:}\AgdaSpace{}%
\AgdaFunction{E}%
\>[16]\AgdaComment{-- Example Statement}\<%
\\
\>[0]\AgdaFunction{example}\AgdaSpace{}%
\AgdaSymbol{=}\AgdaSpace{}%
\AgdaFunction{proof}\<%
\end{code}
\end{minipage}
\caption{Mathematical Assertions and Agda Judgements} \label{fig:O1}
\end{figure}

Formation rules are given by the first line of the data declaration, followed
by some number of constructors which correspond to the introduction forms of the
type being defined. Therefore, to define a type for Booleans, $𝔹$, we present
these rules both in the proof theoretic and Agda syntax. We note that the
context $\Gamma$ is not present in Agda.

\begin{minipage}[t]{.4\textwidth}
\vspace{3mm}
\[
  \begin{prooftree}
    \infer1[]{ \vdash 𝔹 : {\rm type}}
  \end{prooftree}
\]
\[
  \begin{prooftree}
    \infer1[]{ \Gamma \vdash true : 𝔹  }
  \end{prooftree}
  \quad \quad
  \begin{prooftree}
    \infer1[]{ \Gamma \vdash false : 𝔹  }
  \end{prooftree}
\]
\end{minipage}
\begin{minipage}[t]{.3\textwidth}
\begin{code}%
\>[0]\AgdaKeyword{data}\AgdaSpace{}%
\AgdaDatatype{𝔹}\AgdaSpace{}%
\AgdaSymbol{:}\AgdaSpace{}%
\AgdaPrimitive{Set}\AgdaSpace{}%
\AgdaKeyword{where}\AgdaSpace{}%
\AgdaComment{-- formation rule}\<%
\\
\>[0][@{}l@{\AgdaIndent{0}}]%
\>[2]\AgdaInductiveConstructor{true}%
\>[8]\AgdaSymbol{:}\AgdaSpace{}%
\AgdaDatatype{𝔹}\AgdaSpace{}%
\AgdaComment{-- introduction rule}\<%
\\
%
\>[2]\AgdaInductiveConstructor{false}\AgdaSpace{}%
\AgdaSymbol{:}\AgdaSpace{}%
\AgdaDatatype{𝔹}\<%
\end{code}
\end{minipage}

The elimination forms are deriveable from the introduction rules, and the
computation rules can then be extracted by via the harmonious relationship
between the introduction and elmination forms \cite{pfenningHar}. Agda's pattern
matching is equivalent to the deriveable dependently typed elimination forms
\cite{coqPat}, and one can simply pattern match on a boolean, producing multiple
lines for each constructor of the variable's type, to extract the classic
recursion principle for Booleans. The \term{if then else} statement shown below
is really just the boolean elimination form. It is not standard to include the
premises of the eqaulity rules.

\begin{minipage}[t]{.4\textwidth}
\[
  \begin{prooftree}
    \hypo{̌\Gamma \vdash A : {\rm type} }
    \hypo{\Gamma \vdash b : 𝔹 }
    \hypo{\Gamma \vdash a1 : A}
    \hypo{\Gamma \vdash a2 : A }
    \infer4[]{\Gamma \vdash boolrec\{a1;a2\}(b) : A }
  \end{prooftree}
\]
$$\Gamma \vdash boolrec\{a1;a2\}(true) \equiv a1 : A$$
$$\Gamma \vdash boolrec\{a1;a2\}(false) \equiv a2 : A$$
\end{minipage}
\hfill
\begin{minipage}[t]{.5\textwidth}
\begin{code}%
\>[0]\AgdaOperator{\AgdaFunction{if\AgdaUnderscore{}then\AgdaUnderscore{}else\AgdaUnderscore{}}}\AgdaSpace{}%
\AgdaSymbol{:}\<%
\\
\>[0][@{}l@{\AgdaIndent{0}}]%
\>[2]\AgdaSymbol{\{}\AgdaBound{A}\AgdaSpace{}%
\AgdaSymbol{:}\AgdaSpace{}%
\AgdaPrimitive{Set}\AgdaSymbol{\}}\AgdaSpace{}%
\AgdaSymbol{→}\AgdaSpace{}%
\AgdaDatatype{𝔹}\AgdaSpace{}%
\AgdaSymbol{→}\AgdaSpace{}%
\AgdaBound{A}\AgdaSpace{}%
\AgdaSymbol{→}\AgdaSpace{}%
\AgdaBound{A}\AgdaSpace{}%
\AgdaSymbol{→}\AgdaSpace{}%
\AgdaBound{A}\<%
\\
\>[0]\AgdaOperator{\AgdaFunction{if}}\AgdaSpace{}%
\AgdaInductiveConstructor{true}\AgdaSpace{}%
\AgdaOperator{\AgdaFunction{then}}\AgdaSpace{}%
\AgdaBound{a1}\AgdaSpace{}%
\AgdaOperator{\AgdaFunction{else}}\AgdaSpace{}%
\AgdaBound{a2}\AgdaSpace{}%
\AgdaSymbol{=}\AgdaSpace{}%
\AgdaBound{a1}\<%
\\
\>[0]\AgdaOperator{\AgdaFunction{if}}\AgdaSpace{}%
\AgdaInductiveConstructor{false}\AgdaSpace{}%
\AgdaOperator{\AgdaFunction{then}}\AgdaSpace{}%
\AgdaBound{a1}\AgdaSpace{}%
\AgdaOperator{\AgdaFunction{else}}\AgdaSpace{}%
\AgdaBound{a2}\AgdaSpace{}%
\AgdaSymbol{=}\AgdaSpace{}%
\AgdaBound{a2}\<%
\end{code}
\end{minipage}

When using Agda one is interactively building a proof via holes. There is an
Agda Emacs mode which enables this. Glossing over many details, we show sample
code in the proof development state prior to pattern matching on \codeword{b}.
We have a hole, \codeword{{ b }0}, and the proof state is displayed to the
right. It shows both the current context with \codeword{A, b, a1, a2}, the goal
which is something of type \codeword{A}, and what we have, \codeword{B},
represents the type of the variable in the hole.

\hfill
\begin{minipage}[t]{.4\textwidth}
\begin{verbatim}
if_then_else_ :
  {A : Set} → B → A → A → A
if b then a1 else a2 = { b }0
\end{verbatim}
\end{minipage}
\hfill
\begin{minipage}[t]{.5\textwidth}
\begin{verbatim}
Goal: A
Have: B
———————————————
a2 : A
a1 : A
b  : B
A  : Set   (not in scope)
\end{verbatim}
\end{minipage}

The interactivity is performed via emacs commands, and every time one updates
the hole with a new term, we can immediately view the next goal with an updated
context. The underscore in \term{if_then_else_} denotes the placement of the
arguements, as Agda allows mixfix operations. Agda allows for more nuanced
syntacic features like unicode. This is interesting from the \emph{concrete
syntax} perspective as the arguement placement and symbolic expressiveness makes
Agda's syntax feel more familiar to the mathematician. We also observe the use
of parametric polymorphism, namely, that we can extract a member of some
arbtitrary type \term{A} from a boolean value given two members of \term{A}.

This polymorphism allows one to implement simple programs like boolean negation,
\term{~}, and more interestingly, \term{functionalNegation}, where one can use
functions as arguements. \term{functionalNegation} is a functional, or higher
order functions, which take functions as arguements and return functions. We
also notice in \term{functionalNegation} that one can work directly with a
built-in $\lambda$ to ensure the correct return type.

\begin{code}%
\>[0]\AgdaFunction{\textasciitilde{}}\AgdaSpace{}%
\AgdaSymbol{:}\AgdaSpace{}%
\AgdaDatatype{𝔹}\AgdaSpace{}%
\AgdaSymbol{→}\AgdaSpace{}%
\AgdaDatatype{𝔹}\<%
\\
\>[0]\AgdaFunction{\textasciitilde{}}\AgdaSpace{}%
\AgdaBound{b}\AgdaSpace{}%
\AgdaSymbol{=}\AgdaSpace{}%
\AgdaOperator{\AgdaFunction{if}}\AgdaSpace{}%
\AgdaBound{b}\AgdaSpace{}%
\AgdaOperator{\AgdaFunction{then}}\AgdaSpace{}%
\AgdaInductiveConstructor{false}\AgdaSpace{}%
\AgdaOperator{\AgdaFunction{else}}\AgdaSpace{}%
\AgdaInductiveConstructor{true}\<%
\\
%
\\[\AgdaEmptyExtraSkip]%
\>[0]\AgdaFunction{functionalNegation}\AgdaSpace{}%
\AgdaSymbol{:}\AgdaSpace{}%
\AgdaDatatype{𝔹}\AgdaSpace{}%
\AgdaSymbol{→}\AgdaSpace{}%
\AgdaSymbol{(}\AgdaDatatype{𝔹}\AgdaSpace{}%
\AgdaSymbol{→}\AgdaSpace{}%
\AgdaDatatype{𝔹}\AgdaSymbol{)}\AgdaSpace{}%
\AgdaSymbol{→}\AgdaSpace{}%
\AgdaSymbol{(}\AgdaDatatype{𝔹}\AgdaSpace{}%
\AgdaSymbol{→}\AgdaSpace{}%
\AgdaDatatype{𝔹}\AgdaSymbol{)}\<%
\\
\>[0]\AgdaFunction{functionalNegation}\AgdaSpace{}%
\AgdaBound{b}\AgdaSpace{}%
\AgdaBound{f}\AgdaSpace{}%
\AgdaSymbol{=}\AgdaSpace{}%
\AgdaOperator{\AgdaFunction{if}}\AgdaSpace{}%
\AgdaBound{b}\AgdaSpace{}%
\AgdaOperator{\AgdaFunction{then}}\AgdaSpace{}%
\AgdaBound{f}\AgdaSpace{}%
\AgdaOperator{\AgdaFunction{else}}\AgdaSpace{}%
\AgdaSymbol{λ}\AgdaSpace{}%
\AgdaBound{b'}\AgdaSpace{}%
\AgdaSymbol{→}\AgdaSpace{}%
\AgdaBound{f}\AgdaSpace{}%
\AgdaSymbol{(}\AgdaFunction{\textasciitilde{}}\AgdaSpace{}%
\AgdaBound{b'}\AgdaSymbol{)}\<%
\end{code}

This simple example leads us to one of the domains our subsequent grammars will
describe, like arithmetic (see \ref{npf}). We show how to inductively define
natural numbers in Agda, with the formation and introduction rules included
beside for contrast.

\begin{minipage}[t]{.4\textwidth}
\vspace{3mm}
\[
  \begin{prooftree}
    \infer1[]{ \vdash ℕ : {\rm type}}
  \end{prooftree}
\]
\[
  \begin{prooftree}
    \infer1[]{ \Gamma \vdash 0 : ℕ  }
  \end{prooftree}
  \quad \quad
  \begin{prooftree}
    \hypo{\Gamma \vdash n : ℕ}
    \infer1[]{ \Gamma \vdash (suc\ n) : ℕ  }
  \end{prooftree}
\]
\end{minipage}
\begin{minipage}[t]{.3\textwidth}
\begin{code}%
\>[0]\AgdaKeyword{data}\AgdaSpace{}%
\AgdaDatatype{ℕ}\AgdaSpace{}%
\AgdaSymbol{:}\AgdaSpace{}%
\AgdaPrimitive{Set}\AgdaSpace{}%
\AgdaKeyword{where}\<%
\\
\>[0][@{}l@{\AgdaIndent{0}}]%
\>[2]\AgdaInductiveConstructor{zero}\AgdaSpace{}%
\AgdaSymbol{:}\AgdaSpace{}%
\AgdaDatatype{ℕ}\<%
\\
%
\>[2]\AgdaInductiveConstructor{suc}%
\>[7]\AgdaSymbol{:}\AgdaSpace{}%
\AgdaDatatype{ℕ}\AgdaSpace{}%
\AgdaSymbol{→}\AgdaSpace{}%
\AgdaDatatype{ℕ}\<%
\end{code}
\end{minipage}

This is a recursive type, whereby pattern matching over $ℕ$ allows one to use an
induction hypothesis over the subtree and gurantee termination when making
recurive calls on the function being defined. We can define a recursion
principle for $ℕ$, which gives one the power to build iterators.
Again, we include the elimination and equality rules for syntactic
juxtaposition.

\[
  \begin{prooftree}
    \hypo{̌\Gamma \vdash X : {\rm type} }
    \hypo{\Gamma \vdash n : ℕ }
    \hypo{\Gamma \vdash e₀ : X}
    \hypo{\Gamma, x : ℕ, y : X \vdash e₁ : X }
    \infer4[]{\Gamma \vdash natrec\{e\;x.y.e₁\}(n) : X }
  \end{prooftree}
\]
$$\Gamma \vdash natrec\{e₀;x.y.e₁\}(n) \equiv e₀ : X$$
$$\Gamma \vdash natrec\{e₀;x.y.e₁\}(suc\ n) \equiv e₁[x := n,y := natrec\{e₀;x.y.e₁\}(n)] : X$$
\begin{code}%
\>[0]\AgdaFunction{natrec}\AgdaSpace{}%
\AgdaSymbol{:}\AgdaSpace{}%
\AgdaSymbol{\{}\AgdaBound{X}\AgdaSpace{}%
\AgdaSymbol{:}\AgdaSpace{}%
\AgdaPrimitive{Set}\AgdaSymbol{\}}\AgdaSpace{}%
\AgdaSymbol{→}\AgdaSpace{}%
\AgdaDatatype{ℕ}\AgdaSpace{}%
\AgdaSymbol{→}\AgdaSpace{}%
\AgdaBound{X}\AgdaSpace{}%
\AgdaSymbol{→}\AgdaSpace{}%
\AgdaSymbol{(}\AgdaDatatype{ℕ}\AgdaSpace{}%
\AgdaSymbol{→}\AgdaSpace{}%
\AgdaBound{X}\AgdaSpace{}%
\AgdaSymbol{→}\AgdaSpace{}%
\AgdaBound{X}\AgdaSymbol{)}\AgdaSpace{}%
\AgdaSymbol{→}\AgdaSpace{}%
\AgdaBound{X}\<%
\\
\>[0]\AgdaFunction{natrec}\AgdaSpace{}%
\AgdaInductiveConstructor{zero}\AgdaSpace{}%
\AgdaBound{e₀}\AgdaSpace{}%
\AgdaBound{e₁}\AgdaSpace{}%
\AgdaSymbol{=}\AgdaSpace{}%
\AgdaBound{e₀}\<%
\\
\>[0]\AgdaFunction{natrec}\AgdaSpace{}%
\AgdaSymbol{(}\AgdaInductiveConstructor{suc}\AgdaSpace{}%
\AgdaBound{n}\AgdaSymbol{)}\AgdaSpace{}%
\AgdaBound{e₀}\AgdaSpace{}%
\AgdaBound{e₁}\AgdaSpace{}%
\AgdaSymbol{=}\AgdaSpace{}%
\AgdaBound{e₁}\AgdaSpace{}%
\AgdaBound{n}\AgdaSpace{}%
\AgdaSymbol{(}\AgdaFunction{natrec}\AgdaSpace{}%
\AgdaBound{n}\AgdaSpace{}%
\AgdaBound{e₀}\AgdaSpace{}%
\AgdaBound{e₁}\AgdaSymbol{)}\<%
\end{code}

Since we are in a dependently typed setting, however, we prove theorems as well
as write programs. Therefore, we can see this recursion principle as a special
case of the induction principle \term{natind}, which represents the by induction
for the natural numbers. One may notice that while the types are different, the
programs \term{natrec} and \term{natind} are actually the same, up to
α-equivalence. One can therefore, as a corollary, actually just include the type
infomation and Agda can infer the speciliazation for you, as seen in
\term{natrec'} below.

\[
  \begin{prooftree}
    \hypo{̌\Gamma, x : ℕ \vdash X : {\rm type} }
    \hypo{\Gamma \vdash n : ℕ }
    \hypo{\Gamma \vdash e₀ : X[x := 0] }
    \hypo{\Gamma, y : ℕ, z : X[x := y] \vdash e₁ : X[x := suc\ y]}
    \infer4[]{\Gamma \vdash natind\{e₀,\;x.y.e₁\}(n) : X[x := n]}
  \end{prooftree}
\]
$$\Gamma \vdash natind\{e₀;x.y.e₁\}(n) \equiv e₀ : X[x := 0]$$
$$\Gamma \vdash natind\{e₀;x.y.e₁\}(suc\ n) \equiv e₁[x := n,y := natind\{e₀;x.y.e₁\}(n)] : X[x := suc\ n]$$
\begin{code}%
\>[0]\AgdaFunction{natind}\AgdaSpace{}%
\AgdaSymbol{:}\AgdaSpace{}%
\AgdaSymbol{\{}\AgdaBound{X}\AgdaSpace{}%
\AgdaSymbol{:}\AgdaSpace{}%
\AgdaDatatype{ℕ}\AgdaSpace{}%
\AgdaSymbol{→}\AgdaSpace{}%
\AgdaPrimitive{Set}\AgdaSymbol{\}}\AgdaSpace{}%
\AgdaSymbol{→}\AgdaSpace{}%
\AgdaSymbol{(}\AgdaBound{n}\AgdaSpace{}%
\AgdaSymbol{:}\AgdaSpace{}%
\AgdaDatatype{ℕ}\AgdaSymbol{)}\AgdaSpace{}%
\AgdaSymbol{→}\AgdaSpace{}%
\AgdaBound{X}\AgdaSpace{}%
\AgdaInductiveConstructor{zero}\AgdaSpace{}%
\AgdaSymbol{→}\AgdaSpace{}%
\AgdaSymbol{((}\AgdaBound{n}\AgdaSpace{}%
\AgdaSymbol{:}\AgdaSpace{}%
\AgdaDatatype{ℕ}\AgdaSymbol{)}\AgdaSpace{}%
\AgdaSymbol{→}\AgdaSpace{}%
\AgdaBound{X}\AgdaSpace{}%
\AgdaBound{n}\AgdaSpace{}%
\AgdaSymbol{→}\AgdaSpace{}%
\AgdaBound{X}\AgdaSpace{}%
\AgdaSymbol{(}\AgdaInductiveConstructor{suc}\AgdaSpace{}%
\AgdaBound{n}\AgdaSymbol{))}\AgdaSpace{}%
\AgdaSymbol{→}\AgdaSpace{}%
\AgdaBound{X}\AgdaSpace{}%
\AgdaBound{n}\<%
\\
\>[0]\AgdaFunction{natind}\AgdaSpace{}%
\AgdaInductiveConstructor{zero}\AgdaSpace{}%
\AgdaBound{base}\AgdaSpace{}%
\AgdaBound{step}\AgdaSpace{}%
\AgdaSymbol{=}\AgdaSpace{}%
\AgdaBound{base}\<%
\\
\>[0]\AgdaFunction{natind}\AgdaSpace{}%
\AgdaSymbol{(}\AgdaInductiveConstructor{suc}\AgdaSpace{}%
\AgdaBound{n}\AgdaSymbol{)}\AgdaSpace{}%
\AgdaBound{base}\AgdaSpace{}%
\AgdaBound{step}\AgdaSpace{}%
\AgdaSymbol{=}\AgdaSpace{}%
\AgdaBound{step}\AgdaSpace{}%
\AgdaBound{n}\AgdaSpace{}%
\AgdaSymbol{(}\AgdaFunction{natind}\AgdaSpace{}%
\AgdaBound{n}\AgdaSpace{}%
\AgdaBound{base}\AgdaSpace{}%
\AgdaBound{step}\AgdaSymbol{)}\<%
\\
%
\\[\AgdaEmptyExtraSkip]%
\>[0]\AgdaFunction{natrec'}\AgdaSpace{}%
\AgdaSymbol{:}\AgdaSpace{}%
\AgdaSymbol{\{}\AgdaBound{X}\AgdaSpace{}%
\AgdaSymbol{:}\AgdaSpace{}%
\AgdaPrimitive{Set}\AgdaSymbol{\}}\AgdaSpace{}%
\AgdaSymbol{→}\AgdaSpace{}%
\AgdaDatatype{ℕ}\AgdaSpace{}%
\AgdaSymbol{→}\AgdaSpace{}%
\AgdaBound{X}\AgdaSpace{}%
\AgdaSymbol{→}\AgdaSpace{}%
\AgdaSymbol{(}\AgdaDatatype{ℕ}\AgdaSpace{}%
\AgdaSymbol{→}\AgdaSpace{}%
\AgdaBound{X}\AgdaSpace{}%
\AgdaSymbol{→}\AgdaSpace{}%
\AgdaBound{X}\AgdaSymbol{)}\AgdaSpace{}%
\AgdaSymbol{→}\AgdaSpace{}%
\AgdaBound{X}\<%
\\
\>[0]\AgdaFunction{natrec'}\AgdaSpace{}%
\AgdaSymbol{=}\AgdaSpace{}%
\AgdaFunction{natind}\<%
\end{code}
We will defer the details of using induction and recursion principles for later
when we actually give examples of pidgin proofs some of our grammars can
handle.


\begin{code}[hide]%
\>[0]\AgdaKeyword{module}\AgdaSpace{}%
\AgdaModule{twin-primes}\AgdaSpace{}%
\AgdaKeyword{where}\<%
\\
%
\\[\AgdaEmptyExtraSkip]%
\>[0]\AgdaKeyword{open}\AgdaSpace{}%
\AgdaKeyword{import}\AgdaSpace{}%
\AgdaModule{Data.Nat}\AgdaSpace{}%
\AgdaKeyword{renaming}\AgdaSpace{}%
\AgdaSymbol{(}\AgdaOperator{\AgdaPrimitive{\AgdaUnderscore{}+\AgdaUnderscore{}}}\AgdaSpace{}%
\AgdaSymbol{to}\AgdaSpace{}%
\AgdaOperator{\AgdaPrimitive{\AgdaUnderscore{}∔\AgdaUnderscore{}}}\AgdaSymbol{)}\<%
\\
\>[0]\AgdaKeyword{open}\AgdaSpace{}%
\AgdaKeyword{import}\AgdaSpace{}%
\AgdaModule{Data.Product}\AgdaSpace{}%
\AgdaKeyword{using}\AgdaSpace{}%
\AgdaSymbol{(}\AgdaRecord{Σ}\AgdaSymbol{;}\AgdaSpace{}%
\AgdaOperator{\AgdaFunction{\AgdaUnderscore{}×\AgdaUnderscore{}}}\AgdaSymbol{;}\AgdaSpace{}%
\AgdaOperator{\AgdaInductiveConstructor{\AgdaUnderscore{},\AgdaUnderscore{}}}\AgdaSymbol{;}\AgdaSpace{}%
\AgdaField{proj₁}\AgdaSymbol{;}\AgdaSpace{}%
\AgdaField{proj₂}\AgdaSymbol{;}\AgdaSpace{}%
\AgdaFunction{∃}\AgdaSymbol{;}\AgdaSpace{}%
\AgdaFunction{Σ-syntax}\AgdaSymbol{;}\AgdaSpace{}%
\AgdaFunction{∃-syntax}\AgdaSymbol{)}\<%
\\
\>[0]\AgdaKeyword{open}\AgdaSpace{}%
\AgdaKeyword{import}\AgdaSpace{}%
\AgdaModule{Data.Sum}\AgdaSpace{}%
\AgdaKeyword{renaming}\AgdaSpace{}%
\AgdaSymbol{(}\AgdaOperator{\AgdaDatatype{\AgdaUnderscore{}⊎\AgdaUnderscore{}}}\AgdaSpace{}%
\AgdaSymbol{to}\AgdaSpace{}%
\AgdaOperator{\AgdaDatatype{\AgdaUnderscore{}+\AgdaUnderscore{}}}\AgdaSymbol{)}\<%
\\
\>[0]\AgdaKeyword{import}\AgdaSpace{}%
\AgdaModule{Relation.Binary.PropositionalEquality}\AgdaSpace{}%
\AgdaSymbol{as}\AgdaSpace{}%
\AgdaModule{Eq}\<%
\\
\>[0]\AgdaKeyword{open}\AgdaSpace{}%
\AgdaModule{Eq}\AgdaSpace{}%
\AgdaKeyword{using}\AgdaSpace{}%
\AgdaSymbol{(}\AgdaOperator{\AgdaDatatype{\AgdaUnderscore{}≡\AgdaUnderscore{}}}\AgdaSymbol{;}\AgdaSpace{}%
\AgdaInductiveConstructor{refl}\AgdaSymbol{;}\AgdaSpace{}%
\AgdaFunction{trans}\AgdaSymbol{;}\AgdaSpace{}%
\AgdaFunction{sym}\AgdaSymbol{;}\AgdaSpace{}%
\AgdaFunction{cong}\AgdaSymbol{;}\AgdaSpace{}%
\AgdaFunction{cong-app}\AgdaSymbol{;}\AgdaSpace{}%
\AgdaFunction{subst}\AgdaSymbol{)}\<%
\\
\>[0]\AgdaKeyword{open}\AgdaSpace{}%
\AgdaModule{Eq.≡-Reasoning}\AgdaSpace{}%
\AgdaKeyword{using}\AgdaSpace{}%
\AgdaSymbol{(}\AgdaOperator{\AgdaFunction{begin\AgdaUnderscore{}}}\AgdaSymbol{;}\AgdaSpace{}%
\AgdaOperator{\AgdaFunction{\AgdaUnderscore{}≡⟨⟩\AgdaUnderscore{}}}\AgdaSymbol{;}\AgdaSpace{}%
\AgdaFunction{step-≡}\AgdaSymbol{;}\AgdaSpace{}%
\AgdaOperator{\AgdaFunction{\AgdaUnderscore{}∎}}\AgdaSymbol{)}\<%
\\
%
\\[\AgdaEmptyExtraSkip]%
\>[0]\AgdaOperator{\AgdaFunction{\AgdaUnderscore{}-\AgdaUnderscore{}}}\AgdaSpace{}%
\AgdaSymbol{:}\AgdaSpace{}%
\AgdaDatatype{ℕ}\AgdaSpace{}%
\AgdaSymbol{→}\AgdaSpace{}%
\AgdaDatatype{ℕ}\AgdaSpace{}%
\AgdaSymbol{→}\AgdaSpace{}%
\AgdaDatatype{ℕ}\<%
\\
\>[0]\AgdaBound{n}%
\>[6]\AgdaOperator{\AgdaFunction{-}}\AgdaSpace{}%
\AgdaInductiveConstructor{zero}\AgdaSpace{}%
\AgdaSymbol{=}\AgdaSpace{}%
\AgdaBound{n}\<%
\\
\>[0]\AgdaInductiveConstructor{zero}%
\>[6]\AgdaOperator{\AgdaFunction{-}}\AgdaSpace{}%
\AgdaInductiveConstructor{suc}\AgdaSpace{}%
\AgdaBound{m}\AgdaSpace{}%
\AgdaSymbol{=}\AgdaSpace{}%
\AgdaInductiveConstructor{zero}\<%
\\
\>[0]\AgdaInductiveConstructor{suc}\AgdaSpace{}%
\AgdaBound{n}\AgdaSpace{}%
\AgdaOperator{\AgdaFunction{-}}\AgdaSpace{}%
\AgdaInductiveConstructor{suc}\AgdaSpace{}%
\AgdaBound{m}\AgdaSpace{}%
\AgdaSymbol{=}\AgdaSpace{}%
\AgdaBound{n}\AgdaSpace{}%
\AgdaOperator{\AgdaFunction{-}}\AgdaSpace{}%
\AgdaBound{m}\<%
\end{code}
\subsubsection{Formalizing The Twin Prime Conjecture}

Inspired by Escardos's formalization of the twin primes conjecture \cite{escardó2020introduction}, we intend to
demonstrate that while formalizing mathematics can be rewarding, it can also
create immense difficulties, especially if one wishes to do it in a way that
prioritizes natural language. The conjecture is incredibly compact

\begin{lem}
There are infinitely many twin primes.
\end{lem}

Somebody reading for the first time might then pose the immediate question : what is a twin prime?

\begin{definition}\label{def:def1}
A \emph{twin prime} is a prime number that is either 2 less or 2 more than another prime number
\end{definition}

Below Escardo's code is reproduced.
\begin{code}%
\>[0]\AgdaFunction{isPrime}\AgdaSpace{}%
\AgdaSymbol{:}\AgdaSpace{}%
\AgdaDatatype{ℕ}\AgdaSpace{}%
\AgdaSymbol{→}\AgdaSpace{}%
\AgdaPrimitive{Set}\<%
\\
\>[0]\AgdaFunction{isPrime}\AgdaSpace{}%
\AgdaBound{n}\AgdaSpace{}%
\AgdaSymbol{=}\<%
\\
\>[0][@{}l@{\AgdaIndent{0}}]%
\>[2]\AgdaSymbol{(}\AgdaBound{n}\AgdaSpace{}%
\AgdaOperator{\AgdaFunction{≥}}\AgdaSpace{}%
\AgdaNumber{2}\AgdaSymbol{)}\AgdaSpace{}%
\AgdaOperator{\AgdaFunction{×}}\<%
\\
%
\>[2]\AgdaSymbol{((}\AgdaBound{x}\AgdaSpace{}%
\AgdaBound{y}\AgdaSpace{}%
\AgdaSymbol{:}\AgdaSpace{}%
\AgdaDatatype{ℕ}\AgdaSymbol{)}\AgdaSpace{}%
\AgdaSymbol{→}\AgdaSpace{}%
\AgdaBound{x}\AgdaSpace{}%
\AgdaOperator{\AgdaPrimitive{*}}\AgdaSpace{}%
\AgdaBound{y}\AgdaSpace{}%
\AgdaOperator{\AgdaDatatype{≡}}\AgdaSpace{}%
\AgdaBound{n}\AgdaSpace{}%
\AgdaSymbol{→}\AgdaSpace{}%
\AgdaSymbol{(}\AgdaBound{x}\AgdaSpace{}%
\AgdaOperator{\AgdaDatatype{≡}}\AgdaSpace{}%
\AgdaNumber{1}\AgdaSymbol{)}\AgdaSpace{}%
\AgdaOperator{\AgdaDatatype{+}}\AgdaSpace{}%
\AgdaSymbol{(}\AgdaBound{x}\AgdaSpace{}%
\AgdaOperator{\AgdaDatatype{≡}}\AgdaSpace{}%
\AgdaBound{n}\AgdaSymbol{))}\<%
\\
%
\\[\AgdaEmptyExtraSkip]%
\>[0]\AgdaFunction{twinPrimeConjecture}\AgdaSpace{}%
\AgdaSymbol{:}\AgdaSpace{}%
\AgdaPrimitive{Set}\<%
\\
\>[0]\AgdaFunction{twinPrimeConjecture}\AgdaSpace{}%
\AgdaSymbol{=}\AgdaSpace{}%
\AgdaSymbol{(}\AgdaBound{n}\AgdaSpace{}%
\AgdaSymbol{:}\AgdaSpace{}%
\AgdaDatatype{ℕ}\AgdaSymbol{)}\AgdaSpace{}%
\AgdaSymbol{→}\AgdaSpace{}%
\AgdaFunction{Σ[}\AgdaSpace{}%
\AgdaBound{p}\AgdaSpace{}%
\AgdaFunction{∈}\AgdaSpace{}%
\AgdaDatatype{ℕ}\AgdaSpace{}%
\AgdaFunction{]}\AgdaSpace{}%
\AgdaSymbol{(}\AgdaBound{p}\AgdaSpace{}%
\AgdaOperator{\AgdaFunction{≥}}\AgdaSpace{}%
\AgdaBound{n}\AgdaSymbol{)}\<%
\\
\>[0][@{}l@{\AgdaIndent{0}}]%
\>[2]\AgdaOperator{\AgdaFunction{×}}\AgdaSpace{}%
\AgdaFunction{isPrime}\AgdaSpace{}%
\AgdaBound{p}\<%
\\
%
\>[2]\AgdaOperator{\AgdaFunction{×}}\AgdaSpace{}%
\AgdaFunction{isPrime}\AgdaSpace{}%
\AgdaSymbol{(}\AgdaBound{p}\AgdaSpace{}%
\AgdaOperator{\AgdaPrimitive{∔}}\AgdaSpace{}%
\AgdaNumber{2}\AgdaSymbol{)}\<%
\end{code}

We note there are some both subtle and big differences, between the natural
language claim. First, twin prime is defined implicitly via a product
expression, \term{×}. Additionally, the ``either 2 less or 2 more" clause is
oringially read as being interpreted as having ``2 more". This reading ignores
the symmetry of products, however, and both ``p or (p ∔ 2)" could be interpreted
as the twin prime. This phenomenon makes translation highly nontrivial; however,
we will later see that PGF is capable of adding a semantic layer where the
theorem can be evaluated during the translation. Finally, this theorem doesn't
say what it is to be infinite in general, because such a definition would
require a proving a bijection with the real numbers. In this case however, we
can rely on the order of the natural numbers, to simply state what it means to
have infinitely many primes.

Despite the beauty of this, mathematicians always look for alternative, more
general ways of stating things. Generalizing the notion of a twin prime is a
prime gap. And then one immediately has to ask what is a prime gap?

\begin{definition}\label{def:def2}
A \emph{twin prime} is a prime that has a prime gap of two.
\end{definition}
\begin{definition}\label{def:def3}
A \emph{prime gap} is the difference between two successive prime numbers.
\end{definition}

Now we're stuck, at least if you want to scour the internet for the definition
of ``two successive prime numbers". That is because any mathematician will take
for granted what it means, and it would be considered a waste of time and space
to include something \emph{everyone} alternatively knows. Agda, however, must
know in order to typecheck. Below we offer a presentation which suits Agda's
needs, and matches the number theorists presentation of twin prime.


\begin{code}%
\>[0]\AgdaFunction{isSuccessivePrime}\AgdaSpace{}%
\AgdaSymbol{:}\AgdaSpace{}%
\AgdaSymbol{(}\AgdaBound{p}\AgdaSpace{}%
\AgdaBound{p'}\AgdaSpace{}%
\AgdaSymbol{:}\AgdaSpace{}%
\AgdaDatatype{ℕ}\AgdaSymbol{)}\AgdaSpace{}%
\AgdaSymbol{→}\AgdaSpace{}%
\AgdaFunction{isPrime}\AgdaSpace{}%
\AgdaBound{p}\AgdaSpace{}%
\AgdaSymbol{→}\AgdaSpace{}%
\AgdaFunction{isPrime}\AgdaSpace{}%
\AgdaBound{p'}\AgdaSpace{}%
\AgdaSymbol{→}\AgdaSpace{}%
\AgdaPrimitive{Set}\<%
\\
\>[0]\AgdaFunction{isSuccessivePrime}\AgdaSpace{}%
\AgdaBound{p}\AgdaSpace{}%
\AgdaBound{p'}\AgdaSpace{}%
\AgdaBound{x}\AgdaSpace{}%
\AgdaBound{x₁}\AgdaSpace{}%
\AgdaSymbol{=}\<%
\\
\>[0][@{}l@{\AgdaIndent{0}}]%
\>[2]\AgdaSymbol{(}\AgdaBound{p''}\AgdaSpace{}%
\AgdaSymbol{:}\AgdaSpace{}%
\AgdaDatatype{ℕ}\AgdaSymbol{)}\AgdaSpace{}%
\AgdaSymbol{→}\AgdaSpace{}%
\AgdaSymbol{(}\AgdaFunction{isPrime}\AgdaSpace{}%
\AgdaBound{p''}\AgdaSymbol{)}\AgdaSpace{}%
\AgdaSymbol{→}\<%
\\
%
\>[2]\AgdaBound{p}\AgdaSpace{}%
\AgdaOperator{\AgdaDatatype{≤}}\AgdaSpace{}%
\AgdaBound{p'}\AgdaSpace{}%
\AgdaSymbol{→}\AgdaSpace{}%
\AgdaBound{p}\AgdaSpace{}%
\AgdaOperator{\AgdaDatatype{≤}}\AgdaSpace{}%
\AgdaBound{p''}\AgdaSpace{}%
\AgdaSymbol{→}\AgdaSpace{}%
\AgdaBound{p'}\AgdaSpace{}%
\AgdaOperator{\AgdaDatatype{≤}}\AgdaSpace{}%
\AgdaBound{p''}\<%
\\
%
\\[\AgdaEmptyExtraSkip]%
\>[0]\AgdaFunction{primeGap}\AgdaSpace{}%
\AgdaSymbol{:}\<%
\\
\>[0][@{}l@{\AgdaIndent{0}}]%
\>[2]\AgdaSymbol{(}\AgdaBound{p}\AgdaSpace{}%
\AgdaBound{p'}\AgdaSpace{}%
\AgdaSymbol{:}\AgdaSpace{}%
\AgdaDatatype{ℕ}\AgdaSymbol{)}\AgdaSpace{}%
\AgdaSymbol{(}\AgdaBound{pIsPrime}\AgdaSpace{}%
\AgdaSymbol{:}\AgdaSpace{}%
\AgdaFunction{isPrime}\AgdaSpace{}%
\AgdaBound{p}\AgdaSymbol{)}\AgdaSpace{}%
\AgdaSymbol{(}\AgdaBound{p'IsPrime}\AgdaSpace{}%
\AgdaSymbol{:}\AgdaSpace{}%
\AgdaFunction{isPrime}\AgdaSpace{}%
\AgdaBound{p'}\AgdaSymbol{)}\AgdaSpace{}%
\AgdaSymbol{→}\<%
\\
%
\>[2]\AgdaSymbol{(}\AgdaFunction{isSuccessivePrime}\AgdaSpace{}%
\AgdaBound{p}\AgdaSpace{}%
\AgdaBound{p'}\AgdaSpace{}%
\AgdaBound{pIsPrime}\AgdaSpace{}%
\AgdaBound{p'IsPrime}\AgdaSymbol{)}\AgdaSpace{}%
\AgdaSymbol{→}\<%
\\
%
\>[2]\AgdaDatatype{ℕ}\<%
\\
\>[0]\AgdaFunction{primeGap}\AgdaSpace{}%
\AgdaBound{p}\AgdaSpace{}%
\AgdaBound{p'}\AgdaSpace{}%
\AgdaBound{pIsPrime}\AgdaSpace{}%
\AgdaBound{p'IsPrime}\AgdaSpace{}%
\AgdaBound{p'-is-after-p}\AgdaSpace{}%
\AgdaSymbol{=}\AgdaSpace{}%
\AgdaBound{p}\AgdaSpace{}%
\AgdaOperator{\AgdaFunction{-}}\AgdaSpace{}%
\AgdaBound{p'}\<%
\\
%
\\[\AgdaEmptyExtraSkip]%
\>[0]\AgdaFunction{twinPrime}\AgdaSpace{}%
\AgdaSymbol{:}\AgdaSpace{}%
\AgdaSymbol{(}\AgdaBound{p}\AgdaSpace{}%
\AgdaSymbol{:}\AgdaSpace{}%
\AgdaDatatype{ℕ}\AgdaSymbol{)}\AgdaSpace{}%
\AgdaSymbol{→}\AgdaSpace{}%
\AgdaPrimitive{Set}\<%
\\
\>[0]\AgdaFunction{twinPrime}\AgdaSpace{}%
\AgdaBound{p}\AgdaSpace{}%
\AgdaSymbol{=}\<%
\\
\>[0][@{}l@{\AgdaIndent{0}}]%
\>[2]\AgdaSymbol{(}\AgdaBound{pIsPrime}\AgdaSpace{}%
\AgdaSymbol{:}\AgdaSpace{}%
\AgdaFunction{isPrime}\AgdaSpace{}%
\AgdaBound{p}\AgdaSymbol{)}\AgdaSpace{}%
\AgdaSymbol{(}\AgdaBound{p'}\AgdaSpace{}%
\AgdaSymbol{:}\AgdaSpace{}%
\AgdaDatatype{ℕ}\AgdaSymbol{)}\AgdaSpace{}%
\AgdaSymbol{(}\AgdaBound{p'IsPrime}\AgdaSpace{}%
\AgdaSymbol{:}\AgdaSpace{}%
\AgdaFunction{isPrime}\AgdaSpace{}%
\AgdaBound{p'}\AgdaSymbol{)}\<%
\\
%
\>[2]\AgdaSymbol{(}\AgdaBound{p'-is-after-p}\AgdaSpace{}%
\AgdaSymbol{:}\AgdaSpace{}%
\AgdaFunction{isSuccessivePrime}\AgdaSpace{}%
\AgdaBound{p}\AgdaSpace{}%
\AgdaBound{p'}\AgdaSpace{}%
\AgdaBound{pIsPrime}\AgdaSpace{}%
\AgdaBound{p'IsPrime}\AgdaSymbol{)}\AgdaSpace{}%
\AgdaSymbol{→}\<%
\\
%
\>[2]\AgdaSymbol{(}\AgdaFunction{primeGap}\AgdaSpace{}%
\AgdaBound{p}\AgdaSpace{}%
\AgdaBound{p'}\AgdaSpace{}%
\AgdaBound{pIsPrime}\AgdaSpace{}%
\AgdaBound{p'IsPrime}\AgdaSpace{}%
\AgdaBound{p'-is-after-p}\AgdaSymbol{)}\AgdaSpace{}%
\AgdaOperator{\AgdaDatatype{≡}}\AgdaSpace{}%
\AgdaNumber{2}\<%
\\
%
\\[\AgdaEmptyExtraSkip]%
\>[0]\AgdaFunction{twinPrimeConjecture'}\AgdaSpace{}%
\AgdaSymbol{:}\AgdaSpace{}%
\AgdaPrimitive{Set}\<%
\\
\>[0]\AgdaFunction{twinPrimeConjecture'}\AgdaSpace{}%
\AgdaSymbol{=}\AgdaSpace{}%
\AgdaSymbol{(}\AgdaBound{n}\AgdaSpace{}%
\AgdaSymbol{:}\AgdaSpace{}%
\AgdaDatatype{ℕ}\AgdaSymbol{)}\AgdaSpace{}%
\AgdaSymbol{→}\AgdaSpace{}%
\AgdaFunction{Σ[}\AgdaSpace{}%
\AgdaBound{p}\AgdaSpace{}%
\AgdaFunction{∈}\AgdaSpace{}%
\AgdaDatatype{ℕ}\AgdaSpace{}%
\AgdaFunction{]}\AgdaSpace{}%
\AgdaSymbol{(}\AgdaBound{p}\AgdaSpace{}%
\AgdaOperator{\AgdaFunction{≥}}\AgdaSpace{}%
\AgdaBound{n}\AgdaSymbol{)}\<%
\\
\>[0][@{}l@{\AgdaIndent{0}}]%
\>[2]\AgdaOperator{\AgdaFunction{×}}\AgdaSpace{}%
\AgdaFunction{twinPrime}\AgdaSpace{}%
\AgdaBound{p}\<%
\end{code}

We see that \term{isSuccessivePrime} captures this meaning, interpreting
``successive" as the type of suprema in the prime number ordering. We also see that all the primality proofs must be given explicitly.

The term \term{primeGap} then has to reference this successive prime data, even
though most of it is discarded and unused in the actual program returning a
number. One could keep these unused arguements around via extra record fields,
to anticipate future programs calling \prime{Gap}, but ultimately the developer has to
decide what is relevant. A GF translation would ideally be kept as simple as possible. We also use propositional equality here, which is
another departure from classical mathematics, as will be elaborated later.

Finally, \{twinPrime} is a specialized version of \term{primeGap} to 2. ``has a
prime gap of two`` needs to be interpreted ``whose prime gap is equal to two",
and writing a GF grammar capable of disambiguating \emph{has} in mathematics
generally is likely impossible.

While working on this example, I tried to prove that 2 is prime in Agda, which turned out to be nontrivial. When I told this to an analyst (in the mathematical sense) he remarked that couldn't possibly be the case because it's something which a simple algorithm can compute (or generate). This exchange was incredibly stimulating, for the mathematian didn't know about the \emph{propositions as types} principle, and was simply taking for granted his internal computational capacity to confuse it for proof, especially in a constructive setting. He also seemed perplexed that anyone would find it interesting to prove that 2 is prime. As is hopefully revealed by this discussion, seemingly trivial things, when treated by the type theorist or linguist, can become wonderful areas of exploration.




\section{Grammatical Framework}

\subsection{Introducing GF}

A grammar specification in GF is an abstract syntax, where one specifies trees,
and a concrete syntax, where one says how the trees compositionally evaluate to
strings. Multiple concrete syntaxes may be attached to a given abstract syntax,
and these different concrete syntaxes represent different languages. An AST may
then be linearized to a string for each concrete syntax. conversely, given a
string admitted by the language being defined, GF's parser will generate all the
ASTs which linearize to that tree.

When defining a GF pipeline, one has to merely to construct an abstract syntax
file and a concrete syntax file such that they are coherent. In the abstract,
one specifies the \emph{semantics} of the domain one wants to translate over,
which is ironic, because we normally associate abstract syntax with \emph{just
syntax}. However, because GF was intended for implementing the natural language
phenomena, the types of semantic categories (or sorts) can grow much bigger than
is desirable in a programming language, where minimalism is generally favored.
The \emph{foods grammar} is the \emph{hello world} of GF, and should be referred
to for those interested in example of how the abstract syntax serves as a
semantic space in non-formal NL applications \cite{ranta2011grammatical}.

Let us revisit the ``tetrahedral doctrine", now restricting our attention to the
subset of linguistics which GF occupies. We first examine how GF fits into the
trinity, as seen in \autoref{fig:G1}. A GF abstract syntax with dependent types
can just be seen as an implementation of MLTT with the added bonus of a parser
once one specifies the linearizations. Additionally, GF is a relatively minimal
type theory, and therefore it would be easy to construct a model in a general
purpose programming language, like Agda. Embeddings of GF already exist in Coq
\cite{bernardy-chatzikyriakidis-2017-type}, Haskell \cite{angelov2010pgf}, and
MMT \cite{Kohlhase_2019}. These applications allow one to use GF's parser so
that a GF AST may be transformed into some notion of inductively defined tree
these languages all support. From the logical side, we note that GF's parser
specification was done using inference rules \cite{angelov2010phd}. Given the
coupling of Context-Free Grammars (CFGs) and operads (also known as
multicategories) \cite{lambek1989multicategories} \cite{705656} one could use
much more advanced mathematical machinery to articulate a categorical semantics
of GF.

% We sketch this briefly
% below [refer].

\begin{figure}[H]
\centering
\begin{tikzcd}
     &  &  & Logic                                                                                                                                             &  &  &            \\
     &  &  &                                                                                                                                                   &  &  &            \\
     &  &  & GF \arrow[uu, "GF\ Parser\ Specification"'] \arrow[llldd, "Theory\ of\ Operads"']
     \arrow[rrrdd, "Implementation\ of", bend left] \arrow[rrrdd, "Agda\ Embedding", bend right] &  &  &            \\
     &  &  &                                                                                                                                                   &  &  &            \\
Math &  &  &                                                                                                                                                   &  &  & CS\ (MLTT)
\end{tikzcd}
\caption{Models of GF} \label{fig:G1}
\end{figure}

One can additionally model these domains in GF. In \autoref{fig:G2}, we see that
there are 3 grammars which allow one to translate in the Trinitarian domains. Ranta's
grammar from CADE 2011 \cite{rantaLog} built a propositional framework with a core grammar
extended with other categories to capture syntactic nuance. Ranta's grammar from
the Stockholm University mathematics seminar in 2014 \cite{aarneHott} took verbatim text from a
publication of Peter Aczel and sought to show that all the syntactic nuance by
constructing a grammar capable of NL translation. Finally, our work takes a Backus-Naur
Form Converter (BNFC) \cite{bnfc}
grammar for the cubicalTT programming language \cite{cubicaltt}, GFifies it,
producing an unambiguous grammar \cite{warrickCub}.

\begin{figure}[H]
\centering
\begin{tikzcd}
                                              &  &  & Logic \arrow[dd, "Ranta\
                                              Logic\ (CADE\ 11)"] &  &  &                                       \\
                                              &  &  &                                          &  &  &                                       \\
                                              &  &  & GF                                       &  &  &                                       \\
                                              &  &  &                                          &  &  &                                       \\
Math \arrow[rrruu, "Ranta\ (HoTT\ 14)"] &  &  &                                          &  &  & CS\ (MLTT) \arrow[llluu, "cubicalTT"]
\end{tikzcd}
\caption{Trinitarian Grammars} \label{fig:G2}
\end{figure}

While these three grammars offer the most poignant points of comparison between
the computational, logical, and mathematical phenomena they attempt to capture,
we also note that there were many other smaller grammars developed during the
course of this work to supplement and experiment with various ideas presented.
Importantly, the ``Trinitarian Grammars" do not only model these different
domains, but they each do so in a unique way, making compromises and capturing
various linguistic and formal language phenomena. The phenomena should be seen
on a spectrum of semantic adequacy and syntactic completeness, as
in \autoref{fig:G3}.


\begin{figure}[H]
\centering
\begin{tikzcd}
Lexicon\ Size                                                                                                                                          &  &  & Syntactic\ Completeness \\
                                                                                                                                                       &  &  & {}                      \\
                                                                                                                                                       &  &  &                         \\
Spectrum\ of\ GF \arrow[uuu, "Statistical\ Methods?"] \arrow[rrr, "Ranta\ HoTT\ '14"'] \arrow[rrruuu, "cubicalTT"] \arrow[rrruu, "Ranta\ Logic\ '11"'] &  &  & Semantic\ Adequacy
\end{tikzcd}
\caption{The Grammatical Dimension} \label{fig:G3}
\end{figure}

The cubicalTT grammar, seeking syntactic completeness, only has a pidgin
English syntax, and therefore is only to be used for parsing a programming language.
Ranta's HoTT grammar on the other hand, while capable of presenting a
quasi-logical form, would require extensive refactoring in order to transform
the ASTs to something that resembles the ASTs of a programming language. The
Logic grammar, which produces logically coherent and linguistically nuanced
expressions, does not yet cover proofs, and therefore would require additional extensions
to actually express an Agda program. Finally, we note that large-scale
coverage of linguistic phenomena for any of these grammars will additionally
need to incorporate statistical methods in some way. 

GF has been show to exist in the PMCFG class of languages \cite{seki91pmcfg},
between CFGs and context sensitive grammars on the Chomsky Hierarchy
\cite{chomsky1956hierarchy} Thus, the `abstract` and `concrete` coupling is
relatively tight, the evaluation is quite simple, and the programs may suggest
ways of ``writing themselves" once the correct linearization types are chosen.
This is not to say GF programming is easier than in other languages, because
often there are unforeseen constraints that the programmer must get used to,
limiting the available palette. These constraints allow for fast parsing, but
greatly limit the sorts of programs one often thinks of writing.

\subsection{GF's Technicalities}

GF is a very powerful, yet simple system. GF requires the programmer to work
with, in some sense, an incredibly stiff set of constraints compared to general
purpose languages, and therefore its lack of expressiveness requires a different
way of thinking about programming.

The two functions displayed in \autoref{fig:N2}, $Parse : \{Strings\}
\rightarrow \{\{ASTs\}\}$ and $Linearize : \{ASTs\} \rightarrow \{Strings\}$, obey
the important property that :

 $$\forall s \in \{Strings\}. \forall x \in (Parse(s)). Linearize(x) \equiv s$$

Both the $\{Strings\}$ and $\{\{ASTs\}\}$ are really parameterized by a grammar
$G$. This property seems somewhat natural from the programmers perspective. The
limitation on ASTs to linearize uniquely is actually a benefit, because it saves
the user having to make a choice about a translation (although, again, a
statistical mechanism could alleviate this constraint). We also want our
translations to be well-behaved mathematically, i.e. composing $Linearize$ and
$Parse$ ad infinitum should presumably not diverge. Parsing a GF grammar is done
in polynomial time, whereby the degree of the polynomial depends on the grammar
\cite{angelov2010phd} . It comes equipped with 6 basic judgments:

\begin{itemize}[noitemsep]
  \item Abstract : \term{cat} and \term{fun}
  \item Concrete : \term{lincat}, \term{lin}, and \term{param}
  \item Auxiliary : \term{oper}
\end{itemize}

There are two judgments in an abstract file, for categories and named functions
defined over those categories, namely \term{cat} and \term{fun}. The categories
are just (succinct) names. GF dependent types arise as categories which are
parameterized over other categories and thereby allow for more fine-grained
semantic distinctions. We emphasize that GF's dependent types can be used to
implement a programming language which only parses well-typed terms (and can
actually compute with them using auxiliary declarations).

\subsubsection{Gödel's T in GF} \label{godel}

In a simply typed programming language we can choose categories, for variables,
types and expressions. One can then define the functions for the simply typed
lambda calculus extended with natural numbers, known as Gödel's T.

\begin{verbatim} 
cat
  Typ ; Exp ; Var ;
fun
  Tarr : Typ -> Typ -> Typ ;
  Tnat : Typ ;

  Evar : Var -> Exp ;
  Elam : Var -> Typ -> Exp -> Exp ;
  Eapp : Exp -> Exp -> Exp ;

  Ezer : Exp ;
  Esuc : Exp -> Exp ;
  Enatrec : Exp -> Exp -> Exp ->  Exp ;

  X : Var ;
  Y : Var ;
  F : Var ;
  IntV : Int -> Var ;
\end{verbatim}

So far we have specified how to form expressions : types built out of possibly
higher order functions between natural numbers, and expressions built out of
variables, $\lambda$, application, $0$, the successor function, and recursion
principle. The variables are kept as a separate syntactic category, and
integers, \codeword{Int}, are predefined. They allow one to
parse numeric expressions. One may then define a functional which takes a
function over the natural numbers and returns that function applied to $1$ - the
AST for this expression is :

\begin{verbatim} 
Elam
    F
    Tarr
        Tnat Tnat
      Eapp
        Evar
            F
        Evar
            IntV
                1
\end{verbatim} 

Dual to the abstract syntax there are parallel judgments when defining a
concrete syntax in GF, \term{lincat} and \term{lin} corresponding to \term{cat}
and \term{fun}, respectively. If an AST is the specification, the concrete form
is its implementation in a given lanaguage. The \term{lincat} serves to give
linearization types which are quite simply either strings, records (products
which support sub-typing and named fields), or tables (coproducts) which can
make choices when computing with arbitrarily named parameters. Parameters are
naturally isomorphic to the sets of some finite cardinality. The tables are
actually derivable from the records and their projections, which is how PGF is
defined internally, but they are so fundamental to GF programming and
expressiveness that they merit syntactic distincion. The \term{lin} is a term
which matches the type signature of the \term{fun} with which it shares a name.

If we assume we are just working with strings, then we can simply define the
functions as recursively concatenating \codeword{++} strings. The lambda function
for pidgin English then has, as its linearization form as follows :

\begin{verbatim}
lin 
  Elam v t e = "function taking" ++ v ++ "in" ++ t ++ "to" ++ e ;
\end{verbatim}

Once all the relevant function are giving correct linearizations, one can now
parse and linearize to the abstract syntax tree above the to string ``function
taking f in the natural numbers to the natural numbers to apply f to 1". This is
clearly unnatural for a variety of reasons, but it's an approximation of what
a computer scientist might say. Suppose instead, we choose to linearize this same
expression to a pidgin expression modeled off Haskell's syntax,
``$\backslash\backslash$ ( f : nat -> nat ) -> f 1". We should notice the absence of parentheses for
application suggest something more subtle is happening with the linearization
process, for normally programming languages use fixity declarations to avoid
lispy looking code. Here are the linearization functions for our Haskell-like
$\lambda$-terms:

\begin{verbatim}
lincat
  Typ = TermPrec ;
  Exp = TermPrec ;
lin
  Elam v t e = 
    mkPrec 0 ("\\" ++ parenth (v ++ ":" ++ usePrec 0 t) ++ "->" ++ usePrec 0 e) ;
  Eapp = infixl 2 "" ;
\end{verbatim}

Where did \codeword{TermPrec}, \codeword{infixl}, \codeword{parenth},
\codeword{mkPrec}, and \codeword{usePrec} come from? These are all functions
defined in GF's standard library, the RGL \cite{ranta2009rgl}. We show a few of
them below, thereby introducing the final, main GF judgments \term{param} and
\term{oper} for parameters and operations.

\begin{verbatim}
param 
  Bool = True | False ;
oper
  TermPrec : Type = {s : Str ; p : Prec} ;
  usePrec : Prec -> TermPrec -> Str = \p,x ->
    case lessPrec x.p p of {
      True => parenth x.s ;
      False => parenthOpt x.s
    } ;
  parenth : Str -> Str = \s -> "(" ++ s ++ ")" ;
  parenthOpt : Str -> Str = \s -> variants {s ; "(" ++ s ++ ")"} ;
\end{verbatim}

Parameters in GF are data types with nullary constructors - or something
isomorphic to them. Operations, on the other hand, encode the logic of GF
linearization rules. They are syntactic sugar - they allow one to abstract the
function bodies of \term{lin}s and \term{lincat}s so that one may keep the
actual linearization rules looking clean. Since GF also support \term{oper}
overloading, one can often get away with often deceptively sleek looking
linearizations, and this is a key feature of the RGL. The
use of \codeword{variants} is one of the ways to encode multiple linearizations
forms for a given tree, so here, for example, we're breaking the key nice
property from above.

This more or less resembles a typical programming language, with very little
deviation from what when would expect specifying something in Twelf
\cite{twelf}. Nonetheless, because this is both meant to somehow capture the
logical form in addition to the surface appearance of a language, the separation
of concerns leaves the user with an important decision to make regarding how one
couples the linear and abstract syntaxes. There are in some sense two extremes
one can take to get a well performing GF grammar.

Suppose you have a page of text from some random source of length $l$, and you
take it as an exercise to build a GF grammar which translates it. The first
extreme approach you could take would be to give each word in the text to a
unique category, a unique function for each category bearing the word's name,
along with a single really long function with $l$ arguments for the whole sequence of
words in the text. One could then verbatim copy the words as just strings with
their corresponding names in the concrete syntax. This overfitted grammar would
fail : it wouldn't scale to other languages, wouldn't cover any texts other than
the one given to it, and wouldn't be at all informative. Alternatively, one
could create a grammar of a two categories $c$ and $s$ with two functions, $f_0
: c$ and $f_1 : c \rightarrow s$, whereby c would be given $n$ fields, each
strings, with the string given at position $i$ in $f_0$ matching $word_i$ from
the text. $f_1$ would merely concatenate it all. This grammar would be similarly
degenerate, despite also parsing the page of text.

This seemingly silly example highlights the most blatant tension the GF grammar
writer will face : how to balance syntactic and semantic content of the grammar
between the concrete and the abstract syntax. It is also highly
relevant as concerns the domain of translation, for a programming language
with minimal syntax and the mathematicians language in expressing her ideas are
on vastly different sides of this spectrum.

We claim syntactically complete grammars are much more naturally dealt with
using a simple abstract syntax. However, to take allow a syntactically complete
grammar to capture semantic nuance and requires immensely more work on the
concrete side. Semantically adequate grammars on the other hand, require
significantly more attention on the abstract side, because semantically
meaningful expressions often don't generalize - each part of an expressions
exhibits unique behaviors which can't be abstracted to apply to other parts of
the expression. Semantically complete grammars are vulnerable to over-fitting
natural language, making generating formal languages difficult. Producing a
syntactically complete expressions which doesn't overgenerate parses also
requires a lot work from the grammar writer in this case.

The subsequent examples should illuminate this tension. The problem of
merging a syntactically oriented domain like type theory with and a semantically
oriented one like natural language mathematics with the same abstract syntax poses very serious
problems, but also highlights the power and need of other features of GF, like the RGL 
and Haskell embedding made available through the PGF API \cite{angelovApi}.

The GF RGL is a library for parsing grammatically coherent language. It exists
for many different natural languages, with various levels of coverage and
grammaticality, with a core abstract syntax shared by all of them. The API allows
one to easily construct sentence level phrases once the lexicon has been
defined. The API also provides helper functions for lexical constructions.

The Haskell embedding of a GF abstract syntax is given via Angelov's PGF
library, where the categories are given ``shadow types", so that one can
transform an abstract syntax into (a possibly massive) Generalized Algebraic
Data Type (GADT). The syntax of the embedding is a GADT, \term{Tree}, with kind
\codeword{* -> *} where all the functions serve as constructors. If function
\codeword{h} returns category \codeword{c}, the Haskell constructor
\codeword{Gh} returns \codeword{Tree c}. We note that this uses the
\codeword{--haskell=gadt} flag, of which other options are available but weren't
used in this thesis.

The PGF API also allows for the Haskell user to call the parse and linearization
functions, so that once the grammar is built, one can use Haskell as an
interface with the outside world. While GF originally was conceived as allowing
computation with ASTs, using a semantic computation judgment \term{def}, this
has approach has largely been overshadowed by its Haskell embedding. Once a
grammar is embedded in Haskell, one can use general recursion, monads, and all
other types of bells and whistles produced by the functional programming
community to compute with the embedded ASTs.

We note that this further muddies the water of what syntax and semantics refer
to in the GF lexicon. Although a GF abstract syntax represents the
programmers idealized semantic domain, once embedded in Haskell, the trees now may
represent syntactic objects to be evaluated or transformed to some other
semantic domain which may or may not eventually be linked back to a GF
linearization. We will see these tools applied more directly below.

% \subsection{Mathematical Model of GF}
% Note on the construction of free monoids

% Consider a language $L$ we want to represent, and we come up with a model that we
% build as a set of categories and functions over those categories.  Let $Cat(L)$,
% denote the categories.  Also suppose we define functions such that, given an
% ordered list $x_1,...,x_n;y \in Cat(L)$ we define a set of functions,
% $Fun_L(x_1,...,x_n;y)$ defined over the categories. In gf, a function can be
% denoted something like $\phi : x_1 \rightarrow ... \rightarrow x_n$. We may compose these based
% off their arities. So, given a function $\psi \in Fun_L(y_1,,...,y_n;z)$,
% functions $\phi_1,...\phi_n$ such that $\phi_i \in Fun_L(x_{i,1},...,x_{i,m};y_i)}$ 
%  we can plug these functions in together, or nest them such that
% $$\psi \circ (\phi_1,...,\phi_n) : \rightarrow (x_{i,j}) \rightarrow (y_{i})
% \rightarrow Z$$ 

% This is how abstract syntax trees are formed. It is also worth noting that they
% obey an associativity property, namely that 

% \begin{align*}
% &\theta \circ (\psi_1 \circ (\phi_{1,1},...,\phi_{1,k_1}),...,\psi_n \circ
% (\phi_{n,1},...,\phi_{n,k_n}))\\ = &(\theta \circ \psi_1,...\psi_n) \circ (\phi_{1,1},...,\phi_{1,k_1},...,\phi_{n,1},...,\phi_{n,k_n})
% \end{align*}

% This means that trees in GF are invariant as to how they are built - we
% can build a tree from the leaves to the root or vice versa.

% Example : consider the arithmetic grammar of exponentiation, multiplication, and
% addition defined over a single category of natural number expressions, whereby
% the function symbol is to be interpreted as a string and the tensor product,
% $\otimes$ as the concatenation during evaluation. 

% $$\_\^{}\_ : \mathds{N} \to \mathds{N} \to \mathds{N}$$
% $$\_*\_ : \mathds{N} \to \mathds{N} \to \mathds{N}$$
% $$\_+\_ : \mathds{N} \to \mathds{N} \to \mathds{N}$$

% We can think of constructing the trees by partial application, i.e., 

% $(\lambda x.\: 2 \otimes \^{} \otimes x) : \mathds{N} -> \mathds{N}$

% Lets try see the constructions yielding the string $(1 + 2) \^{} (3 * 4)$.

% We can either (i) construct this as the exponent of two fully formed expressions,
% namely a sum and a product applied to some numbers, or we can first apply the
% exponent to the two binary functions, yielding a quaternary function .

% $x ++ y$
% $x \doubleplus y$
% $``x \doubleplus y"$

% \begin{align*}
% &(\lambda x,y.\: x \otimes \^{} \otimes y)\\
% &\hspace{1cm} ((\lambda x,y.\:x \otimes + \otimes y)\; 1\; 2)\\
% &\hspace{1cm} ((\lambda x,y.\: x \otimes * \otimes y)\; 3\; 4) \\
% \mapsto\; &(\lambda x,y.\: x \otimes \^{} \otimes y)\\
% &\hspace{1cm} (1 + 2)\\
% &\hspace{1cm} (3 * 4))\\
% \mapsto\; &((1 + 2) \^{} (3 * 4))\\
% \end{align}

% \begin{align*}
% &((\lambda x,y.\: x \otimes \^{} \otimes y)\\
% &\hspace{1cm} (\lambda x,y.\:x \otimes + \otimes y)\\
% &\hspace{1cm} (\lambda x,y.\: x \otimes * \otimes y)) \\
% &\hspace{1cm} 1\; 2; 3; 4; \\
% \end{align}

% (1 + 2) \^{} (3 * 4)
  

% ((\lambda x,y. x \^{} y)
%   (\lambda x,y. x + y) 
%   (\lambda x,y. x * y))
%     1 2 3 4

% ((\lambda x,y. x + y) \^{} (\lambda x,y. x * y)) 1 2 3 4
% ((\lambda x,y. x + y) \^{} (\lambda x,y. x * y)) 1 2 3 4

% (1 + 2) \^{} (3 * 4)

% and then say
% (\lambda x. 2 \^{} x) (1 + 3) * (4 + 5)
% = 
% (\lambda x. 2 \^{} x) (1 + 3) * (4 + 5)

% $(\lambda x. 2 \wedge x) : \mathds{N} -> \mathds{N}$

% and then apply it to a complex arguement, say 
% (1 + 3) * (4 + 5)
% (\lambda x. 2 ^ x) : Nat -> Nat

% where 


% \lambda y : Pow y 1 : Nat -> Nat

% (times (plus 2 3) (plus 4 5))
% (Pow \circ (1,times)) : Nat -> Nat -> Nat

% (plus 2 3) (plus 4 5)

% can either be 

% 2^(1+3)*(4+5)


%   % \sin {:} \mathbb{R} &\rightarrow \mathbb{R}\\ x &\mapsto \sin ( x )
% % \circ (\phi_1,...,\phi_n) : \rightarrow (x_{i,j}) \rightarrow (y_{i})
% % \rightarrow Z$$ 

% The two functions displayed in, \autoref{fig:N2}.  If we can loosely call String
% the set of strings freely generated osome acan be 

% for now given a single linear presentation $C^{AST}$ , where

% AST_L String_L0 denote the sets GF ASTs and Strings in the languages generated
% by the rules of L's abstract syntax and L0s compositional representation.

% $$Parse : String -> {AST}$$
% $$Linearize : AST -> String$$

% with the important property that given a string s,


% And given an AST a, we can Parse . Linearize a belongs to {AST}

% Now we should explore why the linearizations are interesting. In part, this is
% because they have arisen from the role of grammars have played in the
% intersection and interaction between computer science and linguistics at least
% since Chomsky in the 50s, and they have different understandings and utilities
% in the respective disciplines. These two discplines converge in GF, which allows
% us to talk about natural languages (NLs) from programming languages (PLs)
% perspective.




\subsection{Natural Language and Mathematics}



% should natproof and hottproof be subsections?
\section{Natural Number Proofs}

Here we open with the perhaps the most natural kind of proof one would expect, that of laws over the inductively defined natural numbers.


% remember to refactor this
\section{HoTT Proofs}

\subsection{Why HoTT for natural language?}

We note that all natural language definitions, theorems, and proofs are copied
here verbatim from the HoTT book.  This decision is admittedly arbitrary, but
does have some benefits.  We list some here : 

\begin{itemize}[noitemsep]

\item As the HoTT book was a collaborative effort, it mixes the language of
many individuals and editors, and can be seen as more ``linguistically
neutral''

\item By its very nature HoTT is interdiscplinary, conceived and constructed by
mathematicians, logicians, and computer scientists. It therefore is meant to
interface with all these discplines, and much of the book was indeed formalized
before it was written

\item It has become canonical reference in the field, and therefore benefits
from wide familiarity

\item It is open source, with publically available Latex files free for
modification and distribution

\end{itemize}

The genisis of higher type theory is a somewhat elementary observation : that
the identity type, parameterized by an arbitrary type $A$ and indexed by
elements of $A$, can actually be built iteratively from previous identities.
That is, $A$ may actually already be an identity defined over another type
$A'$, namely $A \defeq x=_{A'} y$ where $x,y:A'$. The key idea is that this
iterating identities admits a homotpical interpretation : 

\begin{itemize}[noitemsep]

\item Types are topological spaces
\item Terms are points in these space

\item Equality types $x=_{A} y$ are paths in $A$ with endpoints $x$ and $y$ in
$A$

\item Iterated equality types are paths between paths, or continuous path
deformations in some higher path space. This is, intuitively, what
mathematicians call a homotopy.

\end{itemize}

To be explicit, given a type $A$, we can form the homotopy $p=_{x=_{A} y}q$
with endpoints $p$ and $q$ inhabiting the path space $x=_{A} y$.

Let's start out by examining the inductive definition of the identity type.  We
present this definition as it appears in section 1.12 of the HoTT book.

\begin{definition}
  The formation rule says that given a type $A:\UU$ and two elements $a,b:A$, we can form the type $(\id[A]{a}{b}):\UU$ in the same universe.
  The basic way to construct an element of $\id{a}{b}$ is to know that $a$ and $b$ are the same.
  Thus, the introduction rule is a dependent function
  \[\refl{} : \prod_{a:A} (\id[A]{a}{a}) \]
  called \define{reflexivity},
  which says that every element of $A$ is equal to itself (in a specified way).  We regard $\refl{a}$ as being the
  constant path %path\indexdef{path!constant}\indexsee{loop!constant}{path, constant}
  at the point $a$.
\end{definition}

We recapitulate this definition in Agda, and treat : 

\begin{code}[hide]%
\>[0]\AgdaSymbol{\{-\#}\AgdaSpace{}%
\AgdaKeyword{OPTIONS}\AgdaSpace{}%
\AgdaPragma{--cubical}\AgdaSpace{}%
\AgdaSymbol{\#-\}}\<%
\\
%
\\[\AgdaEmptyExtraSkip]%
\>[0]\AgdaKeyword{module}\AgdaSpace{}%
\AgdaModule{hott}\AgdaSpace{}%
\AgdaKeyword{where}\<%
\end{code}

\begin{code}[hide]%
\>[0]\<%
\\
\>[0]\AgdaKeyword{module}\AgdaSpace{}%
\AgdaModule{Id}\AgdaSpace{}%
\AgdaKeyword{where}\<%
\\
\>[0]\<%
\end{code}
\begin{code}%
\>[0]\<%
\\
\>[0][@{}l@{\AgdaIndent{1}}]%
\>[2]\AgdaKeyword{data}\AgdaSpace{}%
\AgdaOperator{\AgdaDatatype{\AgdaUnderscore{}≡'\AgdaUnderscore{}}}\AgdaSpace{}%
\AgdaSymbol{\{}\AgdaBound{A}\AgdaSpace{}%
\AgdaSymbol{:}\AgdaSpace{}%
\AgdaPrimitive{Set}\AgdaSymbol{\}}\AgdaSpace{}%
\AgdaSymbol{:}\AgdaSpace{}%
\AgdaSymbol{(}\AgdaBound{a}\AgdaSpace{}%
\AgdaBound{b}\AgdaSpace{}%
\AgdaSymbol{:}\AgdaSpace{}%
\AgdaBound{A}\AgdaSymbol{)}\AgdaSpace{}%
\AgdaSymbol{→}\AgdaSpace{}%
\AgdaPrimitive{Set}\AgdaSpace{}%
\AgdaKeyword{where}\<%
\\
\>[2][@{}l@{\AgdaIndent{0}}]%
\>[4]\AgdaInductiveConstructor{r}\AgdaSpace{}%
\AgdaSymbol{:}\AgdaSpace{}%
\AgdaSymbol{(}\AgdaBound{a}\AgdaSpace{}%
\AgdaSymbol{:}\AgdaSpace{}%
\AgdaBound{A}\AgdaSymbol{)}\AgdaSpace{}%
\AgdaSymbol{→}\AgdaSpace{}%
\AgdaBound{a}\AgdaSpace{}%
\AgdaOperator{\AgdaDatatype{≡'}}\AgdaSpace{}%
\AgdaBound{a}\<%
\\
\>[0]\<%
\end{code}

\subsection{An introduction to equality}

There is already some tension brewing : most mathematicains have an intuition
for equality, that of an identitfication between two pieces of information
which intuitively must be the same thing, i.e. $2+2=4$. They might ask, what
does it mean to ``construct an element of $\id{a}{b}$''? For the mathematician
use to thinking in terms of sets $\{\id{a}{b} \mid a,b \in \mathbb{N} \}$ isn't
a well-defined notion. Due to its use of the axiom of extensionality, the set
theoretic notion of equality is, no suprise, extensional.  This means that sets
are identified when they have the same elements, and equality is therefore
external to the notion of set. To inhabit a type means to provide evidence for
that inhabitation. The reflexivity constructor is therefore a means of
providing evidence of an equality. This evidence approach is disctinctly
constructive, and a big reason why classical and constructive mathematics,
especially when treated in an intuitionistic type theory suitable for a
programming language implementation, are such different beasts.

In Martin-Löf Type Theory, there are two fundamental notions of equality,
propositional and definitional.  While propositional equality is inductively
defined (as above) as a type which may have possibly more than one inhabitant,
definitional equality, denoted $-\equiv -$ and perhaps more aptly named
computational equality, is familiarly what most people think of as equality.
Namely, two terms which compute to the same canonical form are computationally
equal. In intensional type theory, propositional equality is a weaker notion
than computational equality : all propositionally equal terms are
computationally equal. However, computational equality does not imply
propistional equality - if it does, then one enters into the space of
extensional type theory. 

Prior to the homotopical interpretation of identity types, debates about
extensional and intensional type theories centred around two features or bugs :
extensional type theory sacrificed decideable type checking, while intensional
type theories required extra beauracracy when dealing with equality in proofs.
One approach in intensional type theories treated types as setoids, therefore
leading to so-called ``Setoid Hell''. These debates reflected Martin-Löf's
flip-flopping on the issue. His seminal 1979 Constructive Mathematics and
Computer Programming, which took an extensional view, was soon betrayed by
lectures he gave soon thereafter in Padova in 1980.  Martin-Löf was a born
again intensional type theorist.  These Padova lectures were later published in
the "Bibliopolis Book", and went on to inspire the European (and Gothenburg in
particular) approach to implementing proof assitants, whereas the
extensionalists were primarily eminating from Robert Constable's group at
Cornell. 

This tension has now been at least partially resolved, or at the very least
clarified, by an insight Voevodsky was apparently most proud of : the
introduction of h-levels. We'll delegate these details for a later section, it
is mentioned here to indicate that extensional type theory was really ``set
theory'' in disguise, in that it collapses the higher path structure of
identity types. The work over the past 10 years has elucidated the intensional
and extensional positions. HoTT, by allowing higher paths, is unashamedly
intentional, and admits a collapse into the extensional universe if so desired.
We now the examine the structure induced by this propositional equality.

\subsection{All about Identity}

We start with a slight reformulation of the identity type, where the element
determining the equality is treated as a parameter rather than an index. This
is a matter of convenience more than taste, as it delegates work for Agda's
typechecker that the programmer may find a distraction. The reflexivity terms
can generally have their endpoints inferred, and therefore cuts down on the
beauracry which often obscures code. 

\begin{code}%
\>[0]\<%
\\
\>[0][@{}l@{\AgdaIndent{1}}]%
\>[2]\AgdaKeyword{data}\AgdaSpace{}%
\AgdaOperator{\AgdaDatatype{\AgdaUnderscore{}≡\AgdaUnderscore{}}}\AgdaSpace{}%
\AgdaSymbol{\{}\AgdaBound{A}\AgdaSpace{}%
\AgdaSymbol{:}\AgdaSpace{}%
\AgdaPrimitive{Set}\AgdaSymbol{\}}\AgdaSpace{}%
\AgdaSymbol{(}\AgdaBound{a}\AgdaSpace{}%
\AgdaSymbol{:}\AgdaSpace{}%
\AgdaBound{A}\AgdaSymbol{)}\AgdaSpace{}%
\AgdaSymbol{:}\AgdaSpace{}%
\AgdaBound{A}\AgdaSpace{}%
\AgdaSymbol{→}\AgdaSpace{}%
\AgdaPrimitive{Set}\AgdaSpace{}%
\AgdaKeyword{where}\<%
\\
\>[2][@{}l@{\AgdaIndent{0}}]%
\>[4]\AgdaInductiveConstructor{r}\AgdaSpace{}%
\AgdaSymbol{:}\AgdaSpace{}%
\AgdaBound{a}\AgdaSpace{}%
\AgdaOperator{\AgdaDatatype{≡}}\AgdaSpace{}%
\AgdaBound{a}\<%
\\
%
\\[\AgdaEmptyExtraSkip]%
%
\>[2]\AgdaKeyword{infix}\AgdaSpace{}%
\AgdaNumber{20}\AgdaSpace{}%
\AgdaOperator{\AgdaDatatype{\AgdaUnderscore{}≡\AgdaUnderscore{}}}\<%
\\
\>[0]\<%
\end{code}

It is of particular concern in this thesis, because it hightlights a
fundamental difference between the lingusitic and the formal approach to proof
presentation.  While the mathematician can whimsically choose to include the
reflexivity arguement or ignore it if she believes it can be inferred, the
programmer can't afford such a laxidasical attitude. Once the type has been
defined, the arguement strcuture is fixed, all future references to the
definition carefully adhere to its specification. The advantage that the
programmer does gain however, that of Agda's powerful inferential abilities,
allows for the insides to be seen via interaction windown. 

Perhaps not of much interest up front, this is incredibly important detail
which the mathematician never has to deal with explicity, but can easily make
type and term translation infeasible due to the fast and loose nature of the
mathematician's writing. Conversely, it may make Natural Language Generation
(NLG) incredibly clunky, adhering to strict rules when created sentences out of
programs. 

[ToDo, give a GF example]

A prime source of beauty in constructive mathematics arises from Gentzen's
recognition of a natural duality in the rules for introducing and using logical
connectives. The mutually coherence between introduction and elmination rules
form the basis of what has since been labeled harmony in a deductive system.
This harmony isn't just an artifact of beauty, it forms the basis for cuts in
proof normalization, and correspondingly, evaluation of terms in a programming
langauge. 

The idea is simple, each new connective, or type former, needs a means of
constructing its terms from its constiutuent parts, yielding introduction
rules. This however, isn't enough - we need a way of dissecting and using the
terms we construct. This yields an elimantion rule which can be uniquely
derived from an inductively defined type. These elimination forms yield
induction principles, or a general notion of proof by induction, when given an
interpration in mathematics. In the non-depedent case, this is known as a
recursion principle, and corresponds to recursion known by programmers far and
wide.  The proof by induction over natural numbers familiar to mathematicians
is just one special case of this induction principle at work--the power of
induction has been recognized and brought to the fore by computer scientists.

We now elaborate the most important induction principle in HoTT, namely, the
induction of an identity type.

\begin{definition}[Version 1]

Moreover, one of the amazing things about homotopy type theory is that all of the basic constructions and axioms---all of the
higher groupoid structure---arises automatically from the induction
principle for identity types.
Recall from [section 1.12]  that this says that if % \cref{sec:identity-types}
  \begin{itemize}[noitemsep]
    \item for every $x,y:A$ and every $p:\id[A]xy$ we have a type $D(x,y,p)$, and
    \item for every $a:A$ we have an element $d(a):D(a,a,\refl a)$,
  \end{itemize}
then
  \begin{itemize}[noitemsep]
    \item there exists an element $\indid{A}(D,d,x,y,p):D(x,y,p)$ for \emph{every}
    two elements $x,y:A$ and $p:\id[A]xy$, such that $\indid{A}(D,d,a,a,\refl a)
    \jdeq d(a)$.
  \end{itemize}
\end{definition}
The book then reiterates this definition, with basically no natural language,
essentially in the raw logical framework devoid of anything but dependent
function types.
\begin{definition}[Version 2]
In other words, given dependent functions
\begin{align*}
  D & :\prod_{(x,y:A)}(x= y) \; \to \; \type\\
  d & :\prod_{a:A} D(a,a,\refl{a})
\end{align*}
there is a dependent function
\[\indid{A}(D,d):\prod_{(x,y:A)}\prod_{(p:\id{x}{y})} D(x,y,p)\]
such that
\begin{equation}\label{eq:Jconv}
\indid{A}(D,d,a,a,\refl{a})\jdeq d(a)
\end{equation}
for every $a:A$.
Usually, every time we apply this induction rule we will either not care about the specific function being defined, or we will immediately give it a different name.

\end{definition}
Again, we define this, in Agda, staying as true to the syntax as possible.
\begin{code}%
\>[0]\<%
\\
\>[0][@{}l@{\AgdaIndent{1}}]%
\>[2]\AgdaFunction{J}\AgdaSpace{}%
\AgdaSymbol{:}%
\>[46I]\AgdaSymbol{\{}\AgdaBound{A}\AgdaSpace{}%
\AgdaSymbol{:}\AgdaSpace{}%
\AgdaPrimitive{Set}\AgdaSymbol{\}}\<%
\\
\>[.][@{}l@{}]\<[46I]%
\>[6]\AgdaSymbol{→}\AgdaSpace{}%
\AgdaSymbol{(}\AgdaBound{D}\AgdaSpace{}%
\AgdaSymbol{:}\AgdaSpace{}%
\AgdaSymbol{(}\AgdaBound{x}\AgdaSpace{}%
\AgdaBound{y}\AgdaSpace{}%
\AgdaSymbol{:}\AgdaSpace{}%
\AgdaBound{A}\AgdaSymbol{)}\AgdaSpace{}%
\AgdaSymbol{→}\AgdaSpace{}%
\AgdaSymbol{(}\AgdaBound{x}\AgdaSpace{}%
\AgdaOperator{\AgdaDatatype{≡}}\AgdaSpace{}%
\AgdaBound{y}\AgdaSymbol{)}\AgdaSpace{}%
\AgdaSymbol{→}%
\>[36]\AgdaPrimitive{Set}\AgdaSymbol{)}\<%
\\
%
\>[6]\AgdaSymbol{→}\AgdaSpace{}%
\AgdaSymbol{((}\AgdaBound{a}\AgdaSpace{}%
\AgdaSymbol{:}\AgdaSpace{}%
\AgdaBound{A}\AgdaSymbol{)}\AgdaSpace{}%
\AgdaSymbol{→}\AgdaSpace{}%
\AgdaSymbol{(}\AgdaBound{D}\AgdaSpace{}%
\AgdaBound{a}\AgdaSpace{}%
\AgdaBound{a}\AgdaSpace{}%
\AgdaInductiveConstructor{r}\AgdaSpace{}%
\AgdaSymbol{))}\AgdaSpace{}%
\AgdaComment{-- → (d : (a : A) → (D a a r ))}\<%
\\
%
\>[6]\AgdaSymbol{→}\AgdaSpace{}%
\AgdaSymbol{(}\AgdaBound{x}\AgdaSpace{}%
\AgdaBound{y}\AgdaSpace{}%
\AgdaSymbol{:}\AgdaSpace{}%
\AgdaBound{A}\AgdaSymbol{)}\<%
\\
%
\>[6]\AgdaSymbol{→}\AgdaSpace{}%
\AgdaSymbol{(}\AgdaBound{p}\AgdaSpace{}%
\AgdaSymbol{:}\AgdaSpace{}%
\AgdaBound{x}\AgdaSpace{}%
\AgdaOperator{\AgdaDatatype{≡}}\AgdaSpace{}%
\AgdaBound{y}\AgdaSymbol{)}\<%
\\
%
\>[6]\AgdaComment{------------------------------------}\<%
\\
%
\>[6]\AgdaSymbol{→}\AgdaSpace{}%
\AgdaBound{D}\AgdaSpace{}%
\AgdaBound{x}\AgdaSpace{}%
\AgdaBound{y}\AgdaSpace{}%
\AgdaBound{p}\<%
\\
%
\>[2]\AgdaFunction{J}\AgdaSpace{}%
\AgdaBound{D}\AgdaSpace{}%
\AgdaBound{d}\AgdaSpace{}%
\AgdaBound{x}\AgdaSpace{}%
\AgdaDottedPattern{\AgdaSymbol{.}}\AgdaDottedPattern{x}\AgdaSpace{}%
\AgdaInductiveConstructor{r}\AgdaSpace{}%
\AgdaSymbol{=}\AgdaSpace{}%
\AgdaBound{d}\AgdaSpace{}%
\AgdaBound{x}\<%
\\
\>[0]\<%
\end{code}

It should be noted that, for instance, we can choose to leave out the $d$ label
on the third line. Indeed minimizing the amount of dependent typing and using
vanilla function types when dependency is not necessary, is generally
considered ``best practice'' Agda, because it will get desugared by the time it
typechecks anyways. For the writer of the text; however, it was convenient to
define $d$ once, as there are not the same constraints on a mathematician
writing in latex. It will again, serve as a nontrivial exercise to deal with
when specifying the grammar, and will be dealt with later [ToDo add section].
It is also of note that we choose to include Martin-Löf's original name $J$, as
this is more common in the computer science literature.

Once the identity type has been defined, it is natural to develop an ``equality
calculus'',  so that we can actually use it in proof's, as well as develop the
higher groupoid structure of types. The first fact, that propositional equality
is an equivalence relation, is well motivated by needs of practical theorem
proving in Agda and the more homotopically minded mathematician. First, we show the symmetry of equality--that paths are reversible.

\begin{lem}\label{lem:opp}
  For every type $A$ and every $x,y:A$ there is a function
  \begin{equation*}
    (x= y)\to(y= x)
  \end{equation*}
  denoted $p\mapsto \opp{p}$, such that $\opp{\refl{x}}\jdeq\refl{x}$ for each $x:A$.
  We call $\opp{p}$ the \define{inverse} of $p$.
  %\indexdef{path!inverse}%
  %\indexdef{inverse!of path}%
  %\index{equality!symmetry of}%a
  %\index{symmetry!of equality}%
\end{lem}

\begin{proof}[First proof]
  Assume given $A:\UU$, and
  let $D:{\textstyle\prod_{(x,y:A)}}(x= y) \; \to \; \type$ be the type family defined by $D(x,y,p)\defeq (y= x)$.
  %$\prod_{(x:A)} \prod_{y:B}$
  In other words, $D$ is a function assigning to any $x,y:A$ and $p:x=y$ a type, namely the type $y=x$.
  Then we have an element
  \begin{equation*}
    d\defeq \lambda x.\ \refl{x}:\prod_{x:A} D(x,x,\refl{x}).
  \end{equation*}
  Thus, the induction principle for identity types gives us an element
  $\indid{A}(D,d,x,y,p): (y= x)$
  for each $p:(x= y)$.
  We can now define the desired function $\opp{(\blank)}$ to be 
  $\lambda p.\ \indid{A}(D,d,x,y,p)$, 
  i.e.\ we set 
  $\opp{p} \defeq \indid{A}(D,d,x,y,p)$.
  The conversion rule [missing reference] %rule~\eqref{eq:Jconv} 
  gives $\opp{\refl{x}}\jdeq \refl{x}$, as required.
\end{proof}
The Agda code is certainly more brief: 
\begin{code}%
\>[0]\<%
\\
\>[0][@{}l@{\AgdaIndent{1}}]%
\>[2]\AgdaOperator{\AgdaFunction{\AgdaUnderscore{}⁻¹}}\AgdaSpace{}%
\AgdaSymbol{:}\AgdaSpace{}%
\AgdaSymbol{\{}\AgdaBound{A}\AgdaSpace{}%
\AgdaSymbol{:}\AgdaSpace{}%
\AgdaPrimitive{Set}\AgdaSymbol{\}}\AgdaSpace{}%
\AgdaSymbol{\{}\AgdaBound{x}\AgdaSpace{}%
\AgdaBound{y}\AgdaSpace{}%
\AgdaSymbol{:}\AgdaSpace{}%
\AgdaBound{A}\AgdaSymbol{\}}\AgdaSpace{}%
\AgdaSymbol{→}\AgdaSpace{}%
\AgdaBound{x}\AgdaSpace{}%
\AgdaOperator{\AgdaDatatype{≡}}\AgdaSpace{}%
\AgdaBound{y}\AgdaSpace{}%
\AgdaSymbol{→}\AgdaSpace{}%
\AgdaBound{y}\AgdaSpace{}%
\AgdaOperator{\AgdaDatatype{≡}}\AgdaSpace{}%
\AgdaBound{x}\<%
\\
%
\>[2]\AgdaOperator{\AgdaFunction{\AgdaUnderscore{}⁻¹}}\AgdaSpace{}%
\AgdaSymbol{\{}\AgdaBound{A}\AgdaSymbol{\}}\AgdaSpace{}%
\AgdaSymbol{\{}\AgdaBound{x}\AgdaSymbol{\}}\AgdaSpace{}%
\AgdaSymbol{\{}\AgdaBound{y}\AgdaSymbol{\}}\AgdaSpace{}%
\AgdaBound{p}\AgdaSpace{}%
\AgdaSymbol{=}\AgdaSpace{}%
\AgdaFunction{J}\AgdaSpace{}%
\AgdaFunction{D}\AgdaSpace{}%
\AgdaFunction{d}\AgdaSpace{}%
\AgdaBound{x}\AgdaSpace{}%
\AgdaBound{y}\AgdaSpace{}%
\AgdaBound{p}\<%
\\
\>[2][@{}l@{\AgdaIndent{0}}]%
\>[4]\AgdaKeyword{where}\<%
\\
\>[4][@{}l@{\AgdaIndent{0}}]%
\>[6]\AgdaFunction{D}\AgdaSpace{}%
\AgdaSymbol{:}\AgdaSpace{}%
\AgdaSymbol{(}\AgdaBound{x}\AgdaSpace{}%
\AgdaBound{y}\AgdaSpace{}%
\AgdaSymbol{:}\AgdaSpace{}%
\AgdaBound{A}\AgdaSymbol{)}\AgdaSpace{}%
\AgdaSymbol{→}\AgdaSpace{}%
\AgdaBound{x}\AgdaSpace{}%
\AgdaOperator{\AgdaDatatype{≡}}\AgdaSpace{}%
\AgdaBound{y}\AgdaSpace{}%
\AgdaSymbol{→}\AgdaSpace{}%
\AgdaPrimitive{Set}\<%
\\
%
\>[6]\AgdaFunction{D}\AgdaSpace{}%
\AgdaBound{x}\AgdaSpace{}%
\AgdaBound{y}\AgdaSpace{}%
\AgdaBound{p}\AgdaSpace{}%
\AgdaSymbol{=}\AgdaSpace{}%
\AgdaBound{y}\AgdaSpace{}%
\AgdaOperator{\AgdaDatatype{≡}}\AgdaSpace{}%
\AgdaBound{x}\<%
\\
%
\>[6]\AgdaFunction{d}\AgdaSpace{}%
\AgdaSymbol{:}\AgdaSpace{}%
\AgdaSymbol{(}\AgdaBound{a}\AgdaSpace{}%
\AgdaSymbol{:}\AgdaSpace{}%
\AgdaBound{A}\AgdaSymbol{)}\AgdaSpace{}%
\AgdaSymbol{→}\AgdaSpace{}%
\AgdaFunction{D}\AgdaSpace{}%
\AgdaBound{a}\AgdaSpace{}%
\AgdaBound{a}\AgdaSpace{}%
\AgdaInductiveConstructor{r}\<%
\\
%
\>[6]\AgdaFunction{d}\AgdaSpace{}%
\AgdaBound{a}\AgdaSpace{}%
\AgdaSymbol{=}\AgdaSpace{}%
\AgdaInductiveConstructor{r}\<%
\\
%
\\[\AgdaEmptyExtraSkip]%
%
\>[2]\AgdaKeyword{infixr}\AgdaSpace{}%
\AgdaNumber{50}\AgdaSpace{}%
\AgdaOperator{\AgdaFunction{\AgdaUnderscore{}⁻¹}}\<%
\\
\>[0]\<%
\end{code}

While first encountering induction principles can be scary, they are actually
more mechanical than one may think. This is due to the the fact that they
uniquely compliment the introduction rules of an an inductive type, and are
simply a means of showing one can ``map out'', or derive an arbitrary type
dependent on the type which has been inductively defined. The mechanical nature
is what allows for Coq's induction tactic, and perhaps even more elegantly,
Agda's pattern matching capabilities. It is always easier to use pattern
matching for the novice Agda programmer, which almost feels like magic.
Nonetheless, for completeness sake, the book uses the induction principle for
much of Chapter 2. And pattern matching is unique to programming languages,
its elegance isn't matched in the mathematicians' lexicon.

Here is the same proof via ``natural language pattern matching'' and Agda
pattern matching:

\begin{proof}[Second proof]
  We want to construct, for each $x,y:A$ and $p:x=y$, an element $\opp{p}:y=x$.
  By induction, it suffices to do this in the case when $y$ is $x$ and $p$ is $\refl{x}$.
  But in this case, the type $x=y$ of $p$ and the type $y=x$ in which we are trying to construct $\opp{p}$ are both simply $x=x$.
  Thus, in the ``reflexivity case'', we can define $\opp{\refl{x}}$ to be simply $\refl{x}$.
  The general case then follows by the induction principle, and the conversion rule $\opp{\refl{x}}\jdeq\refl{x}$ is precisely the proof in the reflexivity case that we gave.
\end{proof}

\begin{code}%
\>[0]\<%
\\
\>[0][@{}l@{\AgdaIndent{1}}]%
\>[2]\AgdaOperator{\AgdaFunction{\AgdaUnderscore{}⁻¹'}}\AgdaSpace{}%
\AgdaSymbol{:}\AgdaSpace{}%
\AgdaSymbol{\{}\AgdaBound{A}\AgdaSpace{}%
\AgdaSymbol{:}\AgdaSpace{}%
\AgdaPrimitive{Set}\AgdaSymbol{\}}\AgdaSpace{}%
\AgdaSymbol{\{}\AgdaBound{x}\AgdaSpace{}%
\AgdaBound{y}\AgdaSpace{}%
\AgdaSymbol{:}\AgdaSpace{}%
\AgdaBound{A}\AgdaSymbol{\}}\AgdaSpace{}%
\AgdaSymbol{→}\AgdaSpace{}%
\AgdaBound{x}\AgdaSpace{}%
\AgdaOperator{\AgdaDatatype{≡}}\AgdaSpace{}%
\AgdaBound{y}\AgdaSpace{}%
\AgdaSymbol{→}\AgdaSpace{}%
\AgdaBound{y}\AgdaSpace{}%
\AgdaOperator{\AgdaDatatype{≡}}\AgdaSpace{}%
\AgdaBound{x}\<%
\\
%
\>[2]\AgdaOperator{\AgdaFunction{\AgdaUnderscore{}⁻¹'}}\AgdaSpace{}%
\AgdaSymbol{\{}\AgdaBound{A}\AgdaSymbol{\}}\AgdaSpace{}%
\AgdaSymbol{\{}\AgdaBound{x}\AgdaSymbol{\}}\AgdaSpace{}%
\AgdaSymbol{\{}\AgdaBound{y}\AgdaSymbol{\}}\AgdaSpace{}%
\AgdaInductiveConstructor{r}\AgdaSpace{}%
\AgdaSymbol{=}\AgdaSpace{}%
\AgdaInductiveConstructor{r}\<%
\\
\>[0]\<%
\end{code}

Next is trasitivity--concatenation of paths--and we omit the natural language
presentation, which is a slightly more sophisticated arguement than for
symmetry.  


\begin{code}%
\>[0][@{}l@{\AgdaIndent{1}}]%
\>[2]\AgdaOperator{\AgdaFunction{\AgdaUnderscore{}∙\AgdaUnderscore{}}}\AgdaSpace{}%
\AgdaSymbol{:}\AgdaSpace{}%
\AgdaSymbol{\{}\AgdaBound{A}\AgdaSpace{}%
\AgdaSymbol{:}\AgdaSpace{}%
\AgdaPrimitive{Set}\AgdaSymbol{\}}\AgdaSpace{}%
\AgdaSymbol{→}\AgdaSpace{}%
\AgdaSymbol{\{}\AgdaBound{x}\AgdaSpace{}%
\AgdaBound{y}\AgdaSpace{}%
\AgdaSymbol{:}\AgdaSpace{}%
\AgdaBound{A}\AgdaSymbol{\}}\AgdaSpace{}%
\AgdaSymbol{→}\AgdaSpace{}%
\AgdaSymbol{(}\AgdaBound{p}\AgdaSpace{}%
\AgdaSymbol{:}\AgdaSpace{}%
\AgdaBound{x}\AgdaSpace{}%
\AgdaOperator{\AgdaDatatype{≡}}\AgdaSpace{}%
\AgdaBound{y}\AgdaSymbol{)}\AgdaSpace{}%
\AgdaSymbol{→}\AgdaSpace{}%
\AgdaSymbol{\{}\AgdaBound{z}\AgdaSpace{}%
\AgdaSymbol{:}\AgdaSpace{}%
\AgdaBound{A}\AgdaSymbol{\}}\AgdaSpace{}%
\AgdaSymbol{→}\AgdaSpace{}%
\AgdaSymbol{(}\AgdaBound{q}\AgdaSpace{}%
\AgdaSymbol{:}\AgdaSpace{}%
\AgdaBound{y}\AgdaSpace{}%
\AgdaOperator{\AgdaDatatype{≡}}\AgdaSpace{}%
\AgdaBound{z}\AgdaSymbol{)}\AgdaSpace{}%
\AgdaSymbol{→}\AgdaSpace{}%
\AgdaBound{x}\AgdaSpace{}%
\AgdaOperator{\AgdaDatatype{≡}}\AgdaSpace{}%
\AgdaBound{z}\<%
\\
%
\>[2]\AgdaOperator{\AgdaFunction{\AgdaUnderscore{}∙\AgdaUnderscore{}}}%
\>[201I]\AgdaSymbol{\{}\AgdaBound{A}\AgdaSymbol{\}}\AgdaSpace{}%
\AgdaSymbol{\{}\AgdaBound{x}\AgdaSymbol{\}}\AgdaSpace{}%
\AgdaSymbol{\{}\AgdaBound{y}\AgdaSymbol{\}}\AgdaSpace{}%
\AgdaBound{p}\AgdaSpace{}%
\AgdaSymbol{\{}\AgdaBound{z}\AgdaSymbol{\}}\AgdaSpace{}%
\AgdaBound{q}\AgdaSpace{}%
\AgdaSymbol{=}\AgdaSpace{}%
\AgdaFunction{J}\AgdaSpace{}%
\AgdaFunction{D}\AgdaSpace{}%
\AgdaFunction{d}\AgdaSpace{}%
\AgdaBound{x}\AgdaSpace{}%
\AgdaBound{y}\AgdaSpace{}%
\AgdaBound{p}\AgdaSpace{}%
\AgdaBound{z}\AgdaSpace{}%
\AgdaBound{q}\<%
\\
\>[.][@{}l@{}]\<[201I]%
\>[6]\AgdaKeyword{where}\<%
\\
%
\>[6]\AgdaFunction{D}\AgdaSpace{}%
\AgdaSymbol{:}\AgdaSpace{}%
\AgdaSymbol{(}\AgdaBound{x₁}\AgdaSpace{}%
\AgdaBound{y₁}\AgdaSpace{}%
\AgdaSymbol{:}\AgdaSpace{}%
\AgdaBound{A}\AgdaSymbol{)}\AgdaSpace{}%
\AgdaSymbol{→}\AgdaSpace{}%
\AgdaBound{x₁}\AgdaSpace{}%
\AgdaOperator{\AgdaDatatype{≡}}\AgdaSpace{}%
\AgdaBound{y₁}\AgdaSpace{}%
\AgdaSymbol{→}\AgdaSpace{}%
\AgdaPrimitive{Set}\<%
\\
%
\>[6]\AgdaFunction{D}\AgdaSpace{}%
\AgdaBound{x}\AgdaSpace{}%
\AgdaBound{y}\AgdaSpace{}%
\AgdaBound{p}\AgdaSpace{}%
\AgdaSymbol{=}\AgdaSpace{}%
\AgdaSymbol{(}\AgdaBound{z}\AgdaSpace{}%
\AgdaSymbol{:}\AgdaSpace{}%
\AgdaBound{A}\AgdaSymbol{)}\AgdaSpace{}%
\AgdaSymbol{→}\AgdaSpace{}%
\AgdaSymbol{(}\AgdaBound{q}\AgdaSpace{}%
\AgdaSymbol{:}\AgdaSpace{}%
\AgdaBound{y}\AgdaSpace{}%
\AgdaOperator{\AgdaDatatype{≡}}\AgdaSpace{}%
\AgdaBound{z}\AgdaSymbol{)}\AgdaSpace{}%
\AgdaSymbol{→}\AgdaSpace{}%
\AgdaBound{x}\AgdaSpace{}%
\AgdaOperator{\AgdaDatatype{≡}}\AgdaSpace{}%
\AgdaBound{z}\<%
\\
%
\>[6]\AgdaFunction{d}\AgdaSpace{}%
\AgdaSymbol{:}\AgdaSpace{}%
\AgdaSymbol{(}\AgdaBound{z₁}\AgdaSpace{}%
\AgdaSymbol{:}\AgdaSpace{}%
\AgdaBound{A}\AgdaSymbol{)}\AgdaSpace{}%
\AgdaSymbol{→}\AgdaSpace{}%
\AgdaFunction{D}\AgdaSpace{}%
\AgdaBound{z₁}\AgdaSpace{}%
\AgdaBound{z₁}\AgdaSpace{}%
\AgdaInductiveConstructor{r}\<%
\\
%
\>[6]\AgdaFunction{d}\AgdaSpace{}%
\AgdaSymbol{=}\AgdaSpace{}%
\AgdaSymbol{λ}\AgdaSpace{}%
\AgdaBound{v}\AgdaSpace{}%
\AgdaBound{z}\AgdaSpace{}%
\AgdaBound{q}\AgdaSpace{}%
\AgdaSymbol{→}\AgdaSpace{}%
\AgdaBound{q}\<%
\\
%
\\[\AgdaEmptyExtraSkip]%
%
\>[2]\AgdaKeyword{infixl}\AgdaSpace{}%
\AgdaNumber{40}\AgdaSpace{}%
\AgdaOperator{\AgdaFunction{\AgdaUnderscore{}∙\AgdaUnderscore{}}}\<%
\end{code}

Putting on our spectacles, the reflexivity term serves as evidence of a
constant path for any given point of any given type. To the category theorist,
this makes up the data of an identity map. Likewise, conctanation of paths
starts to look like function composition. This, along with the identity laws
and associativity as proven below, gives us the data of a category. And we have
not only have a category, but the symmetry allows us to prove all paths are
isomorphisms, giving us a groupoid. This isn't a coincedence, it's a very deep
and fascinating articulation of power of the machinery we've so far built. The
fact the path space over a type naturally must satisfies coherence laws in an
even higher path space gives leads to this notion of types as higher groupoids.  

As regards the natural language--at this point, the bookkeeping starts to get hairy.  Paths between paths, and paths between paths between paths, these ideas start to lose geometric inutiotion. And the mathematician often fails to express, when writing $p= q$, that she is already reasoning in a path space. While clever, our brains aren't wired to do too much book-keeping.  Fortunately Agda does this for us, and we can use implicit arguements to avoid our code getting too messy.  [ToDo, add example]

We now proceed to show that we have a groupoid, where the objects are points,
the morphisms are paths. The isomorphisms arise from the path reversal.  Many
of the proofs beyond this point are either routinely made via the induction
principle, or even more routinely by just pattern matching on equality paths,
we omit the details which can be found in the HoTT book, but it is expected
that the GF parser will soon cover such examples.

\begin{code}%
%
\>[2]\AgdaFunction{iₗ}\AgdaSpace{}%
\AgdaSymbol{:}\AgdaSpace{}%
\AgdaSymbol{\{}\AgdaBound{A}\AgdaSpace{}%
\AgdaSymbol{:}\AgdaSpace{}%
\AgdaPrimitive{Set}\AgdaSymbol{\}}\AgdaSpace{}%
\AgdaSymbol{\{}\AgdaBound{x}\AgdaSpace{}%
\AgdaBound{y}\AgdaSpace{}%
\AgdaSymbol{:}\AgdaSpace{}%
\AgdaBound{A}\AgdaSymbol{\}}\AgdaSpace{}%
\AgdaSymbol{(}\AgdaBound{p}\AgdaSpace{}%
\AgdaSymbol{:}\AgdaSpace{}%
\AgdaBound{x}\AgdaSpace{}%
\AgdaOperator{\AgdaDatatype{≡}}\AgdaSpace{}%
\AgdaBound{y}\AgdaSymbol{)}\AgdaSpace{}%
\AgdaSymbol{→}\AgdaSpace{}%
\AgdaBound{p}\AgdaSpace{}%
\AgdaOperator{\AgdaDatatype{≡}}\AgdaSpace{}%
\AgdaInductiveConstructor{r}\AgdaSpace{}%
\AgdaOperator{\AgdaFunction{∙}}\AgdaSpace{}%
\AgdaBound{p}\<%
\\
%
\>[2]\AgdaFunction{iₗ}\AgdaSpace{}%
\AgdaSymbol{\{}\AgdaBound{A}\AgdaSymbol{\}}\AgdaSpace{}%
\AgdaSymbol{\{}\AgdaBound{x}\AgdaSymbol{\}}\AgdaSpace{}%
\AgdaSymbol{\{}\AgdaBound{y}\AgdaSymbol{\}}\AgdaSpace{}%
\AgdaBound{p}\AgdaSpace{}%
\AgdaSymbol{=}\AgdaSpace{}%
\AgdaFunction{J}\AgdaSpace{}%
\AgdaFunction{D}\AgdaSpace{}%
\AgdaFunction{d}\AgdaSpace{}%
\AgdaBound{x}\AgdaSpace{}%
\AgdaBound{y}\AgdaSpace{}%
\AgdaBound{p}\<%
\\
\>[2][@{}l@{\AgdaIndent{0}}]%
\>[4]\AgdaKeyword{where}\<%
\\
\>[4][@{}l@{\AgdaIndent{0}}]%
\>[6]\AgdaFunction{D}\AgdaSpace{}%
\AgdaSymbol{:}\AgdaSpace{}%
\AgdaSymbol{(}\AgdaBound{x}\AgdaSpace{}%
\AgdaBound{y}\AgdaSpace{}%
\AgdaSymbol{:}\AgdaSpace{}%
\AgdaBound{A}\AgdaSymbol{)}\AgdaSpace{}%
\AgdaSymbol{→}\AgdaSpace{}%
\AgdaBound{x}\AgdaSpace{}%
\AgdaOperator{\AgdaDatatype{≡}}\AgdaSpace{}%
\AgdaBound{y}\AgdaSpace{}%
\AgdaSymbol{→}\AgdaSpace{}%
\AgdaPrimitive{Set}\<%
\\
%
\>[6]\AgdaFunction{D}\AgdaSpace{}%
\AgdaBound{x}\AgdaSpace{}%
\AgdaBound{y}\AgdaSpace{}%
\AgdaBound{p}\AgdaSpace{}%
\AgdaSymbol{=}\AgdaSpace{}%
\AgdaBound{p}\AgdaSpace{}%
\AgdaOperator{\AgdaDatatype{≡}}\AgdaSpace{}%
\AgdaInductiveConstructor{r}\AgdaSpace{}%
\AgdaOperator{\AgdaFunction{∙}}\AgdaSpace{}%
\AgdaBound{p}\<%
\\
%
\>[6]\AgdaFunction{d}\AgdaSpace{}%
\AgdaSymbol{:}\AgdaSpace{}%
\AgdaSymbol{(}\AgdaBound{a}\AgdaSpace{}%
\AgdaSymbol{:}\AgdaSpace{}%
\AgdaBound{A}\AgdaSymbol{)}\AgdaSpace{}%
\AgdaSymbol{→}\AgdaSpace{}%
\AgdaFunction{D}\AgdaSpace{}%
\AgdaBound{a}\AgdaSpace{}%
\AgdaBound{a}\AgdaSpace{}%
\AgdaInductiveConstructor{r}\<%
\\
%
\>[6]\AgdaFunction{d}\AgdaSpace{}%
\AgdaBound{a}\AgdaSpace{}%
\AgdaSymbol{=}\AgdaSpace{}%
\AgdaInductiveConstructor{r}\<%
\\
%
\\[\AgdaEmptyExtraSkip]%
%
\>[2]\AgdaFunction{iᵣ}\AgdaSpace{}%
\AgdaSymbol{:}\AgdaSpace{}%
\AgdaSymbol{\{}\AgdaBound{A}\AgdaSpace{}%
\AgdaSymbol{:}\AgdaSpace{}%
\AgdaPrimitive{Set}\AgdaSymbol{\}}\AgdaSpace{}%
\AgdaSymbol{\{}\AgdaBound{x}\AgdaSpace{}%
\AgdaBound{y}\AgdaSpace{}%
\AgdaSymbol{:}\AgdaSpace{}%
\AgdaBound{A}\AgdaSymbol{\}}\AgdaSpace{}%
\AgdaSymbol{(}\AgdaBound{p}\AgdaSpace{}%
\AgdaSymbol{:}\AgdaSpace{}%
\AgdaBound{x}\AgdaSpace{}%
\AgdaOperator{\AgdaDatatype{≡}}\AgdaSpace{}%
\AgdaBound{y}\AgdaSymbol{)}\AgdaSpace{}%
\AgdaSymbol{→}\AgdaSpace{}%
\AgdaBound{p}\AgdaSpace{}%
\AgdaOperator{\AgdaDatatype{≡}}\AgdaSpace{}%
\AgdaBound{p}\AgdaSpace{}%
\AgdaOperator{\AgdaFunction{∙}}\AgdaSpace{}%
\AgdaInductiveConstructor{r}\<%
\\
%
\>[2]\AgdaFunction{iᵣ}\AgdaSpace{}%
\AgdaSymbol{\{}\AgdaBound{A}\AgdaSymbol{\}}\AgdaSpace{}%
\AgdaSymbol{\{}\AgdaBound{x}\AgdaSymbol{\}}\AgdaSpace{}%
\AgdaSymbol{\{}\AgdaBound{y}\AgdaSymbol{\}}\AgdaSpace{}%
\AgdaBound{p}\AgdaSpace{}%
\AgdaSymbol{=}\AgdaSpace{}%
\AgdaFunction{J}\AgdaSpace{}%
\AgdaFunction{D}\AgdaSpace{}%
\AgdaFunction{d}\AgdaSpace{}%
\AgdaBound{x}\AgdaSpace{}%
\AgdaBound{y}\AgdaSpace{}%
\AgdaBound{p}\<%
\\
\>[2][@{}l@{\AgdaIndent{0}}]%
\>[4]\AgdaKeyword{where}\<%
\\
\>[4][@{}l@{\AgdaIndent{0}}]%
\>[6]\AgdaFunction{D}\AgdaSpace{}%
\AgdaSymbol{:}\AgdaSpace{}%
\AgdaSymbol{(}\AgdaBound{x}\AgdaSpace{}%
\AgdaBound{y}\AgdaSpace{}%
\AgdaSymbol{:}\AgdaSpace{}%
\AgdaBound{A}\AgdaSymbol{)}\AgdaSpace{}%
\AgdaSymbol{→}\AgdaSpace{}%
\AgdaBound{x}\AgdaSpace{}%
\AgdaOperator{\AgdaDatatype{≡}}\AgdaSpace{}%
\AgdaBound{y}\AgdaSpace{}%
\AgdaSymbol{→}\AgdaSpace{}%
\AgdaPrimitive{Set}\<%
\\
%
\>[6]\AgdaFunction{D}\AgdaSpace{}%
\AgdaBound{x}\AgdaSpace{}%
\AgdaBound{y}\AgdaSpace{}%
\AgdaBound{p}\AgdaSpace{}%
\AgdaSymbol{=}\AgdaSpace{}%
\AgdaBound{p}\AgdaSpace{}%
\AgdaOperator{\AgdaDatatype{≡}}\AgdaSpace{}%
\AgdaBound{p}\AgdaSpace{}%
\AgdaOperator{\AgdaFunction{∙}}\AgdaSpace{}%
\AgdaInductiveConstructor{r}\<%
\\
%
\>[6]\AgdaFunction{d}\AgdaSpace{}%
\AgdaSymbol{:}\AgdaSpace{}%
\AgdaSymbol{(}\AgdaBound{a}\AgdaSpace{}%
\AgdaSymbol{:}\AgdaSpace{}%
\AgdaBound{A}\AgdaSymbol{)}\AgdaSpace{}%
\AgdaSymbol{→}\AgdaSpace{}%
\AgdaFunction{D}\AgdaSpace{}%
\AgdaBound{a}\AgdaSpace{}%
\AgdaBound{a}\AgdaSpace{}%
\AgdaInductiveConstructor{r}\<%
\\
%
\>[6]\AgdaFunction{d}\AgdaSpace{}%
\AgdaBound{a}\AgdaSpace{}%
\AgdaSymbol{=}\AgdaSpace{}%
\AgdaInductiveConstructor{r}\<%
\\
%
\\[\AgdaEmptyExtraSkip]%
%
\>[2]\AgdaFunction{leftInverse}\AgdaSpace{}%
\AgdaSymbol{:}\AgdaSpace{}%
\AgdaSymbol{\{}\AgdaBound{A}\AgdaSpace{}%
\AgdaSymbol{:}\AgdaSpace{}%
\AgdaPrimitive{Set}\AgdaSymbol{\}}\AgdaSpace{}%
\AgdaSymbol{\{}\AgdaBound{x}\AgdaSpace{}%
\AgdaBound{y}\AgdaSpace{}%
\AgdaSymbol{:}\AgdaSpace{}%
\AgdaBound{A}\AgdaSymbol{\}}\AgdaSpace{}%
\AgdaSymbol{(}\AgdaBound{p}\AgdaSpace{}%
\AgdaSymbol{:}\AgdaSpace{}%
\AgdaBound{x}\AgdaSpace{}%
\AgdaOperator{\AgdaDatatype{≡}}\AgdaSpace{}%
\AgdaBound{y}\AgdaSymbol{)}\AgdaSpace{}%
\AgdaSymbol{→}\AgdaSpace{}%
\AgdaBound{p}\AgdaSpace{}%
\AgdaOperator{\AgdaFunction{⁻¹}}\AgdaSpace{}%
\AgdaOperator{\AgdaFunction{∙}}\AgdaSpace{}%
\AgdaBound{p}\AgdaSpace{}%
\AgdaOperator{\AgdaDatatype{≡}}\AgdaSpace{}%
\AgdaInductiveConstructor{r}\<%
\\
%
\>[2]\AgdaFunction{leftInverse}\AgdaSpace{}%
\AgdaSymbol{\{}\AgdaBound{A}\AgdaSymbol{\}}\AgdaSpace{}%
\AgdaSymbol{\{}\AgdaBound{x}\AgdaSymbol{\}}\AgdaSpace{}%
\AgdaSymbol{\{}\AgdaBound{y}\AgdaSymbol{\}}\AgdaSpace{}%
\AgdaBound{p}\AgdaSpace{}%
\AgdaSymbol{=}\AgdaSpace{}%
\AgdaFunction{J}\AgdaSpace{}%
\AgdaFunction{D}\AgdaSpace{}%
\AgdaFunction{d}\AgdaSpace{}%
\AgdaBound{x}\AgdaSpace{}%
\AgdaBound{y}\AgdaSpace{}%
\AgdaBound{p}\<%
\\
\>[2][@{}l@{\AgdaIndent{0}}]%
\>[4]\AgdaKeyword{where}\<%
\\
\>[4][@{}l@{\AgdaIndent{0}}]%
\>[6]\AgdaFunction{D}\AgdaSpace{}%
\AgdaSymbol{:}\AgdaSpace{}%
\AgdaSymbol{(}\AgdaBound{x}\AgdaSpace{}%
\AgdaBound{y}\AgdaSpace{}%
\AgdaSymbol{:}\AgdaSpace{}%
\AgdaBound{A}\AgdaSymbol{)}\AgdaSpace{}%
\AgdaSymbol{→}\AgdaSpace{}%
\AgdaBound{x}\AgdaSpace{}%
\AgdaOperator{\AgdaDatatype{≡}}\AgdaSpace{}%
\AgdaBound{y}\AgdaSpace{}%
\AgdaSymbol{→}\AgdaSpace{}%
\AgdaPrimitive{Set}\<%
\\
%
\>[6]\AgdaFunction{D}\AgdaSpace{}%
\AgdaBound{x}\AgdaSpace{}%
\AgdaBound{y}\AgdaSpace{}%
\AgdaBound{p}\AgdaSpace{}%
\AgdaSymbol{=}\AgdaSpace{}%
\AgdaBound{p}\AgdaSpace{}%
\AgdaOperator{\AgdaFunction{⁻¹}}\AgdaSpace{}%
\AgdaOperator{\AgdaFunction{∙}}\AgdaSpace{}%
\AgdaBound{p}\AgdaSpace{}%
\AgdaOperator{\AgdaDatatype{≡}}\AgdaSpace{}%
\AgdaInductiveConstructor{r}\<%
\\
%
\>[6]\AgdaFunction{d}\AgdaSpace{}%
\AgdaSymbol{:}\AgdaSpace{}%
\AgdaSymbol{(}\AgdaBound{x}\AgdaSpace{}%
\AgdaSymbol{:}\AgdaSpace{}%
\AgdaBound{A}\AgdaSymbol{)}\AgdaSpace{}%
\AgdaSymbol{→}\AgdaSpace{}%
\AgdaFunction{D}\AgdaSpace{}%
\AgdaBound{x}\AgdaSpace{}%
\AgdaBound{x}\AgdaSpace{}%
\AgdaInductiveConstructor{r}\<%
\\
%
\>[6]\AgdaFunction{d}\AgdaSpace{}%
\AgdaBound{x}\AgdaSpace{}%
\AgdaSymbol{=}\AgdaSpace{}%
\AgdaInductiveConstructor{r}\<%
\\
%
\\[\AgdaEmptyExtraSkip]%
%
\>[2]\AgdaFunction{rightInverse}\AgdaSpace{}%
\AgdaSymbol{:}\AgdaSpace{}%
\AgdaSymbol{\{}\AgdaBound{A}\AgdaSpace{}%
\AgdaSymbol{:}\AgdaSpace{}%
\AgdaPrimitive{Set}\AgdaSymbol{\}}\AgdaSpace{}%
\AgdaSymbol{\{}\AgdaBound{x}\AgdaSpace{}%
\AgdaBound{y}\AgdaSpace{}%
\AgdaSymbol{:}\AgdaSpace{}%
\AgdaBound{A}\AgdaSymbol{\}}\AgdaSpace{}%
\AgdaSymbol{(}\AgdaBound{p}\AgdaSpace{}%
\AgdaSymbol{:}\AgdaSpace{}%
\AgdaBound{x}\AgdaSpace{}%
\AgdaOperator{\AgdaDatatype{≡}}\AgdaSpace{}%
\AgdaBound{y}\AgdaSymbol{)}\AgdaSpace{}%
\AgdaSymbol{→}\AgdaSpace{}%
\AgdaBound{p}\AgdaSpace{}%
\AgdaOperator{\AgdaFunction{∙}}\AgdaSpace{}%
\AgdaBound{p}\AgdaSpace{}%
\AgdaOperator{\AgdaFunction{⁻¹}}\AgdaSpace{}%
\AgdaOperator{\AgdaDatatype{≡}}\AgdaSpace{}%
\AgdaInductiveConstructor{r}\<%
\\
%
\>[2]\AgdaFunction{rightInverse}\AgdaSpace{}%
\AgdaSymbol{\{}\AgdaBound{A}\AgdaSymbol{\}}\AgdaSpace{}%
\AgdaSymbol{\{}\AgdaBound{x}\AgdaSymbol{\}}\AgdaSpace{}%
\AgdaSymbol{\{}\AgdaBound{y}\AgdaSymbol{\}}\AgdaSpace{}%
\AgdaBound{p}\AgdaSpace{}%
\AgdaSymbol{=}\AgdaSpace{}%
\AgdaFunction{J}\AgdaSpace{}%
\AgdaFunction{D}\AgdaSpace{}%
\AgdaFunction{d}\AgdaSpace{}%
\AgdaBound{x}\AgdaSpace{}%
\AgdaBound{y}\AgdaSpace{}%
\AgdaBound{p}\<%
\\
\>[2][@{}l@{\AgdaIndent{0}}]%
\>[4]\AgdaKeyword{where}\<%
\\
\>[4][@{}l@{\AgdaIndent{0}}]%
\>[6]\AgdaFunction{D}\AgdaSpace{}%
\AgdaSymbol{:}\AgdaSpace{}%
\AgdaSymbol{(}\AgdaBound{x}\AgdaSpace{}%
\AgdaBound{y}\AgdaSpace{}%
\AgdaSymbol{:}\AgdaSpace{}%
\AgdaBound{A}\AgdaSymbol{)}\AgdaSpace{}%
\AgdaSymbol{→}\AgdaSpace{}%
\AgdaBound{x}\AgdaSpace{}%
\AgdaOperator{\AgdaDatatype{≡}}\AgdaSpace{}%
\AgdaBound{y}\AgdaSpace{}%
\AgdaSymbol{→}\AgdaSpace{}%
\AgdaPrimitive{Set}\<%
\\
%
\>[6]\AgdaFunction{D}\AgdaSpace{}%
\AgdaBound{x}\AgdaSpace{}%
\AgdaBound{y}\AgdaSpace{}%
\AgdaBound{p}\AgdaSpace{}%
\AgdaSymbol{=}\AgdaSpace{}%
\AgdaBound{p}\AgdaSpace{}%
\AgdaOperator{\AgdaFunction{∙}}\AgdaSpace{}%
\AgdaBound{p}\AgdaSpace{}%
\AgdaOperator{\AgdaFunction{⁻¹}}\AgdaSpace{}%
\AgdaOperator{\AgdaDatatype{≡}}\AgdaSpace{}%
\AgdaInductiveConstructor{r}\<%
\\
%
\>[6]\AgdaFunction{d}\AgdaSpace{}%
\AgdaSymbol{:}\AgdaSpace{}%
\AgdaSymbol{(}\AgdaBound{a}\AgdaSpace{}%
\AgdaSymbol{:}\AgdaSpace{}%
\AgdaBound{A}\AgdaSymbol{)}\AgdaSpace{}%
\AgdaSymbol{→}\AgdaSpace{}%
\AgdaFunction{D}\AgdaSpace{}%
\AgdaBound{a}\AgdaSpace{}%
\AgdaBound{a}\AgdaSpace{}%
\AgdaInductiveConstructor{r}\<%
\\
%
\>[6]\AgdaFunction{d}\AgdaSpace{}%
\AgdaBound{a}\AgdaSpace{}%
\AgdaSymbol{=}\AgdaSpace{}%
\AgdaInductiveConstructor{r}\<%
\\
%
\\[\AgdaEmptyExtraSkip]%
%
\>[2]\AgdaFunction{doubleInv}\AgdaSpace{}%
\AgdaSymbol{:}\AgdaSpace{}%
\AgdaSymbol{\{}\AgdaBound{A}\AgdaSpace{}%
\AgdaSymbol{:}\AgdaSpace{}%
\AgdaPrimitive{Set}\AgdaSymbol{\}}\AgdaSpace{}%
\AgdaSymbol{\{}\AgdaBound{x}\AgdaSpace{}%
\AgdaBound{y}\AgdaSpace{}%
\AgdaSymbol{:}\AgdaSpace{}%
\AgdaBound{A}\AgdaSymbol{\}}\AgdaSpace{}%
\AgdaSymbol{(}\AgdaBound{p}\AgdaSpace{}%
\AgdaSymbol{:}\AgdaSpace{}%
\AgdaBound{x}\AgdaSpace{}%
\AgdaOperator{\AgdaDatatype{≡}}\AgdaSpace{}%
\AgdaBound{y}\AgdaSymbol{)}\AgdaSpace{}%
\AgdaSymbol{→}\AgdaSpace{}%
\AgdaBound{p}\AgdaSpace{}%
\AgdaOperator{\AgdaFunction{⁻¹}}\AgdaSpace{}%
\AgdaOperator{\AgdaFunction{⁻¹}}\AgdaSpace{}%
\AgdaOperator{\AgdaDatatype{≡}}\AgdaSpace{}%
\AgdaBound{p}\<%
\\
%
\>[2]\AgdaFunction{doubleInv}\AgdaSpace{}%
\AgdaSymbol{\{}\AgdaBound{A}\AgdaSymbol{\}}\AgdaSpace{}%
\AgdaSymbol{\{}\AgdaBound{x}\AgdaSymbol{\}}\AgdaSpace{}%
\AgdaSymbol{\{}\AgdaBound{y}\AgdaSymbol{\}}\AgdaSpace{}%
\AgdaBound{p}\AgdaSpace{}%
\AgdaSymbol{=}\AgdaSpace{}%
\AgdaFunction{J}\AgdaSpace{}%
\AgdaFunction{D}\AgdaSpace{}%
\AgdaFunction{d}\AgdaSpace{}%
\AgdaBound{x}\AgdaSpace{}%
\AgdaBound{y}\AgdaSpace{}%
\AgdaBound{p}\<%
\\
\>[2][@{}l@{\AgdaIndent{0}}]%
\>[4]\AgdaKeyword{where}\<%
\\
\>[4][@{}l@{\AgdaIndent{0}}]%
\>[6]\AgdaFunction{D}\AgdaSpace{}%
\AgdaSymbol{:}\AgdaSpace{}%
\AgdaSymbol{(}\AgdaBound{x}\AgdaSpace{}%
\AgdaBound{y}\AgdaSpace{}%
\AgdaSymbol{:}\AgdaSpace{}%
\AgdaBound{A}\AgdaSymbol{)}\AgdaSpace{}%
\AgdaSymbol{→}\AgdaSpace{}%
\AgdaBound{x}\AgdaSpace{}%
\AgdaOperator{\AgdaDatatype{≡}}\AgdaSpace{}%
\AgdaBound{y}\AgdaSpace{}%
\AgdaSymbol{→}\AgdaSpace{}%
\AgdaPrimitive{Set}\<%
\\
%
\>[6]\AgdaFunction{D}\AgdaSpace{}%
\AgdaBound{x}\AgdaSpace{}%
\AgdaBound{y}\AgdaSpace{}%
\AgdaBound{p}\AgdaSpace{}%
\AgdaSymbol{=}\AgdaSpace{}%
\AgdaBound{p}\AgdaSpace{}%
\AgdaOperator{\AgdaFunction{⁻¹}}\AgdaSpace{}%
\AgdaOperator{\AgdaFunction{⁻¹}}\AgdaSpace{}%
\AgdaOperator{\AgdaDatatype{≡}}\AgdaSpace{}%
\AgdaBound{p}\<%
\\
%
\>[6]\AgdaFunction{d}\AgdaSpace{}%
\AgdaSymbol{:}\AgdaSpace{}%
\AgdaSymbol{(}\AgdaBound{a}\AgdaSpace{}%
\AgdaSymbol{:}\AgdaSpace{}%
\AgdaBound{A}\AgdaSymbol{)}\AgdaSpace{}%
\AgdaSymbol{→}\AgdaSpace{}%
\AgdaFunction{D}\AgdaSpace{}%
\AgdaBound{a}\AgdaSpace{}%
\AgdaBound{a}\AgdaSpace{}%
\AgdaInductiveConstructor{r}\<%
\\
%
\>[6]\AgdaFunction{d}\AgdaSpace{}%
\AgdaBound{a}\AgdaSpace{}%
\AgdaSymbol{=}\AgdaSpace{}%
\AgdaInductiveConstructor{r}\<%
\\
%
\\[\AgdaEmptyExtraSkip]%
%
\>[2]\AgdaFunction{associativity}\AgdaSpace{}%
\AgdaSymbol{:\{}\AgdaBound{A}\AgdaSpace{}%
\AgdaSymbol{:}\AgdaSpace{}%
\AgdaPrimitive{Set}\AgdaSymbol{\}}\AgdaSpace{}%
\AgdaSymbol{\{}\AgdaBound{x}\AgdaSpace{}%
\AgdaBound{y}\AgdaSpace{}%
\AgdaBound{z}\AgdaSpace{}%
\AgdaBound{w}\AgdaSpace{}%
\AgdaSymbol{:}\AgdaSpace{}%
\AgdaBound{A}\AgdaSymbol{\}}\AgdaSpace{}%
\AgdaSymbol{(}\AgdaBound{p}\AgdaSpace{}%
\AgdaSymbol{:}\AgdaSpace{}%
\AgdaBound{x}\AgdaSpace{}%
\AgdaOperator{\AgdaDatatype{≡}}\AgdaSpace{}%
\AgdaBound{y}\AgdaSymbol{)}\AgdaSpace{}%
\AgdaSymbol{(}\AgdaBound{q}\AgdaSpace{}%
\AgdaSymbol{:}\AgdaSpace{}%
\AgdaBound{y}\AgdaSpace{}%
\AgdaOperator{\AgdaDatatype{≡}}\AgdaSpace{}%
\AgdaBound{z}\AgdaSymbol{)}\AgdaSpace{}%
\AgdaSymbol{(}\AgdaBound{r'}\AgdaSpace{}%
\AgdaSymbol{:}\AgdaSpace{}%
\AgdaBound{z}\AgdaSpace{}%
\AgdaOperator{\AgdaDatatype{≡}}\AgdaSpace{}%
\AgdaBound{w}\AgdaSpace{}%
\AgdaSymbol{)}\AgdaSpace{}%
\AgdaSymbol{→}\AgdaSpace{}%
\AgdaBound{p}\AgdaSpace{}%
\AgdaOperator{\AgdaFunction{∙}}\AgdaSpace{}%
\AgdaSymbol{(}\AgdaBound{q}\AgdaSpace{}%
\AgdaOperator{\AgdaFunction{∙}}\AgdaSpace{}%
\AgdaBound{r'}\AgdaSymbol{)}\AgdaSpace{}%
\AgdaOperator{\AgdaDatatype{≡}}\AgdaSpace{}%
\AgdaBound{p}\AgdaSpace{}%
\AgdaOperator{\AgdaFunction{∙}}\AgdaSpace{}%
\AgdaBound{q}\AgdaSpace{}%
\AgdaOperator{\AgdaFunction{∙}}\AgdaSpace{}%
\AgdaBound{r'}\<%
\\
%
\>[2]\AgdaFunction{associativity}\AgdaSpace{}%
\AgdaSymbol{\{}\AgdaBound{A}\AgdaSymbol{\}}\AgdaSpace{}%
\AgdaSymbol{\{}\AgdaBound{x}\AgdaSymbol{\}}\AgdaSpace{}%
\AgdaSymbol{\{}\AgdaBound{y}\AgdaSymbol{\}}\AgdaSpace{}%
\AgdaSymbol{\{}\AgdaBound{z}\AgdaSymbol{\}}\AgdaSpace{}%
\AgdaSymbol{\{}\AgdaBound{w}\AgdaSymbol{\}}\AgdaSpace{}%
\AgdaBound{p}\AgdaSpace{}%
\AgdaBound{q}\AgdaSpace{}%
\AgdaBound{r'}\AgdaSpace{}%
\AgdaSymbol{=}\AgdaSpace{}%
\AgdaFunction{J}\AgdaSpace{}%
\AgdaFunction{D₁}\AgdaSpace{}%
\AgdaFunction{d₁}\AgdaSpace{}%
\AgdaBound{x}\AgdaSpace{}%
\AgdaBound{y}\AgdaSpace{}%
\AgdaBound{p}\AgdaSpace{}%
\AgdaBound{z}\AgdaSpace{}%
\AgdaBound{w}\AgdaSpace{}%
\AgdaBound{q}\AgdaSpace{}%
\AgdaBound{r'}\<%
\\
\>[2][@{}l@{\AgdaIndent{0}}]%
\>[4]\AgdaKeyword{where}\<%
\\
\>[4][@{}l@{\AgdaIndent{0}}]%
\>[6]\AgdaFunction{D₁}\AgdaSpace{}%
\AgdaSymbol{:}\AgdaSpace{}%
\AgdaSymbol{(}\AgdaBound{x}\AgdaSpace{}%
\AgdaBound{y}\AgdaSpace{}%
\AgdaSymbol{:}\AgdaSpace{}%
\AgdaBound{A}\AgdaSymbol{)}\AgdaSpace{}%
\AgdaSymbol{→}\AgdaSpace{}%
\AgdaBound{x}\AgdaSpace{}%
\AgdaOperator{\AgdaDatatype{≡}}\AgdaSpace{}%
\AgdaBound{y}\AgdaSpace{}%
\AgdaSymbol{→}\AgdaSpace{}%
\AgdaPrimitive{Set}\<%
\\
%
\>[6]\AgdaFunction{D₁}\AgdaSpace{}%
\AgdaBound{x}\AgdaSpace{}%
\AgdaBound{y}\AgdaSpace{}%
\AgdaBound{p}\AgdaSpace{}%
\AgdaSymbol{=}\AgdaSpace{}%
\AgdaSymbol{(}\AgdaBound{z}\AgdaSpace{}%
\AgdaBound{w}\AgdaSpace{}%
\AgdaSymbol{:}\AgdaSpace{}%
\AgdaBound{A}\AgdaSymbol{)}\AgdaSpace{}%
\AgdaSymbol{(}\AgdaBound{q}\AgdaSpace{}%
\AgdaSymbol{:}\AgdaSpace{}%
\AgdaBound{y}\AgdaSpace{}%
\AgdaOperator{\AgdaDatatype{≡}}\AgdaSpace{}%
\AgdaBound{z}\AgdaSymbol{)}\AgdaSpace{}%
\AgdaSymbol{(}\AgdaBound{r'}\AgdaSpace{}%
\AgdaSymbol{:}\AgdaSpace{}%
\AgdaBound{z}\AgdaSpace{}%
\AgdaOperator{\AgdaDatatype{≡}}\AgdaSpace{}%
\AgdaBound{w}\AgdaSpace{}%
\AgdaSymbol{)}\AgdaSpace{}%
\AgdaSymbol{→}\AgdaSpace{}%
\AgdaBound{p}\AgdaSpace{}%
\AgdaOperator{\AgdaFunction{∙}}\AgdaSpace{}%
\AgdaSymbol{(}\AgdaBound{q}\AgdaSpace{}%
\AgdaOperator{\AgdaFunction{∙}}\AgdaSpace{}%
\AgdaBound{r'}\AgdaSymbol{)}\AgdaSpace{}%
\AgdaOperator{\AgdaDatatype{≡}}\AgdaSpace{}%
\AgdaBound{p}\AgdaSpace{}%
\AgdaOperator{\AgdaFunction{∙}}\AgdaSpace{}%
\AgdaBound{q}\AgdaSpace{}%
\AgdaOperator{\AgdaFunction{∙}}\AgdaSpace{}%
\AgdaBound{r'}\<%
\\
%
\>[6]\AgdaComment{-- d₁ : (x : A) → D₁ x x r }\<%
\\
%
\>[6]\AgdaComment{-- d₁ x z w q r' = r -- why can it infer this }\<%
\\
%
\>[6]\AgdaFunction{D₂}\AgdaSpace{}%
\AgdaSymbol{:}\AgdaSpace{}%
\AgdaSymbol{(}\AgdaBound{x}\AgdaSpace{}%
\AgdaBound{z}\AgdaSpace{}%
\AgdaSymbol{:}\AgdaSpace{}%
\AgdaBound{A}\AgdaSymbol{)}\AgdaSpace{}%
\AgdaSymbol{→}\AgdaSpace{}%
\AgdaBound{x}\AgdaSpace{}%
\AgdaOperator{\AgdaDatatype{≡}}\AgdaSpace{}%
\AgdaBound{z}\AgdaSpace{}%
\AgdaSymbol{→}\AgdaSpace{}%
\AgdaPrimitive{Set}\<%
\\
%
\>[6]\AgdaFunction{D₂}\AgdaSpace{}%
\AgdaBound{x}\AgdaSpace{}%
\AgdaBound{z}\AgdaSpace{}%
\AgdaBound{q}\AgdaSpace{}%
\AgdaSymbol{=}\AgdaSpace{}%
\AgdaSymbol{(}\AgdaBound{w}\AgdaSpace{}%
\AgdaSymbol{:}\AgdaSpace{}%
\AgdaBound{A}\AgdaSymbol{)}\AgdaSpace{}%
\AgdaSymbol{(}\AgdaBound{r'}\AgdaSpace{}%
\AgdaSymbol{:}\AgdaSpace{}%
\AgdaBound{z}\AgdaSpace{}%
\AgdaOperator{\AgdaDatatype{≡}}\AgdaSpace{}%
\AgdaBound{w}\AgdaSpace{}%
\AgdaSymbol{)}\AgdaSpace{}%
\AgdaSymbol{→}\AgdaSpace{}%
\AgdaInductiveConstructor{r}\AgdaSpace{}%
\AgdaOperator{\AgdaFunction{∙}}\AgdaSpace{}%
\AgdaSymbol{(}\AgdaBound{q}\AgdaSpace{}%
\AgdaOperator{\AgdaFunction{∙}}\AgdaSpace{}%
\AgdaBound{r'}\AgdaSymbol{)}\AgdaSpace{}%
\AgdaOperator{\AgdaDatatype{≡}}\AgdaSpace{}%
\AgdaInductiveConstructor{r}\AgdaSpace{}%
\AgdaOperator{\AgdaFunction{∙}}\AgdaSpace{}%
\AgdaBound{q}\AgdaSpace{}%
\AgdaOperator{\AgdaFunction{∙}}\AgdaSpace{}%
\AgdaBound{r'}\<%
\\
%
\>[6]\AgdaFunction{D₃}\AgdaSpace{}%
\AgdaSymbol{:}\AgdaSpace{}%
\AgdaSymbol{(}\AgdaBound{x}\AgdaSpace{}%
\AgdaBound{w}\AgdaSpace{}%
\AgdaSymbol{:}\AgdaSpace{}%
\AgdaBound{A}\AgdaSymbol{)}\AgdaSpace{}%
\AgdaSymbol{→}\AgdaSpace{}%
\AgdaBound{x}\AgdaSpace{}%
\AgdaOperator{\AgdaDatatype{≡}}\AgdaSpace{}%
\AgdaBound{w}\AgdaSpace{}%
\AgdaSymbol{→}\AgdaSpace{}%
\AgdaPrimitive{Set}\<%
\\
%
\>[6]\AgdaFunction{D₃}\AgdaSpace{}%
\AgdaBound{x}\AgdaSpace{}%
\AgdaBound{w}\AgdaSpace{}%
\AgdaBound{r'}\AgdaSpace{}%
\AgdaSymbol{=}%
\>[19]\AgdaInductiveConstructor{r}\AgdaSpace{}%
\AgdaOperator{\AgdaFunction{∙}}\AgdaSpace{}%
\AgdaSymbol{(}\AgdaInductiveConstructor{r}\AgdaSpace{}%
\AgdaOperator{\AgdaFunction{∙}}\AgdaSpace{}%
\AgdaBound{r'}\AgdaSymbol{)}\AgdaSpace{}%
\AgdaOperator{\AgdaDatatype{≡}}\AgdaSpace{}%
\AgdaInductiveConstructor{r}\AgdaSpace{}%
\AgdaOperator{\AgdaFunction{∙}}\AgdaSpace{}%
\AgdaInductiveConstructor{r}\AgdaSpace{}%
\AgdaOperator{\AgdaFunction{∙}}\AgdaSpace{}%
\AgdaBound{r'}\<%
\\
%
\>[6]\AgdaFunction{d₃}\AgdaSpace{}%
\AgdaSymbol{:}\AgdaSpace{}%
\AgdaSymbol{(}\AgdaBound{x}\AgdaSpace{}%
\AgdaSymbol{:}\AgdaSpace{}%
\AgdaBound{A}\AgdaSymbol{)}\AgdaSpace{}%
\AgdaSymbol{→}\AgdaSpace{}%
\AgdaFunction{D₃}\AgdaSpace{}%
\AgdaBound{x}\AgdaSpace{}%
\AgdaBound{x}\AgdaSpace{}%
\AgdaInductiveConstructor{r}\<%
\\
%
\>[6]\AgdaFunction{d₃}\AgdaSpace{}%
\AgdaBound{x}\AgdaSpace{}%
\AgdaSymbol{=}\AgdaSpace{}%
\AgdaInductiveConstructor{r}\<%
\\
%
\>[6]\AgdaFunction{d₂}\AgdaSpace{}%
\AgdaSymbol{:}\AgdaSpace{}%
\AgdaSymbol{(}\AgdaBound{x}\AgdaSpace{}%
\AgdaSymbol{:}\AgdaSpace{}%
\AgdaBound{A}\AgdaSymbol{)}\AgdaSpace{}%
\AgdaSymbol{→}\AgdaSpace{}%
\AgdaFunction{D₂}\AgdaSpace{}%
\AgdaBound{x}\AgdaSpace{}%
\AgdaBound{x}\AgdaSpace{}%
\AgdaInductiveConstructor{r}\<%
\\
%
\>[6]\AgdaFunction{d₂}\AgdaSpace{}%
\AgdaBound{x}\AgdaSpace{}%
\AgdaBound{w}\AgdaSpace{}%
\AgdaBound{r'}\AgdaSpace{}%
\AgdaSymbol{=}\AgdaSpace{}%
\AgdaFunction{J}\AgdaSpace{}%
\AgdaFunction{D₃}\AgdaSpace{}%
\AgdaFunction{d₃}\AgdaSpace{}%
\AgdaBound{x}\AgdaSpace{}%
\AgdaBound{w}\AgdaSpace{}%
\AgdaBound{r'}\<%
\\
%
\>[6]\AgdaFunction{d₁}\AgdaSpace{}%
\AgdaSymbol{:}\AgdaSpace{}%
\AgdaSymbol{(}\AgdaBound{x}\AgdaSpace{}%
\AgdaSymbol{:}\AgdaSpace{}%
\AgdaBound{A}\AgdaSymbol{)}\AgdaSpace{}%
\AgdaSymbol{→}\AgdaSpace{}%
\AgdaFunction{D₁}\AgdaSpace{}%
\AgdaBound{x}\AgdaSpace{}%
\AgdaBound{x}\AgdaSpace{}%
\AgdaInductiveConstructor{r}\<%
\\
%
\>[6]\AgdaFunction{d₁}\AgdaSpace{}%
\AgdaBound{x}\AgdaSpace{}%
\AgdaBound{z}\AgdaSpace{}%
\AgdaBound{w}\AgdaSpace{}%
\AgdaBound{q}\AgdaSpace{}%
\AgdaBound{r'}\AgdaSpace{}%
\AgdaSymbol{=}\AgdaSpace{}%
\AgdaFunction{J}\AgdaSpace{}%
\AgdaFunction{D₂}\AgdaSpace{}%
\AgdaFunction{d₂}\AgdaSpace{}%
\AgdaBound{x}\AgdaSpace{}%
\AgdaBound{z}\AgdaSpace{}%
\AgdaBound{q}\AgdaSpace{}%
\AgdaBound{w}\AgdaSpace{}%
\AgdaBound{r'}\<%
\\
\>[0]\<%
\end{code}

When one starts to look at structure above the groupoid level, i.e., the paths between paths between paths level, some interesting and nonintuitive results emerge. If one defines a path space that is seemingly trivial, namely, taking the same starting and end points, the higherdimensional structure yields non-trivial structure. 
We now arrive at the first ``interesting'' result in this book, the Eckmann-Hilton Arguement. It says that composition on the loop space of a loop space, the second loop space, is commutitive.



\begin{definition}

Thus, given a type $A$ with a point $a:A$, we define its \define{loop space}
\index{loop space}%
$\Omega(A,a)$ to be the type $\id[A]{a}{a}$.
We may sometimes write simply $\Omega A$ if the point $a$ is understood from context.

\end {definition}


\begin{definition}
It can also be useful to consider the loop space\index{loop space!iterated}\index{iterated loop space} \emph{of} the loop space of $A$, which is the space of 2-dimensional loops on the identity loop at $a$.
This is written $\Omega^2(A,a)$ and represented in type theory by the type $\id[({\id[A]{a}{a}})]{\refl{a}}{\refl{a}}$.
\end {definition}

\begin{thm}[Eckmann--Hilton]%\label{thm:EckmannHilton}
  The composition operation on the second loop space
  %
  \begin{equation*}
    \Omega^2(A)\times \Omega^2(A)\to \Omega^2(A)
  \end{equation*}
  is commutative: $\alpha\cdot\beta = \beta\cdot\alpha$, for any $\alpha, \beta:\Omega^2(A)$.
  %\index{Eckmann--Hilton argument}%
\end{thm}

\begin{proof}
First, observe that the composition of $1$-loops $\Omega A\times \Omega A\to \Omega A$ induces an operation
\[
\star : \Omega^2(A)\times \Omega^2(A)\to \Omega^2(A)
\]
as follows: consider elements $a, b, c : A$ and 1- and 2-paths,
%
\begin{align*}
 p &: a = b,       &       r &: b = c \\
 q &: a = b,       &       s &: b = c \\
 \alpha &: p = q,  &   \beta &: r = s
\end{align*}
%
as depicted in the following diagram (with paths drawn as arrows).

[TODO Finish Eckmann Hilton Arguement]
%\[
 %\xymatrix@+5em{
   %{a} \rtwocell<10>^p_q{\alpha}
   %&
   %{b} \rtwocell<10>^r_s{\beta}
   %&
   %{c}
 %}
%\]
%Composing the upper and lower 1-paths, respectively, we get two paths $p\ct r,\ q\ct s : a = c$, and there is then a ``horizontal composition''
%%
%\begin{equation*}
  %\alpha\hct\beta : p\ct r = q\ct s
%\end{equation*}
%%
%between them, defined as follows.
%First, we define $\alpha \rightwhisker r : p\ct r = q\ct r$ by path induction on $r$, so that
%\[ \alpha \rightwhisker \refl{b} \jdeq \opp{\mathsf{ru}_p} \ct \alpha \ct \mathsf{ru}_q \]
%where $\mathsf{ru}_p : p = p \ct \refl{b}$ is the right unit law from \cref{thm:omg}\ref{item:omg1}.
%We could similarly define $\rightwhisker$ by induction on $\alpha$, or on all paths in sight, resulting in different judgmental equalities, but for present purposes the definition by induction on $r$ will make things simpler.
%Similarly, we define $q\leftwhisker \beta : q\ct r = q\ct s$ by induction on $q$, so that
%\[ \refl{b} \leftwhisker \beta \jdeq \opp{\mathsf{lu}_r} \ct \beta \ct \mathsf{lu}_s \]
%where $\mathsf{lu}_r$ denotes the left unit law.
%The operations $\leftwhisker$ and $\rightwhisker$ are called \define{whiskering}\indexdef{whiskering}.
%Next, since $\alpha \rightwhisker r$ and $q\leftwhisker \beta$ are composable 2-paths, we can define the \define{horizontal composition}
%\indexdef{horizontal composition!of paths}%
%\indexdef{composition!of paths!horizontal}%
%by:
%\[
%\alpha\hct\beta\ \defeq\ (\alpha\rightwhisker r) \ct (q\leftwhisker \beta).
%\]
%Now suppose that $a \jdeq  b \jdeq  c$, so that all the 1-paths $p$, $q$, $r$, and $s$ are elements of $\Omega(A,a)$, and assume moreover that $p\jdeq q \jdeq r \jdeq s\jdeq \refl{a}$, so that $\alpha:\refl{a} = \refl{a}$ and $\beta:\refl{a} = \refl{a}$ are composable in both orders.
%In that case, we have
%\begin{align*}
  %\alpha\hct\beta
  %&\jdeq (\alpha\rightwhisker\refl{a}) \ct (\refl{a}\leftwhisker \beta)\\
  %&= \opp{\mathsf{ru}_{\refl{a}}} \ct \alpha \ct \mathsf{ru}_{\refl{a}} \ct \opp{\mathsf{lu}_{\refl a}} \ct \beta \ct \mathsf{lu}_{\refl{a}}\\
  %&\jdeq \opp{\refl{\refl{a}}} \ct \alpha \ct \refl{\refl{a}} \ct \opp{\refl{\refl a}} \ct \beta \ct \refl{\refl{a}}\\
  %&= \alpha \ct \beta.
%\end{align*}
%(Recall that $\mathsf{ru}_{\refl{a}} \jdeq \mathsf{lu}_{\refl{a}} \jdeq \refl{\refl{a}}$, by the computation rule for path induction.)
%On the other hand, we can define another horizontal composition analogously by
%\[
%\alpha\hct'\beta\ \defeq\ (p\leftwhisker \beta)\ct (\alpha\rightwhisker s)
%\]
%and we similarly learn that
%\[
%\alpha\hct'\beta = \beta\ct\alpha.
%\]
%\index{interchange law}%
%But, in general, the two ways of defining horizontal composition agree, $\alpha\hct\beta = \alpha\hct'\beta$, as we can see by induction on $\alpha$ and $\beta$ and then on the two remaining 1-paths, to reduce everything to reflexivity.
%Thus we have
%\[\alpha \ct \beta = \alpha\hct\beta = \alpha\hct'\beta = \beta\ct\alpha.
%\qedhere
%\]
\end{proof}


[Todo, clean up code so that its more tightly correspondent to the book proof]
The corresponding agda code is below :

\begin{code}%
\>[0]\<%
\\
\>[0][@{}l@{\AgdaIndent{1}}]%
\>[2]\AgdaComment{-- whiskering}\<%
\\
%
\>[2]\AgdaOperator{\AgdaFunction{\AgdaUnderscore{}∙ᵣ\AgdaUnderscore{}}}\AgdaSpace{}%
\AgdaSymbol{:}\AgdaSpace{}%
\AgdaSymbol{\{}\AgdaBound{A}\AgdaSpace{}%
\AgdaSymbol{:}\AgdaSpace{}%
\AgdaPrimitive{Set}\AgdaSymbol{\}}\AgdaSpace{}%
\AgdaSymbol{→}\AgdaSpace{}%
\AgdaSymbol{\{}\AgdaBound{b}\AgdaSpace{}%
\AgdaBound{c}\AgdaSpace{}%
\AgdaSymbol{:}\AgdaSpace{}%
\AgdaBound{A}\AgdaSymbol{\}}\AgdaSpace{}%
\AgdaSymbol{\{}\AgdaBound{a}\AgdaSpace{}%
\AgdaSymbol{:}\AgdaSpace{}%
\AgdaBound{A}\AgdaSymbol{\}}\AgdaSpace{}%
\AgdaSymbol{\{}\AgdaBound{p}\AgdaSpace{}%
\AgdaBound{q}\AgdaSpace{}%
\AgdaSymbol{:}\AgdaSpace{}%
\AgdaBound{a}\AgdaSpace{}%
\AgdaOperator{\AgdaDatatype{≡}}\AgdaSpace{}%
\AgdaBound{b}\AgdaSymbol{\}}\AgdaSpace{}%
\AgdaSymbol{(}\AgdaBound{α}\AgdaSpace{}%
\AgdaSymbol{:}\AgdaSpace{}%
\AgdaBound{p}\AgdaSpace{}%
\AgdaOperator{\AgdaDatatype{≡}}\AgdaSpace{}%
\AgdaBound{q}\AgdaSymbol{)}\AgdaSpace{}%
\AgdaSymbol{(}\AgdaBound{r'}\AgdaSpace{}%
\AgdaSymbol{:}\AgdaSpace{}%
\AgdaBound{b}\AgdaSpace{}%
\AgdaOperator{\AgdaDatatype{≡}}\AgdaSpace{}%
\AgdaBound{c}\AgdaSymbol{)}\AgdaSpace{}%
\AgdaSymbol{→}\AgdaSpace{}%
\AgdaBound{p}\AgdaSpace{}%
\AgdaOperator{\AgdaFunction{∙}}\AgdaSpace{}%
\AgdaBound{r'}\AgdaSpace{}%
\AgdaOperator{\AgdaDatatype{≡}}\AgdaSpace{}%
\AgdaBound{q}\AgdaSpace{}%
\AgdaOperator{\AgdaFunction{∙}}\AgdaSpace{}%
\AgdaBound{r'}\<%
\\
%
\>[2]\AgdaOperator{\AgdaFunction{\AgdaUnderscore{}∙ᵣ\AgdaUnderscore{}}}\AgdaSpace{}%
\AgdaSymbol{\{}\AgdaBound{A}\AgdaSymbol{\}}\AgdaSpace{}%
\AgdaSymbol{\{}\AgdaBound{b}\AgdaSymbol{\}}\AgdaSpace{}%
\AgdaSymbol{\{}\AgdaBound{c}\AgdaSymbol{\}}\AgdaSpace{}%
\AgdaSymbol{\{}\AgdaBound{a}\AgdaSymbol{\}}\AgdaSpace{}%
\AgdaSymbol{\{}\AgdaBound{p}\AgdaSymbol{\}}\AgdaSpace{}%
\AgdaSymbol{\{}\AgdaBound{q}\AgdaSymbol{\}}\AgdaSpace{}%
\AgdaBound{α}\AgdaSpace{}%
\AgdaBound{r'}\AgdaSpace{}%
\AgdaSymbol{=}\AgdaSpace{}%
\AgdaFunction{J}\AgdaSpace{}%
\AgdaFunction{D}\AgdaSpace{}%
\AgdaFunction{d}\AgdaSpace{}%
\AgdaBound{b}\AgdaSpace{}%
\AgdaBound{c}\AgdaSpace{}%
\AgdaBound{r'}\AgdaSpace{}%
\AgdaBound{a}\AgdaSpace{}%
\AgdaBound{α}\<%
\\
\>[2][@{}l@{\AgdaIndent{0}}]%
\>[4]\AgdaKeyword{where}\<%
\\
\>[4][@{}l@{\AgdaIndent{0}}]%
\>[6]\AgdaFunction{D}\AgdaSpace{}%
\AgdaSymbol{:}\AgdaSpace{}%
\AgdaSymbol{(}\AgdaBound{b}\AgdaSpace{}%
\AgdaBound{c}\AgdaSpace{}%
\AgdaSymbol{:}\AgdaSpace{}%
\AgdaBound{A}\AgdaSymbol{)}\AgdaSpace{}%
\AgdaSymbol{→}\AgdaSpace{}%
\AgdaBound{b}\AgdaSpace{}%
\AgdaOperator{\AgdaDatatype{≡}}\AgdaSpace{}%
\AgdaBound{c}\AgdaSpace{}%
\AgdaSymbol{→}\AgdaSpace{}%
\AgdaPrimitive{Set}\<%
\\
%
\>[6]\AgdaFunction{D}\AgdaSpace{}%
\AgdaBound{b}\AgdaSpace{}%
\AgdaBound{c}\AgdaSpace{}%
\AgdaBound{r'}\AgdaSpace{}%
\AgdaSymbol{=}\AgdaSpace{}%
\AgdaSymbol{(}\AgdaBound{a}\AgdaSpace{}%
\AgdaSymbol{:}\AgdaSpace{}%
\AgdaBound{A}\AgdaSymbol{)}\AgdaSpace{}%
\AgdaSymbol{\{}\AgdaBound{p}\AgdaSpace{}%
\AgdaBound{q}\AgdaSpace{}%
\AgdaSymbol{:}\AgdaSpace{}%
\AgdaBound{a}\AgdaSpace{}%
\AgdaOperator{\AgdaDatatype{≡}}\AgdaSpace{}%
\AgdaBound{b}\AgdaSymbol{\}}\AgdaSpace{}%
\AgdaSymbol{(}\AgdaBound{α}\AgdaSpace{}%
\AgdaSymbol{:}\AgdaSpace{}%
\AgdaBound{p}\AgdaSpace{}%
\AgdaOperator{\AgdaDatatype{≡}}\AgdaSpace{}%
\AgdaBound{q}\AgdaSymbol{)}\AgdaSpace{}%
\AgdaSymbol{→}\AgdaSpace{}%
\AgdaBound{p}\AgdaSpace{}%
\AgdaOperator{\AgdaFunction{∙}}\AgdaSpace{}%
\AgdaBound{r'}\AgdaSpace{}%
\AgdaOperator{\AgdaDatatype{≡}}\AgdaSpace{}%
\AgdaBound{q}\AgdaSpace{}%
\AgdaOperator{\AgdaFunction{∙}}\AgdaSpace{}%
\AgdaBound{r'}\<%
\\
%
\>[6]\AgdaFunction{d}\AgdaSpace{}%
\AgdaSymbol{:}\AgdaSpace{}%
\AgdaSymbol{(}\AgdaBound{a}\AgdaSpace{}%
\AgdaSymbol{:}\AgdaSpace{}%
\AgdaBound{A}\AgdaSymbol{)}\AgdaSpace{}%
\AgdaSymbol{→}\AgdaSpace{}%
\AgdaFunction{D}\AgdaSpace{}%
\AgdaBound{a}\AgdaSpace{}%
\AgdaBound{a}\AgdaSpace{}%
\AgdaInductiveConstructor{r}\<%
\\
%
\>[6]\AgdaFunction{d}\AgdaSpace{}%
\AgdaBound{a}\AgdaSpace{}%
\AgdaBound{a'}\AgdaSpace{}%
\AgdaSymbol{\{}\AgdaBound{p}\AgdaSymbol{\}}\AgdaSpace{}%
\AgdaSymbol{\{}\AgdaBound{q}\AgdaSymbol{\}}\AgdaSpace{}%
\AgdaBound{α}\AgdaSpace{}%
\AgdaSymbol{=}\AgdaSpace{}%
\AgdaFunction{iᵣ}\AgdaSpace{}%
\AgdaBound{p}\AgdaSpace{}%
\AgdaOperator{\AgdaFunction{⁻¹}}\AgdaSpace{}%
\AgdaOperator{\AgdaFunction{∙}}\AgdaSpace{}%
\AgdaBound{α}\AgdaSpace{}%
\AgdaOperator{\AgdaFunction{∙}}\AgdaSpace{}%
\AgdaFunction{iᵣ}\AgdaSpace{}%
\AgdaBound{q}\<%
\\
%
\\[\AgdaEmptyExtraSkip]%
%
\>[2]\AgdaComment{-- iᵣ == ruₚ}\<%
\\
%
\\[\AgdaEmptyExtraSkip]%
%
\>[2]\AgdaOperator{\AgdaFunction{\AgdaUnderscore{}∙ₗ\AgdaUnderscore{}}}\AgdaSpace{}%
\AgdaSymbol{:}\AgdaSpace{}%
\AgdaSymbol{\{}\AgdaBound{A}\AgdaSpace{}%
\AgdaSymbol{:}\AgdaSpace{}%
\AgdaPrimitive{Set}\AgdaSymbol{\}}\AgdaSpace{}%
\AgdaSymbol{→}\AgdaSpace{}%
\AgdaSymbol{\{}\AgdaBound{a}\AgdaSpace{}%
\AgdaBound{b}\AgdaSpace{}%
\AgdaSymbol{:}\AgdaSpace{}%
\AgdaBound{A}\AgdaSymbol{\}}\AgdaSpace{}%
\AgdaSymbol{(}\AgdaBound{q}\AgdaSpace{}%
\AgdaSymbol{:}\AgdaSpace{}%
\AgdaBound{a}\AgdaSpace{}%
\AgdaOperator{\AgdaDatatype{≡}}\AgdaSpace{}%
\AgdaBound{b}\AgdaSymbol{)}\AgdaSpace{}%
\AgdaSymbol{\{}\AgdaBound{c}\AgdaSpace{}%
\AgdaSymbol{:}\AgdaSpace{}%
\AgdaBound{A}\AgdaSymbol{\}}\AgdaSpace{}%
\AgdaSymbol{\{}\AgdaBound{r'}\AgdaSpace{}%
\AgdaBound{s}\AgdaSpace{}%
\AgdaSymbol{:}\AgdaSpace{}%
\AgdaBound{b}\AgdaSpace{}%
\AgdaOperator{\AgdaDatatype{≡}}\AgdaSpace{}%
\AgdaBound{c}\AgdaSymbol{\}}\AgdaSpace{}%
\AgdaSymbol{(}\AgdaBound{β}\AgdaSpace{}%
\AgdaSymbol{:}\AgdaSpace{}%
\AgdaBound{r'}\AgdaSpace{}%
\AgdaOperator{\AgdaDatatype{≡}}\AgdaSpace{}%
\AgdaBound{s}\AgdaSymbol{)}\AgdaSpace{}%
\AgdaSymbol{→}\AgdaSpace{}%
\AgdaBound{q}\AgdaSpace{}%
\AgdaOperator{\AgdaFunction{∙}}\AgdaSpace{}%
\AgdaBound{r'}\AgdaSpace{}%
\AgdaOperator{\AgdaDatatype{≡}}\AgdaSpace{}%
\AgdaBound{q}\AgdaSpace{}%
\AgdaOperator{\AgdaFunction{∙}}\AgdaSpace{}%
\AgdaBound{s}\<%
\\
%
\>[2]\AgdaOperator{\AgdaFunction{\AgdaUnderscore{}∙ₗ\AgdaUnderscore{}}}\AgdaSpace{}%
\AgdaSymbol{\{}\AgdaBound{A}\AgdaSymbol{\}}\AgdaSpace{}%
\AgdaSymbol{\{}\AgdaBound{a}\AgdaSymbol{\}}\AgdaSpace{}%
\AgdaSymbol{\{}\AgdaBound{b}\AgdaSymbol{\}}\AgdaSpace{}%
\AgdaBound{q}\AgdaSpace{}%
\AgdaSymbol{\{}\AgdaBound{c}\AgdaSymbol{\}}\AgdaSpace{}%
\AgdaSymbol{\{}\AgdaBound{r'}\AgdaSymbol{\}}\AgdaSpace{}%
\AgdaSymbol{\{}\AgdaBound{s}\AgdaSymbol{\}}\AgdaSpace{}%
\AgdaBound{β}\AgdaSpace{}%
\AgdaSymbol{=}\AgdaSpace{}%
\AgdaFunction{J}\AgdaSpace{}%
\AgdaFunction{D}\AgdaSpace{}%
\AgdaFunction{d}\AgdaSpace{}%
\AgdaBound{a}\AgdaSpace{}%
\AgdaBound{b}\AgdaSpace{}%
\AgdaBound{q}\AgdaSpace{}%
\AgdaBound{c}\AgdaSpace{}%
\AgdaBound{β}\<%
\\
\>[2][@{}l@{\AgdaIndent{0}}]%
\>[4]\AgdaKeyword{where}\<%
\\
\>[4][@{}l@{\AgdaIndent{0}}]%
\>[6]\AgdaFunction{D}\AgdaSpace{}%
\AgdaSymbol{:}\AgdaSpace{}%
\AgdaSymbol{(}\AgdaBound{a}\AgdaSpace{}%
\AgdaBound{b}\AgdaSpace{}%
\AgdaSymbol{:}\AgdaSpace{}%
\AgdaBound{A}\AgdaSymbol{)}\AgdaSpace{}%
\AgdaSymbol{→}\AgdaSpace{}%
\AgdaBound{a}\AgdaSpace{}%
\AgdaOperator{\AgdaDatatype{≡}}\AgdaSpace{}%
\AgdaBound{b}\AgdaSpace{}%
\AgdaSymbol{→}\AgdaSpace{}%
\AgdaPrimitive{Set}\<%
\\
%
\>[6]\AgdaFunction{D}\AgdaSpace{}%
\AgdaBound{a}\AgdaSpace{}%
\AgdaBound{b}\AgdaSpace{}%
\AgdaBound{q}\AgdaSpace{}%
\AgdaSymbol{=}\AgdaSpace{}%
\AgdaSymbol{(}\AgdaBound{c}\AgdaSpace{}%
\AgdaSymbol{:}\AgdaSpace{}%
\AgdaBound{A}\AgdaSymbol{)}\AgdaSpace{}%
\AgdaSymbol{\{}\AgdaBound{r'}\AgdaSpace{}%
\AgdaBound{s}\AgdaSpace{}%
\AgdaSymbol{:}\AgdaSpace{}%
\AgdaBound{b}\AgdaSpace{}%
\AgdaOperator{\AgdaDatatype{≡}}\AgdaSpace{}%
\AgdaBound{c}\AgdaSymbol{\}}\AgdaSpace{}%
\AgdaSymbol{(}\AgdaBound{β}\AgdaSpace{}%
\AgdaSymbol{:}\AgdaSpace{}%
\AgdaBound{r'}\AgdaSpace{}%
\AgdaOperator{\AgdaDatatype{≡}}\AgdaSpace{}%
\AgdaBound{s}\AgdaSymbol{)}\AgdaSpace{}%
\AgdaSymbol{→}\AgdaSpace{}%
\AgdaBound{q}\AgdaSpace{}%
\AgdaOperator{\AgdaFunction{∙}}\AgdaSpace{}%
\AgdaBound{r'}\AgdaSpace{}%
\AgdaOperator{\AgdaDatatype{≡}}\AgdaSpace{}%
\AgdaBound{q}\AgdaSpace{}%
\AgdaOperator{\AgdaFunction{∙}}\AgdaSpace{}%
\AgdaBound{s}\<%
\\
%
\>[6]\AgdaFunction{d}\AgdaSpace{}%
\AgdaSymbol{:}\AgdaSpace{}%
\AgdaSymbol{(}\AgdaBound{a}\AgdaSpace{}%
\AgdaSymbol{:}\AgdaSpace{}%
\AgdaBound{A}\AgdaSymbol{)}\AgdaSpace{}%
\AgdaSymbol{→}\AgdaSpace{}%
\AgdaFunction{D}\AgdaSpace{}%
\AgdaBound{a}\AgdaSpace{}%
\AgdaBound{a}\AgdaSpace{}%
\AgdaInductiveConstructor{r}\<%
\\
%
\>[6]\AgdaFunction{d}\AgdaSpace{}%
\AgdaBound{a}\AgdaSpace{}%
\AgdaBound{a'}\AgdaSpace{}%
\AgdaSymbol{\{}\AgdaBound{r'}\AgdaSymbol{\}}\AgdaSpace{}%
\AgdaSymbol{\{}\AgdaBound{s}\AgdaSymbol{\}}\AgdaSpace{}%
\AgdaBound{β}\AgdaSpace{}%
\AgdaSymbol{=}\AgdaSpace{}%
\AgdaFunction{iₗ}\AgdaSpace{}%
\AgdaBound{r'}\AgdaSpace{}%
\AgdaOperator{\AgdaFunction{⁻¹}}\AgdaSpace{}%
\AgdaOperator{\AgdaFunction{∙}}\AgdaSpace{}%
\AgdaBound{β}\AgdaSpace{}%
\AgdaOperator{\AgdaFunction{∙}}\AgdaSpace{}%
\AgdaFunction{iₗ}\AgdaSpace{}%
\AgdaBound{s}\<%
\\
%
\\[\AgdaEmptyExtraSkip]%
%
\>[2]\AgdaOperator{\AgdaFunction{\AgdaUnderscore{}⋆\AgdaUnderscore{}}}\AgdaSpace{}%
\AgdaSymbol{:}\AgdaSpace{}%
\AgdaSymbol{\{}\AgdaBound{A}\AgdaSpace{}%
\AgdaSymbol{:}\AgdaSpace{}%
\AgdaPrimitive{Set}\AgdaSymbol{\}}\AgdaSpace{}%
\AgdaSymbol{→}\AgdaSpace{}%
\AgdaSymbol{\{}\AgdaBound{a}\AgdaSpace{}%
\AgdaBound{b}\AgdaSpace{}%
\AgdaBound{c}\AgdaSpace{}%
\AgdaSymbol{:}\AgdaSpace{}%
\AgdaBound{A}\AgdaSymbol{\}}\AgdaSpace{}%
\AgdaSymbol{\{}\AgdaBound{p}\AgdaSpace{}%
\AgdaBound{q}\AgdaSpace{}%
\AgdaSymbol{:}\AgdaSpace{}%
\AgdaBound{a}\AgdaSpace{}%
\AgdaOperator{\AgdaDatatype{≡}}\AgdaSpace{}%
\AgdaBound{b}\AgdaSymbol{\}}\AgdaSpace{}%
\AgdaSymbol{\{}\AgdaBound{r'}\AgdaSpace{}%
\AgdaBound{s}\AgdaSpace{}%
\AgdaSymbol{:}\AgdaSpace{}%
\AgdaBound{b}\AgdaSpace{}%
\AgdaOperator{\AgdaDatatype{≡}}\AgdaSpace{}%
\AgdaBound{c}\AgdaSymbol{\}}\AgdaSpace{}%
\AgdaSymbol{(}\AgdaBound{α}\AgdaSpace{}%
\AgdaSymbol{:}\AgdaSpace{}%
\AgdaBound{p}\AgdaSpace{}%
\AgdaOperator{\AgdaDatatype{≡}}\AgdaSpace{}%
\AgdaBound{q}\AgdaSymbol{)}\AgdaSpace{}%
\AgdaSymbol{(}\AgdaBound{β}\AgdaSpace{}%
\AgdaSymbol{:}\AgdaSpace{}%
\AgdaBound{r'}\AgdaSpace{}%
\AgdaOperator{\AgdaDatatype{≡}}\AgdaSpace{}%
\AgdaBound{s}\AgdaSymbol{)}\AgdaSpace{}%
\AgdaSymbol{→}\AgdaSpace{}%
\AgdaBound{p}\AgdaSpace{}%
\AgdaOperator{\AgdaFunction{∙}}\AgdaSpace{}%
\AgdaBound{r'}\AgdaSpace{}%
\AgdaOperator{\AgdaDatatype{≡}}\AgdaSpace{}%
\AgdaBound{q}\AgdaSpace{}%
\AgdaOperator{\AgdaFunction{∙}}\AgdaSpace{}%
\AgdaBound{s}\<%
\\
%
\>[2]\AgdaOperator{\AgdaFunction{\AgdaUnderscore{}⋆\AgdaUnderscore{}}}\AgdaSpace{}%
\AgdaSymbol{\{}\AgdaBound{A}\AgdaSymbol{\}}\AgdaSpace{}%
\AgdaSymbol{\{}\AgdaArgument{q}\AgdaSpace{}%
\AgdaSymbol{=}\AgdaSpace{}%
\AgdaBound{q}\AgdaSymbol{\}}\AgdaSpace{}%
\AgdaSymbol{\{}\AgdaArgument{r'}\AgdaSpace{}%
\AgdaSymbol{=}\AgdaSpace{}%
\AgdaBound{r'}\AgdaSymbol{\}}\AgdaSpace{}%
\AgdaBound{α}\AgdaSpace{}%
\AgdaBound{β}\AgdaSpace{}%
\AgdaSymbol{=}\AgdaSpace{}%
\AgdaSymbol{(}\AgdaBound{α}\AgdaSpace{}%
\AgdaOperator{\AgdaFunction{∙ᵣ}}\AgdaSpace{}%
\AgdaBound{r'}\AgdaSymbol{)}\AgdaSpace{}%
\AgdaOperator{\AgdaFunction{∙}}\AgdaSpace{}%
\AgdaSymbol{(}\AgdaBound{q}\AgdaSpace{}%
\AgdaOperator{\AgdaFunction{∙ₗ}}\AgdaSpace{}%
\AgdaBound{β}\AgdaSymbol{)}\<%
\\
%
\\[\AgdaEmptyExtraSkip]%
%
\>[2]\AgdaOperator{\AgdaFunction{\AgdaUnderscore{}⋆'\AgdaUnderscore{}}}\AgdaSpace{}%
\AgdaSymbol{:}\AgdaSpace{}%
\AgdaSymbol{\{}\AgdaBound{A}\AgdaSpace{}%
\AgdaSymbol{:}\AgdaSpace{}%
\AgdaPrimitive{Set}\AgdaSymbol{\}}\AgdaSpace{}%
\AgdaSymbol{→}\AgdaSpace{}%
\AgdaSymbol{\{}\AgdaBound{a}\AgdaSpace{}%
\AgdaBound{b}\AgdaSpace{}%
\AgdaBound{c}\AgdaSpace{}%
\AgdaSymbol{:}\AgdaSpace{}%
\AgdaBound{A}\AgdaSymbol{\}}\AgdaSpace{}%
\AgdaSymbol{\{}\AgdaBound{p}\AgdaSpace{}%
\AgdaBound{q}\AgdaSpace{}%
\AgdaSymbol{:}\AgdaSpace{}%
\AgdaBound{a}\AgdaSpace{}%
\AgdaOperator{\AgdaDatatype{≡}}\AgdaSpace{}%
\AgdaBound{b}\AgdaSymbol{\}}\AgdaSpace{}%
\AgdaSymbol{\{}\AgdaBound{r'}\AgdaSpace{}%
\AgdaBound{s}\AgdaSpace{}%
\AgdaSymbol{:}\AgdaSpace{}%
\AgdaBound{b}\AgdaSpace{}%
\AgdaOperator{\AgdaDatatype{≡}}\AgdaSpace{}%
\AgdaBound{c}\AgdaSymbol{\}}\AgdaSpace{}%
\AgdaSymbol{(}\AgdaBound{α}\AgdaSpace{}%
\AgdaSymbol{:}\AgdaSpace{}%
\AgdaBound{p}\AgdaSpace{}%
\AgdaOperator{\AgdaDatatype{≡}}\AgdaSpace{}%
\AgdaBound{q}\AgdaSymbol{)}\AgdaSpace{}%
\AgdaSymbol{(}\AgdaBound{β}\AgdaSpace{}%
\AgdaSymbol{:}\AgdaSpace{}%
\AgdaBound{r'}\AgdaSpace{}%
\AgdaOperator{\AgdaDatatype{≡}}\AgdaSpace{}%
\AgdaBound{s}\AgdaSymbol{)}\AgdaSpace{}%
\AgdaSymbol{→}\AgdaSpace{}%
\AgdaBound{p}\AgdaSpace{}%
\AgdaOperator{\AgdaFunction{∙}}\AgdaSpace{}%
\AgdaBound{r'}\AgdaSpace{}%
\AgdaOperator{\AgdaDatatype{≡}}\AgdaSpace{}%
\AgdaBound{q}\AgdaSpace{}%
\AgdaOperator{\AgdaFunction{∙}}\AgdaSpace{}%
\AgdaBound{s}\<%
\\
%
\>[2]\AgdaOperator{\AgdaFunction{\AgdaUnderscore{}⋆'\AgdaUnderscore{}}}\AgdaSpace{}%
\AgdaSymbol{\{}\AgdaBound{A}\AgdaSymbol{\}}\AgdaSpace{}%
\AgdaSymbol{\{}\AgdaArgument{p}\AgdaSpace{}%
\AgdaSymbol{=}\AgdaSpace{}%
\AgdaBound{p}\AgdaSymbol{\}}\AgdaSpace{}%
\AgdaSymbol{\{}\AgdaArgument{s}\AgdaSpace{}%
\AgdaSymbol{=}\AgdaSpace{}%
\AgdaBound{s}\AgdaSymbol{\}}\AgdaSpace{}%
\AgdaBound{α}\AgdaSpace{}%
\AgdaBound{β}\AgdaSpace{}%
\AgdaSymbol{=}%
\>[34]\AgdaSymbol{(}\AgdaBound{p}\AgdaSpace{}%
\AgdaOperator{\AgdaFunction{∙ₗ}}\AgdaSpace{}%
\AgdaBound{β}\AgdaSymbol{)}\AgdaSpace{}%
\AgdaOperator{\AgdaFunction{∙}}\AgdaSpace{}%
\AgdaSymbol{(}\AgdaBound{α}\AgdaSpace{}%
\AgdaOperator{\AgdaFunction{∙ᵣ}}\AgdaSpace{}%
\AgdaBound{s}\AgdaSymbol{)}\<%
\\
%
\\[\AgdaEmptyExtraSkip]%
%
\>[2]\AgdaFunction{Ω}\AgdaSpace{}%
\AgdaSymbol{:}\AgdaSpace{}%
\AgdaSymbol{\{}\AgdaBound{A}\AgdaSpace{}%
\AgdaSymbol{:}\AgdaSpace{}%
\AgdaPrimitive{Set}\AgdaSymbol{\}}\AgdaSpace{}%
\AgdaSymbol{(}\AgdaBound{a}\AgdaSpace{}%
\AgdaSymbol{:}\AgdaSpace{}%
\AgdaBound{A}\AgdaSymbol{)}\AgdaSpace{}%
\AgdaSymbol{→}\AgdaSpace{}%
\AgdaPrimitive{Set}\<%
\\
%
\>[2]\AgdaFunction{Ω}\AgdaSpace{}%
\AgdaSymbol{\{}\AgdaBound{A}\AgdaSymbol{\}}\AgdaSpace{}%
\AgdaBound{a}\AgdaSpace{}%
\AgdaSymbol{=}\AgdaSpace{}%
\AgdaBound{a}\AgdaSpace{}%
\AgdaOperator{\AgdaDatatype{≡}}\AgdaSpace{}%
\AgdaBound{a}\<%
\\
%
\\[\AgdaEmptyExtraSkip]%
%
\>[2]\AgdaFunction{Ω²}\AgdaSpace{}%
\AgdaSymbol{:}\AgdaSpace{}%
\AgdaSymbol{\{}\AgdaBound{A}\AgdaSpace{}%
\AgdaSymbol{:}\AgdaSpace{}%
\AgdaPrimitive{Set}\AgdaSymbol{\}}\AgdaSpace{}%
\AgdaSymbol{(}\AgdaBound{a}\AgdaSpace{}%
\AgdaSymbol{:}\AgdaSpace{}%
\AgdaBound{A}\AgdaSymbol{)}\AgdaSpace{}%
\AgdaSymbol{→}\AgdaSpace{}%
\AgdaPrimitive{Set}\<%
\\
%
\>[2]\AgdaFunction{Ω²}\AgdaSpace{}%
\AgdaSymbol{\{}\AgdaBound{A}\AgdaSymbol{\}}\AgdaSpace{}%
\AgdaBound{a}\AgdaSpace{}%
\AgdaSymbol{=}\AgdaSpace{}%
\AgdaOperator{\AgdaDatatype{\AgdaUnderscore{}≡\AgdaUnderscore{}}}\AgdaSpace{}%
\AgdaSymbol{\{}\AgdaBound{a}\AgdaSpace{}%
\AgdaOperator{\AgdaDatatype{≡}}\AgdaSpace{}%
\AgdaBound{a}\AgdaSymbol{\}}\AgdaSpace{}%
\AgdaInductiveConstructor{r}\AgdaSpace{}%
\AgdaInductiveConstructor{r}\<%
\\
%
\\[\AgdaEmptyExtraSkip]%
%
\>[2]\AgdaFunction{lem1}\AgdaSpace{}%
\AgdaSymbol{:}\AgdaSpace{}%
\AgdaSymbol{\{}\AgdaBound{A}\AgdaSpace{}%
\AgdaSymbol{:}\AgdaSpace{}%
\AgdaPrimitive{Set}\AgdaSymbol{\}}\AgdaSpace{}%
\AgdaSymbol{→}\AgdaSpace{}%
\AgdaSymbol{(}\AgdaBound{a}\AgdaSpace{}%
\AgdaSymbol{:}\AgdaSpace{}%
\AgdaBound{A}\AgdaSymbol{)}\AgdaSpace{}%
\AgdaSymbol{→}\AgdaSpace{}%
\AgdaSymbol{(}\AgdaBound{α}\AgdaSpace{}%
\AgdaBound{β}\AgdaSpace{}%
\AgdaSymbol{:}\AgdaSpace{}%
\AgdaFunction{Ω²}\AgdaSpace{}%
\AgdaSymbol{\{}\AgdaBound{A}\AgdaSymbol{\}}\AgdaSpace{}%
\AgdaBound{a}\AgdaSymbol{)}\AgdaSpace{}%
\AgdaSymbol{→}\AgdaSpace{}%
\AgdaSymbol{(}\AgdaBound{α}\AgdaSpace{}%
\AgdaOperator{\AgdaFunction{⋆}}\AgdaSpace{}%
\AgdaBound{β}\AgdaSymbol{)}\AgdaSpace{}%
\AgdaOperator{\AgdaDatatype{≡}}\AgdaSpace{}%
\AgdaSymbol{(}\AgdaFunction{iᵣ}\AgdaSpace{}%
\AgdaInductiveConstructor{r}\AgdaSpace{}%
\AgdaOperator{\AgdaFunction{⁻¹}}\AgdaSpace{}%
\AgdaOperator{\AgdaFunction{∙}}\AgdaSpace{}%
\AgdaBound{α}\AgdaSpace{}%
\AgdaOperator{\AgdaFunction{∙}}\AgdaSpace{}%
\AgdaFunction{iᵣ}\AgdaSpace{}%
\AgdaInductiveConstructor{r}\AgdaSymbol{)}\AgdaSpace{}%
\AgdaOperator{\AgdaFunction{∙}}\AgdaSpace{}%
\AgdaSymbol{(}\AgdaFunction{iₗ}\AgdaSpace{}%
\AgdaInductiveConstructor{r}\AgdaSpace{}%
\AgdaOperator{\AgdaFunction{⁻¹}}\AgdaSpace{}%
\AgdaOperator{\AgdaFunction{∙}}\AgdaSpace{}%
\AgdaBound{β}\AgdaSpace{}%
\AgdaOperator{\AgdaFunction{∙}}\AgdaSpace{}%
\AgdaFunction{iₗ}\AgdaSpace{}%
\AgdaInductiveConstructor{r}\AgdaSymbol{)}\<%
\\
%
\>[2]\AgdaFunction{lem1}\AgdaSpace{}%
\AgdaBound{a}\AgdaSpace{}%
\AgdaBound{α}\AgdaSpace{}%
\AgdaBound{β}\AgdaSpace{}%
\AgdaSymbol{=}\AgdaSpace{}%
\AgdaInductiveConstructor{r}\<%
\\
%
\\[\AgdaEmptyExtraSkip]%
%
\>[2]\AgdaFunction{lem1'}\AgdaSpace{}%
\AgdaSymbol{:}\AgdaSpace{}%
\AgdaSymbol{\{}\AgdaBound{A}\AgdaSpace{}%
\AgdaSymbol{:}\AgdaSpace{}%
\AgdaPrimitive{Set}\AgdaSymbol{\}}\AgdaSpace{}%
\AgdaSymbol{→}\AgdaSpace{}%
\AgdaSymbol{(}\AgdaBound{a}\AgdaSpace{}%
\AgdaSymbol{:}\AgdaSpace{}%
\AgdaBound{A}\AgdaSymbol{)}\AgdaSpace{}%
\AgdaSymbol{→}\AgdaSpace{}%
\AgdaSymbol{(}\AgdaBound{α}\AgdaSpace{}%
\AgdaBound{β}\AgdaSpace{}%
\AgdaSymbol{:}\AgdaSpace{}%
\AgdaFunction{Ω²}\AgdaSpace{}%
\AgdaSymbol{\{}\AgdaBound{A}\AgdaSymbol{\}}\AgdaSpace{}%
\AgdaBound{a}\AgdaSymbol{)}\AgdaSpace{}%
\AgdaSymbol{→}\AgdaSpace{}%
\AgdaSymbol{(}\AgdaBound{α}\AgdaSpace{}%
\AgdaOperator{\AgdaFunction{⋆'}}\AgdaSpace{}%
\AgdaBound{β}\AgdaSymbol{)}\AgdaSpace{}%
\AgdaOperator{\AgdaDatatype{≡}}%
\>[63]\AgdaSymbol{(}\AgdaFunction{iₗ}\AgdaSpace{}%
\AgdaInductiveConstructor{r}\AgdaSpace{}%
\AgdaOperator{\AgdaFunction{⁻¹}}\AgdaSpace{}%
\AgdaOperator{\AgdaFunction{∙}}\AgdaSpace{}%
\AgdaBound{β}\AgdaSpace{}%
\AgdaOperator{\AgdaFunction{∙}}\AgdaSpace{}%
\AgdaFunction{iₗ}\AgdaSpace{}%
\AgdaInductiveConstructor{r}\AgdaSymbol{)}\AgdaSpace{}%
\AgdaOperator{\AgdaFunction{∙}}\AgdaSpace{}%
\AgdaSymbol{(}\AgdaFunction{iᵣ}\AgdaSpace{}%
\AgdaInductiveConstructor{r}\AgdaSpace{}%
\AgdaOperator{\AgdaFunction{⁻¹}}\AgdaSpace{}%
\AgdaOperator{\AgdaFunction{∙}}\AgdaSpace{}%
\AgdaBound{α}\AgdaSpace{}%
\AgdaOperator{\AgdaFunction{∙}}\AgdaSpace{}%
\AgdaFunction{iᵣ}\AgdaSpace{}%
\AgdaInductiveConstructor{r}\AgdaSymbol{)}\<%
\\
%
\>[2]\AgdaFunction{lem1'}\AgdaSpace{}%
\AgdaBound{a}\AgdaSpace{}%
\AgdaBound{α}\AgdaSpace{}%
\AgdaBound{β}\AgdaSpace{}%
\AgdaSymbol{=}\AgdaSpace{}%
\AgdaInductiveConstructor{r}\<%
\\
%
\\[\AgdaEmptyExtraSkip]%
%
\>[2]\AgdaComment{-- ap\textbackslash{}\AgdaUnderscore{}}\<%
\\
%
\>[2]\AgdaFunction{apf}\AgdaSpace{}%
\AgdaSymbol{:}\AgdaSpace{}%
\AgdaSymbol{\{}\AgdaBound{A}\AgdaSpace{}%
\AgdaBound{B}\AgdaSpace{}%
\AgdaSymbol{:}\AgdaSpace{}%
\AgdaPrimitive{Set}\AgdaSymbol{\}}\AgdaSpace{}%
\AgdaSymbol{→}\AgdaSpace{}%
\AgdaSymbol{\{}\AgdaBound{x}\AgdaSpace{}%
\AgdaBound{y}\AgdaSpace{}%
\AgdaSymbol{:}\AgdaSpace{}%
\AgdaBound{A}\AgdaSymbol{\}}\AgdaSpace{}%
\AgdaSymbol{→}\AgdaSpace{}%
\AgdaSymbol{(}\AgdaBound{f}\AgdaSpace{}%
\AgdaSymbol{:}\AgdaSpace{}%
\AgdaBound{A}\AgdaSpace{}%
\AgdaSymbol{→}\AgdaSpace{}%
\AgdaBound{B}\AgdaSymbol{)}\AgdaSpace{}%
\AgdaSymbol{→}\AgdaSpace{}%
\AgdaSymbol{(}\AgdaBound{x}\AgdaSpace{}%
\AgdaOperator{\AgdaDatatype{≡}}\AgdaSpace{}%
\AgdaBound{y}\AgdaSymbol{)}\AgdaSpace{}%
\AgdaSymbol{→}\AgdaSpace{}%
\AgdaBound{f}\AgdaSpace{}%
\AgdaBound{x}\AgdaSpace{}%
\AgdaOperator{\AgdaDatatype{≡}}\AgdaSpace{}%
\AgdaBound{f}\AgdaSpace{}%
\AgdaBound{y}\<%
\\
%
\>[2]\AgdaFunction{apf}\AgdaSpace{}%
\AgdaSymbol{\{}\AgdaBound{A}\AgdaSymbol{\}}\AgdaSpace{}%
\AgdaSymbol{\{}\AgdaBound{B}\AgdaSymbol{\}}\AgdaSpace{}%
\AgdaSymbol{\{}\AgdaBound{x}\AgdaSymbol{\}}\AgdaSpace{}%
\AgdaSymbol{\{}\AgdaBound{y}\AgdaSymbol{\}}\AgdaSpace{}%
\AgdaBound{f}\AgdaSpace{}%
\AgdaBound{p}\AgdaSpace{}%
\AgdaSymbol{=}\AgdaSpace{}%
\AgdaFunction{J}\AgdaSpace{}%
\AgdaFunction{D}\AgdaSpace{}%
\AgdaFunction{d}\AgdaSpace{}%
\AgdaBound{x}\AgdaSpace{}%
\AgdaBound{y}\AgdaSpace{}%
\AgdaBound{p}\<%
\\
\>[2][@{}l@{\AgdaIndent{0}}]%
\>[4]\AgdaKeyword{where}\<%
\\
\>[4][@{}l@{\AgdaIndent{0}}]%
\>[6]\AgdaFunction{D}\AgdaSpace{}%
\AgdaSymbol{:}\AgdaSpace{}%
\AgdaSymbol{(}\AgdaBound{x}\AgdaSpace{}%
\AgdaBound{y}\AgdaSpace{}%
\AgdaSymbol{:}\AgdaSpace{}%
\AgdaBound{A}\AgdaSymbol{)}\AgdaSpace{}%
\AgdaSymbol{→}\AgdaSpace{}%
\AgdaBound{x}\AgdaSpace{}%
\AgdaOperator{\AgdaDatatype{≡}}\AgdaSpace{}%
\AgdaBound{y}\AgdaSpace{}%
\AgdaSymbol{→}\AgdaSpace{}%
\AgdaPrimitive{Set}\<%
\\
%
\>[6]\AgdaFunction{D}\AgdaSpace{}%
\AgdaBound{x}\AgdaSpace{}%
\AgdaBound{y}\AgdaSpace{}%
\AgdaBound{p}\AgdaSpace{}%
\AgdaSymbol{=}\AgdaSpace{}%
\AgdaSymbol{\{}\AgdaBound{f}\AgdaSpace{}%
\AgdaSymbol{:}\AgdaSpace{}%
\AgdaBound{A}\AgdaSpace{}%
\AgdaSymbol{→}\AgdaSpace{}%
\AgdaBound{B}\AgdaSymbol{\}}\AgdaSpace{}%
\AgdaSymbol{→}\AgdaSpace{}%
\AgdaBound{f}\AgdaSpace{}%
\AgdaBound{x}\AgdaSpace{}%
\AgdaOperator{\AgdaDatatype{≡}}\AgdaSpace{}%
\AgdaBound{f}\AgdaSpace{}%
\AgdaBound{y}\<%
\\
%
\>[6]\AgdaFunction{d}\AgdaSpace{}%
\AgdaSymbol{:}\AgdaSpace{}%
\AgdaSymbol{(}\AgdaBound{x}\AgdaSpace{}%
\AgdaSymbol{:}\AgdaSpace{}%
\AgdaBound{A}\AgdaSymbol{)}\AgdaSpace{}%
\AgdaSymbol{→}\AgdaSpace{}%
\AgdaFunction{D}\AgdaSpace{}%
\AgdaBound{x}\AgdaSpace{}%
\AgdaBound{x}\AgdaSpace{}%
\AgdaInductiveConstructor{r}\<%
\\
%
\>[6]\AgdaFunction{d}\AgdaSpace{}%
\AgdaSymbol{=}\AgdaSpace{}%
\AgdaSymbol{λ}\AgdaSpace{}%
\AgdaBound{x}\AgdaSpace{}%
\AgdaSymbol{→}\AgdaSpace{}%
\AgdaInductiveConstructor{r}\<%
\\
%
\\[\AgdaEmptyExtraSkip]%
%
\>[2]\AgdaFunction{ap}\AgdaSpace{}%
\AgdaSymbol{:}\AgdaSpace{}%
\AgdaSymbol{\{}\AgdaBound{A}\AgdaSpace{}%
\AgdaBound{B}\AgdaSpace{}%
\AgdaSymbol{:}\AgdaSpace{}%
\AgdaPrimitive{Set}\AgdaSymbol{\}}\AgdaSpace{}%
\AgdaSymbol{→}\AgdaSpace{}%
\AgdaSymbol{\{}\AgdaBound{x}\AgdaSpace{}%
\AgdaBound{y}\AgdaSpace{}%
\AgdaSymbol{:}\AgdaSpace{}%
\AgdaBound{A}\AgdaSymbol{\}}\AgdaSpace{}%
\AgdaSymbol{→}\AgdaSpace{}%
\AgdaSymbol{(}\AgdaBound{f}\AgdaSpace{}%
\AgdaSymbol{:}\AgdaSpace{}%
\AgdaBound{A}\AgdaSpace{}%
\AgdaSymbol{→}\AgdaSpace{}%
\AgdaBound{B}\AgdaSymbol{)}\AgdaSpace{}%
\AgdaSymbol{→}\AgdaSpace{}%
\AgdaSymbol{(}\AgdaBound{x}\AgdaSpace{}%
\AgdaOperator{\AgdaDatatype{≡}}\AgdaSpace{}%
\AgdaBound{y}\AgdaSymbol{)}\AgdaSpace{}%
\AgdaSymbol{→}\AgdaSpace{}%
\AgdaBound{f}\AgdaSpace{}%
\AgdaBound{x}\AgdaSpace{}%
\AgdaOperator{\AgdaDatatype{≡}}\AgdaSpace{}%
\AgdaBound{f}\AgdaSpace{}%
\AgdaBound{y}\<%
\\
%
\>[2]\AgdaFunction{ap}\AgdaSpace{}%
\AgdaBound{f}\AgdaSpace{}%
\AgdaInductiveConstructor{r}\AgdaSpace{}%
\AgdaSymbol{=}\AgdaSpace{}%
\AgdaInductiveConstructor{r}\<%
\\
%
\\[\AgdaEmptyExtraSkip]%
%
\>[2]\AgdaFunction{lem20}\AgdaSpace{}%
\AgdaSymbol{:}\AgdaSpace{}%
\AgdaSymbol{\{}\AgdaBound{A}\AgdaSpace{}%
\AgdaSymbol{:}\AgdaSpace{}%
\AgdaPrimitive{Set}\AgdaSymbol{\}}\AgdaSpace{}%
\AgdaSymbol{→}\AgdaSpace{}%
\AgdaSymbol{\{}\AgdaBound{a}\AgdaSpace{}%
\AgdaSymbol{:}\AgdaSpace{}%
\AgdaBound{A}\AgdaSymbol{\}}\AgdaSpace{}%
\AgdaSymbol{→}\AgdaSpace{}%
\AgdaSymbol{(}\AgdaBound{α}\AgdaSpace{}%
\AgdaSymbol{:}\AgdaSpace{}%
\AgdaFunction{Ω²}\AgdaSpace{}%
\AgdaSymbol{\{}\AgdaBound{A}\AgdaSymbol{\}}\AgdaSpace{}%
\AgdaBound{a}\AgdaSymbol{)}\AgdaSpace{}%
\AgdaSymbol{→}\AgdaSpace{}%
\AgdaSymbol{(}\AgdaFunction{iᵣ}\AgdaSpace{}%
\AgdaInductiveConstructor{r}\AgdaSpace{}%
\AgdaOperator{\AgdaFunction{⁻¹}}\AgdaSpace{}%
\AgdaOperator{\AgdaFunction{∙}}\AgdaSpace{}%
\AgdaBound{α}\AgdaSpace{}%
\AgdaOperator{\AgdaFunction{∙}}\AgdaSpace{}%
\AgdaFunction{iᵣ}\AgdaSpace{}%
\AgdaInductiveConstructor{r}\AgdaSymbol{)}\AgdaSpace{}%
\AgdaOperator{\AgdaDatatype{≡}}\AgdaSpace{}%
\AgdaBound{α}\<%
\\
%
\>[2]\AgdaFunction{lem20}\AgdaSpace{}%
\AgdaBound{α}\AgdaSpace{}%
\AgdaSymbol{=}\AgdaSpace{}%
\AgdaFunction{iᵣ}\AgdaSpace{}%
\AgdaSymbol{(}\AgdaBound{α}\AgdaSymbol{)}\AgdaSpace{}%
\AgdaOperator{\AgdaFunction{⁻¹}}\<%
\\
%
\\[\AgdaEmptyExtraSkip]%
%
\>[2]\AgdaFunction{lem21}\AgdaSpace{}%
\AgdaSymbol{:}\AgdaSpace{}%
\AgdaSymbol{\{}\AgdaBound{A}\AgdaSpace{}%
\AgdaSymbol{:}\AgdaSpace{}%
\AgdaPrimitive{Set}\AgdaSymbol{\}}\AgdaSpace{}%
\AgdaSymbol{→}\AgdaSpace{}%
\AgdaSymbol{\{}\AgdaBound{a}\AgdaSpace{}%
\AgdaSymbol{:}\AgdaSpace{}%
\AgdaBound{A}\AgdaSymbol{\}}\AgdaSpace{}%
\AgdaSymbol{→}\AgdaSpace{}%
\AgdaSymbol{(}\AgdaBound{β}\AgdaSpace{}%
\AgdaSymbol{:}\AgdaSpace{}%
\AgdaFunction{Ω²}\AgdaSpace{}%
\AgdaSymbol{\{}\AgdaBound{A}\AgdaSymbol{\}}\AgdaSpace{}%
\AgdaBound{a}\AgdaSymbol{)}\AgdaSpace{}%
\AgdaSymbol{→}\AgdaSpace{}%
\AgdaSymbol{(}\AgdaFunction{iₗ}\AgdaSpace{}%
\AgdaInductiveConstructor{r}\AgdaSpace{}%
\AgdaOperator{\AgdaFunction{⁻¹}}\AgdaSpace{}%
\AgdaOperator{\AgdaFunction{∙}}\AgdaSpace{}%
\AgdaBound{β}\AgdaSpace{}%
\AgdaOperator{\AgdaFunction{∙}}\AgdaSpace{}%
\AgdaFunction{iₗ}\AgdaSpace{}%
\AgdaInductiveConstructor{r}\AgdaSymbol{)}\AgdaSpace{}%
\AgdaOperator{\AgdaDatatype{≡}}\AgdaSpace{}%
\AgdaBound{β}\<%
\\
%
\>[2]\AgdaFunction{lem21}\AgdaSpace{}%
\AgdaBound{β}\AgdaSpace{}%
\AgdaSymbol{=}\AgdaSpace{}%
\AgdaFunction{iᵣ}\AgdaSpace{}%
\AgdaSymbol{(}\AgdaBound{β}\AgdaSymbol{)}\AgdaSpace{}%
\AgdaOperator{\AgdaFunction{⁻¹}}\<%
\\
%
\\[\AgdaEmptyExtraSkip]%
%
\>[2]\AgdaFunction{lem2}\AgdaSpace{}%
\AgdaSymbol{:}\AgdaSpace{}%
\AgdaSymbol{\{}\AgdaBound{A}\AgdaSpace{}%
\AgdaSymbol{:}\AgdaSpace{}%
\AgdaPrimitive{Set}\AgdaSymbol{\}}\AgdaSpace{}%
\AgdaSymbol{→}\AgdaSpace{}%
\AgdaSymbol{(}\AgdaBound{a}\AgdaSpace{}%
\AgdaSymbol{:}\AgdaSpace{}%
\AgdaBound{A}\AgdaSymbol{)}\AgdaSpace{}%
\AgdaSymbol{→}\AgdaSpace{}%
\AgdaSymbol{(}\AgdaBound{α}\AgdaSpace{}%
\AgdaBound{β}\AgdaSpace{}%
\AgdaSymbol{:}\AgdaSpace{}%
\AgdaFunction{Ω²}\AgdaSpace{}%
\AgdaSymbol{\{}\AgdaBound{A}\AgdaSymbol{\}}\AgdaSpace{}%
\AgdaBound{a}\AgdaSymbol{)}\AgdaSpace{}%
\AgdaSymbol{→}\AgdaSpace{}%
\AgdaSymbol{(}\AgdaFunction{iᵣ}\AgdaSpace{}%
\AgdaInductiveConstructor{r}\AgdaSpace{}%
\AgdaOperator{\AgdaFunction{⁻¹}}\AgdaSpace{}%
\AgdaOperator{\AgdaFunction{∙}}\AgdaSpace{}%
\AgdaBound{α}\AgdaSpace{}%
\AgdaOperator{\AgdaFunction{∙}}\AgdaSpace{}%
\AgdaFunction{iᵣ}\AgdaSpace{}%
\AgdaInductiveConstructor{r}\AgdaSymbol{)}\AgdaSpace{}%
\AgdaOperator{\AgdaFunction{∙}}\AgdaSpace{}%
\AgdaSymbol{(}\AgdaFunction{iₗ}\AgdaSpace{}%
\AgdaInductiveConstructor{r}\AgdaSpace{}%
\AgdaOperator{\AgdaFunction{⁻¹}}\AgdaSpace{}%
\AgdaOperator{\AgdaFunction{∙}}\AgdaSpace{}%
\AgdaBound{β}\AgdaSpace{}%
\AgdaOperator{\AgdaFunction{∙}}\AgdaSpace{}%
\AgdaFunction{iₗ}\AgdaSpace{}%
\AgdaInductiveConstructor{r}\AgdaSymbol{)}\AgdaSpace{}%
\AgdaOperator{\AgdaDatatype{≡}}\AgdaSpace{}%
\AgdaSymbol{(}\AgdaBound{α}\AgdaSpace{}%
\AgdaOperator{\AgdaFunction{∙}}\AgdaSpace{}%
\AgdaBound{β}\AgdaSymbol{)}\<%
\\
%
\>[2]\AgdaFunction{lem2}\AgdaSpace{}%
\AgdaSymbol{\{}\AgdaBound{A}\AgdaSymbol{\}}\AgdaSpace{}%
\AgdaBound{a}\AgdaSpace{}%
\AgdaBound{α}\AgdaSpace{}%
\AgdaBound{β}\AgdaSpace{}%
\AgdaSymbol{=}\AgdaSpace{}%
\AgdaFunction{apf}\AgdaSpace{}%
\AgdaSymbol{(λ}\AgdaSpace{}%
\AgdaBound{-}\AgdaSpace{}%
\AgdaSymbol{→}\AgdaSpace{}%
\AgdaBound{-}\AgdaSpace{}%
\AgdaOperator{\AgdaFunction{∙}}\AgdaSpace{}%
\AgdaSymbol{(}\AgdaFunction{iₗ}\AgdaSpace{}%
\AgdaInductiveConstructor{r}\AgdaSpace{}%
\AgdaOperator{\AgdaFunction{⁻¹}}\AgdaSpace{}%
\AgdaOperator{\AgdaFunction{∙}}\AgdaSpace{}%
\AgdaBound{β}\AgdaSpace{}%
\AgdaOperator{\AgdaFunction{∙}}\AgdaSpace{}%
\AgdaFunction{iₗ}\AgdaSpace{}%
\AgdaInductiveConstructor{r}\AgdaSymbol{)}\AgdaSpace{}%
\AgdaSymbol{)}\AgdaSpace{}%
\AgdaSymbol{(}\AgdaFunction{lem20}\AgdaSpace{}%
\AgdaBound{α}\AgdaSymbol{)}\AgdaSpace{}%
\AgdaOperator{\AgdaFunction{∙}}\AgdaSpace{}%
\AgdaFunction{apf}\AgdaSpace{}%
\AgdaSymbol{(λ}\AgdaSpace{}%
\AgdaBound{-}\AgdaSpace{}%
\AgdaSymbol{→}\AgdaSpace{}%
\AgdaBound{α}\AgdaSpace{}%
\AgdaOperator{\AgdaFunction{∙}}\AgdaSpace{}%
\AgdaBound{-}\AgdaSymbol{)}\AgdaSpace{}%
\AgdaSymbol{(}\AgdaFunction{lem21}\AgdaSpace{}%
\AgdaBound{β}\AgdaSymbol{)}\<%
\\
%
\\[\AgdaEmptyExtraSkip]%
%
\>[2]\AgdaFunction{lem2'}\AgdaSpace{}%
\AgdaSymbol{:}\AgdaSpace{}%
\AgdaSymbol{\{}\AgdaBound{A}\AgdaSpace{}%
\AgdaSymbol{:}\AgdaSpace{}%
\AgdaPrimitive{Set}\AgdaSymbol{\}}\AgdaSpace{}%
\AgdaSymbol{→}\AgdaSpace{}%
\AgdaSymbol{(}\AgdaBound{a}\AgdaSpace{}%
\AgdaSymbol{:}\AgdaSpace{}%
\AgdaBound{A}\AgdaSymbol{)}\AgdaSpace{}%
\AgdaSymbol{→}\AgdaSpace{}%
\AgdaSymbol{(}\AgdaBound{α}\AgdaSpace{}%
\AgdaBound{β}\AgdaSpace{}%
\AgdaSymbol{:}\AgdaSpace{}%
\AgdaFunction{Ω²}\AgdaSpace{}%
\AgdaSymbol{\{}\AgdaBound{A}\AgdaSymbol{\}}\AgdaSpace{}%
\AgdaBound{a}\AgdaSymbol{)}\AgdaSpace{}%
\AgdaSymbol{→}\AgdaSpace{}%
\AgdaSymbol{(}\AgdaFunction{iₗ}\AgdaSpace{}%
\AgdaInductiveConstructor{r}\AgdaSpace{}%
\AgdaOperator{\AgdaFunction{⁻¹}}\AgdaSpace{}%
\AgdaOperator{\AgdaFunction{∙}}\AgdaSpace{}%
\AgdaBound{β}\AgdaSpace{}%
\AgdaOperator{\AgdaFunction{∙}}\AgdaSpace{}%
\AgdaFunction{iₗ}\AgdaSpace{}%
\AgdaInductiveConstructor{r}\AgdaSymbol{)}\AgdaSpace{}%
\AgdaOperator{\AgdaFunction{∙}}\AgdaSpace{}%
\AgdaSymbol{(}\AgdaFunction{iᵣ}\AgdaSpace{}%
\AgdaInductiveConstructor{r}\AgdaSpace{}%
\AgdaOperator{\AgdaFunction{⁻¹}}\AgdaSpace{}%
\AgdaOperator{\AgdaFunction{∙}}\AgdaSpace{}%
\AgdaBound{α}\AgdaSpace{}%
\AgdaOperator{\AgdaFunction{∙}}\AgdaSpace{}%
\AgdaFunction{iᵣ}\AgdaSpace{}%
\AgdaInductiveConstructor{r}\AgdaSymbol{)}\AgdaSpace{}%
\AgdaOperator{\AgdaDatatype{≡}}\AgdaSpace{}%
\AgdaSymbol{(}\AgdaBound{β}\AgdaSpace{}%
\AgdaOperator{\AgdaFunction{∙}}\AgdaSpace{}%
\AgdaBound{α}\AgdaSpace{}%
\AgdaSymbol{)}\<%
\\
%
\>[2]\AgdaFunction{lem2'}\AgdaSpace{}%
\AgdaSymbol{\{}\AgdaBound{A}\AgdaSymbol{\}}\AgdaSpace{}%
\AgdaBound{a}\AgdaSpace{}%
\AgdaBound{α}\AgdaSpace{}%
\AgdaBound{β}\AgdaSpace{}%
\AgdaSymbol{=}%
\>[21]\AgdaFunction{apf}%
\>[26]\AgdaSymbol{(λ}\AgdaSpace{}%
\AgdaBound{-}\AgdaSpace{}%
\AgdaSymbol{→}\AgdaSpace{}%
\AgdaBound{-}\AgdaSpace{}%
\AgdaOperator{\AgdaFunction{∙}}\AgdaSpace{}%
\AgdaSymbol{(}\AgdaFunction{iᵣ}\AgdaSpace{}%
\AgdaInductiveConstructor{r}\AgdaSpace{}%
\AgdaOperator{\AgdaFunction{⁻¹}}\AgdaSpace{}%
\AgdaOperator{\AgdaFunction{∙}}\AgdaSpace{}%
\AgdaBound{α}\AgdaSpace{}%
\AgdaOperator{\AgdaFunction{∙}}\AgdaSpace{}%
\AgdaFunction{iᵣ}\AgdaSpace{}%
\AgdaInductiveConstructor{r}\AgdaSymbol{))}\AgdaSpace{}%
\AgdaSymbol{(}\AgdaFunction{lem21}\AgdaSpace{}%
\AgdaBound{β}\AgdaSymbol{)}\AgdaSpace{}%
\AgdaOperator{\AgdaFunction{∙}}\AgdaSpace{}%
\AgdaFunction{apf}\AgdaSpace{}%
\AgdaSymbol{(λ}\AgdaSpace{}%
\AgdaBound{-}\AgdaSpace{}%
\AgdaSymbol{→}\AgdaSpace{}%
\AgdaBound{β}\AgdaSpace{}%
\AgdaOperator{\AgdaFunction{∙}}\AgdaSpace{}%
\AgdaBound{-}\AgdaSymbol{)}\AgdaSpace{}%
\AgdaSymbol{(}\AgdaFunction{lem20}\AgdaSpace{}%
\AgdaBound{α}\AgdaSymbol{)}\<%
\\
%
\>[2]\AgdaComment{-- apf (λ - → - ∙ (iₗ r ⁻¹ ∙ β ∙ iₗ r) ) (lem20 α) ∙ apf (λ - → α ∙ -) (lem21 β)}\<%
\\
%
\\[\AgdaEmptyExtraSkip]%
%
\>[2]\AgdaFunction{⋆≡∙}\AgdaSpace{}%
\AgdaSymbol{:}\AgdaSpace{}%
\AgdaSymbol{\{}\AgdaBound{A}\AgdaSpace{}%
\AgdaSymbol{:}\AgdaSpace{}%
\AgdaPrimitive{Set}\AgdaSymbol{\}}\AgdaSpace{}%
\AgdaSymbol{→}\AgdaSpace{}%
\AgdaSymbol{(}\AgdaBound{a}\AgdaSpace{}%
\AgdaSymbol{:}\AgdaSpace{}%
\AgdaBound{A}\AgdaSymbol{)}\AgdaSpace{}%
\AgdaSymbol{→}\AgdaSpace{}%
\AgdaSymbol{(}\AgdaBound{α}\AgdaSpace{}%
\AgdaBound{β}\AgdaSpace{}%
\AgdaSymbol{:}\AgdaSpace{}%
\AgdaFunction{Ω²}\AgdaSpace{}%
\AgdaSymbol{\{}\AgdaBound{A}\AgdaSymbol{\}}\AgdaSpace{}%
\AgdaBound{a}\AgdaSymbol{)}\AgdaSpace{}%
\AgdaSymbol{→}\AgdaSpace{}%
\AgdaSymbol{(}\AgdaBound{α}\AgdaSpace{}%
\AgdaOperator{\AgdaFunction{⋆}}\AgdaSpace{}%
\AgdaBound{β}\AgdaSymbol{)}\AgdaSpace{}%
\AgdaOperator{\AgdaDatatype{≡}}\AgdaSpace{}%
\AgdaSymbol{(}\AgdaBound{α}\AgdaSpace{}%
\AgdaOperator{\AgdaFunction{∙}}\AgdaSpace{}%
\AgdaBound{β}\AgdaSymbol{)}\<%
\\
%
\>[2]\AgdaFunction{⋆≡∙}\AgdaSpace{}%
\AgdaBound{a}\AgdaSpace{}%
\AgdaBound{α}\AgdaSpace{}%
\AgdaBound{β}\AgdaSpace{}%
\AgdaSymbol{=}\AgdaSpace{}%
\AgdaFunction{lem1}\AgdaSpace{}%
\AgdaBound{a}\AgdaSpace{}%
\AgdaBound{α}\AgdaSpace{}%
\AgdaBound{β}\AgdaSpace{}%
\AgdaOperator{\AgdaFunction{∙}}\AgdaSpace{}%
\AgdaFunction{lem2}\AgdaSpace{}%
\AgdaBound{a}\AgdaSpace{}%
\AgdaBound{α}\AgdaSpace{}%
\AgdaBound{β}\<%
\\
%
\\[\AgdaEmptyExtraSkip]%
%
\>[2]\AgdaComment{-- proven similairly to above }\<%
\\
%
\>[2]\AgdaFunction{⋆'≡∙}\AgdaSpace{}%
\AgdaSymbol{:}\AgdaSpace{}%
\AgdaSymbol{\{}\AgdaBound{A}\AgdaSpace{}%
\AgdaSymbol{:}\AgdaSpace{}%
\AgdaPrimitive{Set}\AgdaSymbol{\}}\AgdaSpace{}%
\AgdaSymbol{→}\AgdaSpace{}%
\AgdaSymbol{(}\AgdaBound{a}\AgdaSpace{}%
\AgdaSymbol{:}\AgdaSpace{}%
\AgdaBound{A}\AgdaSymbol{)}\AgdaSpace{}%
\AgdaSymbol{→}\AgdaSpace{}%
\AgdaSymbol{(}\AgdaBound{α}\AgdaSpace{}%
\AgdaBound{β}\AgdaSpace{}%
\AgdaSymbol{:}\AgdaSpace{}%
\AgdaFunction{Ω²}\AgdaSpace{}%
\AgdaSymbol{\{}\AgdaBound{A}\AgdaSymbol{\}}\AgdaSpace{}%
\AgdaBound{a}\AgdaSymbol{)}\AgdaSpace{}%
\AgdaSymbol{→}\AgdaSpace{}%
\AgdaSymbol{(}\AgdaBound{α}\AgdaSpace{}%
\AgdaOperator{\AgdaFunction{⋆'}}\AgdaSpace{}%
\AgdaBound{β}\AgdaSymbol{)}\AgdaSpace{}%
\AgdaOperator{\AgdaDatatype{≡}}\AgdaSpace{}%
\AgdaSymbol{(}\AgdaBound{β}\AgdaSpace{}%
\AgdaOperator{\AgdaFunction{∙}}\AgdaSpace{}%
\AgdaBound{α}\AgdaSymbol{)}\<%
\\
%
\>[2]\AgdaFunction{⋆'≡∙}\AgdaSpace{}%
\AgdaBound{a}\AgdaSpace{}%
\AgdaBound{α}\AgdaSpace{}%
\AgdaBound{β}\AgdaSpace{}%
\AgdaSymbol{=}\AgdaSpace{}%
\AgdaFunction{lem1'}\AgdaSpace{}%
\AgdaBound{a}\AgdaSpace{}%
\AgdaBound{α}\AgdaSpace{}%
\AgdaBound{β}\AgdaSpace{}%
\AgdaOperator{\AgdaFunction{∙}}\AgdaSpace{}%
\AgdaFunction{lem2'}\AgdaSpace{}%
\AgdaBound{a}\AgdaSpace{}%
\AgdaBound{α}\AgdaSpace{}%
\AgdaBound{β}\<%
\\
%
\\[\AgdaEmptyExtraSkip]%
%
\\[\AgdaEmptyExtraSkip]%
%
\>[2]\AgdaComment{--eckmannHilton : \{A : Set\} → (a : A) → (α β : Ω² \{A\} a) → α ∙ β ≡ β ∙ α }\<%
\\
%
\>[2]\AgdaComment{--eckmannHilton a r r = r}\<%
\\
\>[0]\<%
\end{code}

[TODO, fix without k errors]




\section{A Spectrum of GF Grammars for types}

We now discuss the various iterations of code which experimented with NL aspects

We should again emphasize the role of, in particular, Rantas two grammars, one
formalizing logic, and the other working with a case study of a real text\cite{aarneHott}



We now discuss the GF implementation, capable of parsing both natural language
and Agda syntax. The parser was appropriated from the cubicaltt BNFC parser,
de-cubified and then gf-ified. The languages are tightly coupled, so the
translation is actually quite simple. Some main differences are:

\begin{itemize}[noitemsep]

\item GF treats abstract and concrete syntax seperately. This allows GF to
support many concrete syntax implementation of a given grammar

\item Fixity is dealt with at the concrete syntax layer in GF.  This allows for
more refined control of fixity, but also results in difficulties : during
linearization their can be the insertion of extra parens.

\item GF supports dependent hypes and higher order abstract syntax, which makes
it suitable to typecheck at the parsing stage. It would very interesting to see
if this is interoperable with the current version of this work in later
iterations [Todo - add github link referncing work I've done in this direction]

\item GF also is enhanced by a PGF back-end, allowing an embedding of grammars
into, among other languages, Haskell.

\end{itemize}

While GF is targeted towards natural language translation, there's nothing
stopping it from being used as a PL tool as well, like, for instance, the
front-end of a compiler. The innovation of this thesis is to combine both uses,
thereby allowing translation between Controlled Natural Languages and
programming languages.

Example expressions the grammar can parse are seen below, which have been
verified by hand to be isomorphic to the corresponding cubicaltt BNFC trees:

\begin{verbatim}

data bool : Set where true | false 
data nat : Set where zero | suc ( n : nat )  
caseBool ( x : Set ) ( y z : x ) : bool -> Set = split false -> y || true -> z
indBool ( x : bool -> Set ) ( y : x false ) ( z : x true ) : ( b : bool ) -> x b = split false -> y || true  -> z
funExt  ( a : Set )   ( b : a -> Set )   ( f g :  ( x : a )  -> b x )   ( p :  ( x : a )  -> ( b x )   ( f x ) == ( g x )  )  : (  ( y : a )  -> b y )  f == g = undefined
foo ( b : bool ) : bool = b

\end{verbatim}

[Todo] add use cases



\include{horizon}

\section{Goals and Challenges}

The parser is still quite primitive, and needs to be extended extensively to
support natural language ambiguity in mathematics as well as other linguistic
nuance that GF captures well, like tense and aspect. This can follow a method
expored in Aarne's paper : "Translating between Language and Logic: What Is
Easy and What Is Difficult" where one develops a denotational semantics for
translating between natural language expressions with the desired AST. The bulk
of this work will be writing a Haskell back-end implementing this AST
transformation. The extended syntax, designed for linguistic nuance, will be
filtered into the core syntax, which is essentially what I have done.

The Resource Grammar Library (RGL) is designed for out-of-the box grammar
writing, and therefore much of the linearization nuance can be outsourced to
this robust and well-studied library. Nonetheless, each application grammar
brings its own unique challenges, and the RGL will only get one so far. My
linearization may require extensive tweaking.

Thus far, our parser is only able to parse non-cubical fragments of the
cubicalTT standard library. Dealing with Agda pattern matching, it was
realized, is outside the theoretical boundaries of GF (at least, if one were to
do it in a non ad-hoc way) due to its inability to pass arbitrary strings down
the syntax tree nodes during linearization. Pattern matching therefore needs to be dealt
with via pre and post processing.  Additionally, cubicaltt is weaker at
dealing with telescopes than Agda, and so a full generalization to Agda is not
yet possible. Universes are another feature future iterations of this Grammar
would need to deal with, but as they aren't present in most mathematician's
vernacular, it is not seen as relevant for the state of this project.

Records should also be added, but because this grammar supports sigma types,
there is no rush. The Identity type is so far deeply embedded in our grammar,
so the first code fragment may just be for explanatory purposes.  The degree to
which the library is extended to cover domain specific information is up to
debate, but for now the grammar is meant to be kept as minimal as possible.

One interesting extension, time dependnet, would be to allow for a bidrectional
feedback between GF and Agda : thereby allowing ad hoc extensions to GF's ASTs
to allow for newly defined Agda functions to be treated with more care, i.e.
have an arguement structure rather than just treating everything as variables.
This may be too ambitious for the time being.

% Random ideas

Category theory in agda paper, differences in formalization

* my agda hott library
* escardo's hott library 
  - if successful on mine, with universe support
  - mix of latex, agda code , and natural language 
* dummy example for non-hott math (spivak et al, type-in-type)
* alternatively, trying digging in the mountain at the other end, and try extedning ad-hoc grammar with various syntactic nuance
* Latex & Unicode support  - 
* Degenerate cases
  - find examples which are unable to be supported by this grammar, explain why and offer future possible patches

Talk about all the things that need to be done

Pattern Matching, additional parser vs internal to GF

How to decide an optimal phrase (this seems like itd be some rule based) from agda program

* Support for cs math - e.g. specifications of algorithms and their actual implementations
* Alternative syntaxes - graphical languages like grasshopper
* user interface
  - QA
  - Hoogle for proofs
* NL semantics (the semantic content is precisely the formal statements)
* Comparison / integration with ML approaches
* studies in concerete syntax -Harper psychology {\intersect} programming

Testing, with particular reference to the pgf grammar I developed




\section{Code}

\subsection{GF Parser}

\subsection{Additional Agda Hott Code}



\section{Testing}

Hello world 

Two citation examples: 
\cite{dunning1993} introduced a well-known method for extracting
collocations. Bilingual data can be used to train part-of-speech
taggers \citep{das2011}. Another one: \citep{cortes2014}


\newpage

\addcontentsline{toc}{section}{References}
\bibliographystyle{plain}
\bibliography{example_bibliography}

\newpage
\section{Appendices}

\end{document}


