\subsection{Agda}

Agda is an attempt to faithfully formalize Martin-Löf's intensional type theory
\cite{ml1984}. It is a functionaly programming language which, through an
interactive environment, allows one to iteratively apply rules and develop
constructive mathematics. It's current incarnation, Agda2 (but just called
Agda), was preceded by ALF, Cayenne, and Alfa, and the Agda1. On top of the
basic MLTT, Agda incorporates dependent records, inductive definitions, pattern
matching, a versatile module system, and a myriad of other bells and whistles
which are of interest generally and in various states of development but not
relevant to this work.

For our purposes, we will only look at what can in some sense be seen as the
kernel of Agda. Developing a full-blown GF grammar to incorporate more
advanced Agda features would require efforts beyond the scope of this work.

Agda's purpose is to manifest the propositions-as-types paradigm in a practical
and useable programming language. And while there are still many reasons one may
wish to use other programming languages, or just pen and paper to do her work,
there is a sense of purity one gets when writing Agda code. There are many good
resources for learning Agda,
  


% \cite{dybjer}
% \cite{norrell}
% \cite{wadler}

so we'll only give a cursory overview of what is relevant for this thesis.


% \begin{code}[hide]%
\>[0]\<%
\\
\>[0]\AgdaKeyword{module}\AgdaSpace{}%
\AgdaModule{primitives}\AgdaSpace{}%
\AgdaKeyword{where}\<%
\\
\>[0]\<%
\end{code}

Formation rules, are given by the data declaration, followed by some number of
constructors which correspond to the 


A proof the proof-theoretic this looks like the following


\begin{prooftree}
  \hypo{ \Gamma, A &\vdash B }
  \infer1[abs]{ \Gamma &\vdash A\to B }
  \hypo{ \Gamma \vdash A }
  \infer2[app]{ \Gamma \vdash B }
\end{prooftree}


\begin{code}%
\>[0]\<%
\\
\>[0]\AgdaKeyword{data}\AgdaSpace{}%
\AgdaDatatype{𝔹}\AgdaSpace{}%
\AgdaSymbol{:}\AgdaSpace{}%
\AgdaPrimitive{Set}\AgdaSpace{}%
\AgdaKeyword{where}\<%
\\
\>[0][@{}l@{\AgdaIndent{0}}]%
\>[2]\AgdaInductiveConstructor{true}\AgdaSpace{}%
\AgdaSymbol{:}\AgdaSpace{}%
\AgdaDatatype{𝔹}\<%
\\
%
\>[2]\AgdaInductiveConstructor{false}\AgdaSpace{}%
\AgdaSymbol{:}\AgdaSpace{}%
\AgdaDatatype{𝔹}\<%
\\
\>[0]\<%
\end{code}


-- $ \frac{\Gamma, x : A \vdash b : B} {\Gamma \vdash \lambda x. b : A \rightarrow
-- B} $

\begin{code}%
\>[0]\<%
\\
\>[0]\AgdaComment{-- if\AgdaUnderscore{}then\AgdaUnderscore{}else\AgdaUnderscore{} : \{A : Set\} → 𝔹 → A → A → A}\<%
\\
\>[0]\AgdaComment{-- if true then a1 else a2 = a1}\<%
\\
\>[0]\AgdaComment{-- if false then a1 else a2 = a2}\<%
\\
\>[0]\<%
\end{code}

-- data ℕ : Type where
--   zero : ℕ
--   suc  : ℕ → ℕ

-- data List (A : Type) : Type where
  

-- data Vector : 



-- \begin{code}%
\>[0]\<%
\\
\>[0]\AgdaComment{-- Type : Set₁}\<%
\\
\>[0]\AgdaComment{-- Type = Set}\<%
\\
%
\\[\AgdaEmptyExtraSkip]%
\>[0]\AgdaComment{-- \textbackslash{}end\{code\}}\<%


Formation rules, are given by the data declaration, followed by some number of
constructors which correspond to the 
Introduction rules are 

