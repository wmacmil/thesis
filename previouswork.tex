\section{Previous Work}

The prior exploration of these interleaving subjects is vast, and we can only
sample the available literature here. Indeed, there are so many approaches that
this work should be seen in a small (but important) case in the context of a
deep and broad literature \cite{surveyLang}. Acquiring expertise in such a
breadth of work is outside the scope of this thesis. Our approach, using
GF ASTs as a basis language for Mathematics and the logic the mathematical
objects are described in, is both distinct but has many roots and
interconnections with the remaining literature. The success of finding a
suitable language for mathematics will obviously require a comparative analysis
of the strengths and weaknesses in the goals in such a vast bibliography. 
 How the GF approach compares with this long merits careful consideration and
 future work.

It will function of our purpose, constrained by the limited scope of this work,
to focus on a few important resources.

\subsection{Ranta}

We give consideration to some of the historical precedents of GF, with respect
to Ranta. 

Hallgren/Ranta/Alfa

HoTT Grammar

\subsection{Mohan Ganesalingam}

The Language of Mathematics

\subsection{other authors}

The question 

We note that 
NaProche, Mizar, Coq (coquand), 
