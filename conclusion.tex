\section{Conclusion}

\begin{displayquote}
Concrete syntax is in some sense where programming language theory meets
psychology. \emph{Robert Harper}, \emph{Oregon Programming Languages Summer School 2017}
\end{displayquote}

There are two major problems in the reformulation of mathematics via typed
languages which underlay interactive proofs assistants and which fall under the
scope of this thesis.

The first is how to make a dependently typed programming language, capable of
formulating proposition and proof, more amenable to mathematicians with the goal
of improving semantic adequacy. The second asks how to facilitate the
formalization of mathematics, specifically the translation of theorem statements
and proofs to controlled natural languages, so-as to slightly disambiguate the
many ambiguities which pervade mathematical language and practice.

Progress in either of these directions will only be realized and through
significant time, labor, expertise, and most vitally, \emph{original thinking}
through collaborative efforts. It is uncertain what the role of parsers,
abstract syntax trees, linearization schemes, and other components of the GF
ecosystem will have on these efforts, but we have offered a taste of how these
tools may be applied..

Our work has perhaps only made a small contribution to these incredibly
difficult problems. Compiling various ideas across many
different fields has given some philosophical clarifications as to why
the problems are so difficult.

Additionally, through the analysis and comparison different GF grammars gives
has provided evidence that there's a feasibility of actually applying these
ideas to solve real problems. There is doubtless a role to play for
statistical methods in tackling these problems as well, and how these data-based
methods interface with the rule-based techniques is up to future scholarship.

Our contributions, partially original and partially extrapolated from others,
are the following :

\begin{itemize}
\item Introduce notions of \emph{syntactic completeness} and \emph{semantic adequacy}, so-as to
allow for understanding a piece of mathematics on a spectrum of formality and
clarity
\item Offer explicit comparisons, through examples of mathematics in a textual form and a type
  theoretic presentation
\item The developments of new GF grammars for analyzing this problem
\item The first comparison of all known GF grammars in this domain with respect
  to \emph{syntactic completeness} and \emph{semantic adequacy}
\item The development of an Agda library which mirrors the HoTT book so that
  future work can seek a possible ``large-scale" translation case study
\item Recognition that the GF approach is limited, especially as regards pragmatic
  concerns, but that it still provides insights here as well
\end{itemize}

It has been remarked that the bigger grammars gets, the more it begins to
resemble a domain specific resource grammar \cite{angelovSS}. We advocate to
actually produce a ``formal language RGL", whereby many of the ideas observed in
this work, like document structure, latex (and symbolic support generally),
custom lexical classes (like in BNFC), and many more may be accounted for.
The future grammar writers' time could be spent either focusing on
the scaling of programming language features or the actual linguistic analysis
of mathematics text - thereby making a more natural CNL for mathematics.

Despite the promise of various topics discussed here like Cubical Agda, the
Formal Abstracts Project , and the use of ITPs in mathematics education
\cite{buzzard2020will}, we don't foresee a convergence of type theorists and
mathematicians, even though devotion to the holy trinity would compel us to
believe so. GF as a programming language paradigm applied to this problem gives
us a stark contrast of how different these two approaches to mathematical
language. For the grammars of proof are insufficient to capture the complexity
and nuance about the language of proof, so much of which has yet to captured in
an existing linguistic framework. Grammars for propositions and definitions
offer a much more limited and seemingly feasible solution, because
mathematicians make these utterances with the explicit intention of
being comprehensible and unambiguous.


\subsection{The Mathematical Library of Babel}

\emph{The Library of Babel} \cite{borges} is a profound mirror held up to the
human species as regards our comprehension of the world through language. It
reflects our inability to grasp and reconcile human finitude. The infinite stack
of shelves, containing every book with every permutation of letters from the
Hebrew alphabet, leaves the humans who inhabit the closed space in a state of
discontent as regards their failures to navigate and interpret the myriad texts.

\emph{The Library} most certainly contains all mathematical statements, with all
possible foundations of mathematics, theorems and proofs of those theorems, in
all the possible syntactic presentations. In addition, it contains a catalogue
documenting the mathematical constructions, and how these constructions can be
encoded in the multiplicity of foundational systems. If there is a master GF
grammar for translating all of the mathematics, \emph{The Library} certainly
contains the source code for that as well.

Unfortunately, the library also contains all the erroneous proofs, whether they
be lexical errors or a reference to flawed lemma somewhere much deeper in the
library. There are certainly proofs of the Riemann conjecture, its negation, and
its undecidability.

When one perceives mathematics through the lens of human language, we must
acknowledge that mathematical content, constructions, and discoveries, are not
developments that come by chance, sifting through bags of words until
some gemstone gleams through the noise. Humans have to produce mathematical
constructions through hard labor, sweat, and tears. More importantly we create
mathematics through dialogue, laughter, and occasionally even dreams.

To imbue the sentences of mathematics which we see on paper, or in the terminal,
with meaning, we have some kind of internal mental mechanism that is at play
with our other mental faculties : our motor system and sensory capabilities
generally. We don't merely derive formulas by computing, but we distill ideas in
our general linguistic capacity to some kind of unambiguous, undeniable kernel.
The view that mathematics is just some subset of \emph{The Library} waiting
to be discovered or verified by a machine, is an incredibly misinformed and
myopic view of the subject. That mathematics is a human endeavor, complete with
all our lust, flaws, and ingenuity should be more clear after contemplating how
difficult it is to construct a grammar of proof.
