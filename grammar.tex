
\section{A Spectrum of GF Grammars for types}

We now discuss the various iterations of code which experimented with NL aspects

We should again emphasize the role of, in particular, Rantas two grammars, one
formalizing logic, and the other working with a case study of a real text\cite{aarneHott}



We now discuss the GF implementation, capable of parsing both natural language
and Agda syntax. The parser was appropriated from the cubicaltt BNFC parser,
de-cubified and then gf-ified. The languages are tightly coupled, so the
translation is actually quite simple. Some main differences are:

\begin{itemize}[noitemsep]

\item GF treats abstract and concrete syntax seperately. This allows GF to
support many concrete syntax implementation of a given grammar

\item Fixity is dealt with at the concrete syntax layer in GF.  This allows for
more refined control of fixity, but also results in difficulties : during
linearization their can be the insertion of extra parens.

\item GF supports dependent hypes and higher order abstract syntax, which makes
it suitable to typecheck at the parsing stage. It would very interesting to see
if this is interoperable with the current version of this work in later
iterations [Todo - add github link referncing work I've done in this direction]

\item GF also is enhanced by a PGF back-end, allowing an embedding of grammars
into, among other languages, Haskell.

\end{itemize}

While GF is targeted towards natural language translation, there's nothing
stopping it from being used as a PL tool as well, like, for instance, the
front-end of a compiler. The innovation of this thesis is to combine both uses,
thereby allowing translation between Controlled Natural Languages and
programming languages.

Example expressions the grammar can parse are seen below, which have been
verified by hand to be isomorphic to the corresponding cubicaltt BNFC trees:

\begin{verbatim}

data bool : Set where true | false 
data nat : Set where zero | suc ( n : nat )  
caseBool ( x : Set ) ( y z : x ) : bool -> Set = split false -> y || true -> z
indBool ( x : bool -> Set ) ( y : x false ) ( z : x true ) : ( b : bool ) -> x b = split false -> y || true  -> z
funExt  ( a : Set )   ( b : a -> Set )   ( f g :  ( x : a )  -> b x )   ( p :  ( x : a )  -> ( b x )   ( f x ) == ( g x )  )  : (  ( y : a )  -> b y )  f == g = undefined
foo ( b : bool ) : bool = b

\end{verbatim}

[Todo] add use cases

