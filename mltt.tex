\subsection{Propositions, Sets, and Types}

Complete overviews of Martin-Löf type theory been well-articulated elsewhere
\cite{dybjerITT}, and we have given a brief introduction in the appendix
\ref{judge}. We compare the syntax of mathematical constructions in FOL, a
possible natural language use from \cite{rantaLog}, and MLTT. From this vantage,
these look like simple symbolic translations, and in some sense, one doesn't
need the expressive power of system like GF to parse these to the same form.

Additionally, it is worth comparing the type theoretic and natural language
syntax with set theory, as is done in \autoref{fig:P1} and \autoref{fig:P2}. Now
we bear witness to some deeper cracks than were visible above. We note that the
type theoretic syntax is \emph{the same} in both tables, whereas the set
theoretic and logical syntax shares no overlap. This is because set theory and
first order logic are distinct domains classically, whereas in MLTT,
there is no distinguishing mathematical types from logical types - everything is
a type.

\begin{figure}[H]
\centering
\begin{tabular}{|c|c|c|c|} \hline
  FOL & MLTT & NL FOL & NL MLTT \\ \hline
  $\forall\ x\ \in \tau P(x)$ & $\Pi x : \tau.\ P(x)$     & $for\ all\ x\ in\ \tau,\ p$  & $the\  product\  over\  x\  in\ p$ \\ 
  $\exists\ x\ \in \tau P(x)$ & $\Sigma x : \tau.\ P(x)$  & $there\ exists\ an\ x\ in\ \tau\ such\ that\ p$ & $sigma\ x\ in\ \tau\ such\ that\ p$ \\ 
  $p\ \supset\ q$    & $p\ \to\ q$               & $if\ p\ then\ q$   & $p\  to\  q$ \\ 
  $p\ \wedge\ q$     & $p\ \times\ q$            & $p\ and\ q$        & $the\  product\  of\  p\  and\  q$ \\ 
  $p\ \lor\ q$       & $p\ +\ q$                 & $p\ or\ q$         & $the\  coproduct\  of\  p\  and\  q$ \\ 
  $\neg\ p$          & $\neg\ p$                 & $it\ is\ not\ the\ case\ that\ p$ & $not\ p$ \\ 
  $\top$             & $\top$                    & $true$             & $top$ \\ 
  $\bot$             & $\bot$                    & $false$            & $bottom$ \\ 
  $p\ =\ q$          & $p\ =_A\ q$            & $p\ equals\ q$     & $p\ propositionally\ equals\ q\ at\ A$ \\ \hline
\end{tabular}
\caption{FOL vs MLTT} \label{fig:P1}
\end{figure}

\begin{figure}[H]
\centering
\begin{tabular}{|c|c|c|c|} \hline
 Set Theory & MLTT & NL Set Theory & NL MLTT \\ \hline
 $S$          & $\tau$                 & $the\ set\ S$                     & $the\ type\ \tau$ \\ 
 $\mathbb{N}$ & $Nat$                  & $the\ set\ of\ natural\ numbers$  & $the\ type\ nat$ \\
 $S \times T$ & $S \times T$           & $the\ product\ of\ S\ and\ T$     & $the\  product\  of\  S\  and\  T$ \\
 $S \to T$    & $S \to T$              & $the\ function\ \from\ S\ to\ T$  & $p\  to\  q$ \\
 $\{x|P(x)\}$ & $\Sigma x : \tau.\ P(x)$ & $the\ set\ of\ x\ such\ that\ P$  & $there\ exists\ an\ x\ in\ \tau such\ that\ p$ \\
  $\emptyset$  & $\bot$                 & $the\ empty\ set$                 & $bottom$ \\
 $\{\emptyset\}$          & $\top$     & $the\ singleton\ set$ & $top$ \\
 $S \cup T$   & $?$                    & $the\ union\ of\ S\ and\ T$       & $?$ \\
 $S \subset T$ & $S <: T$              & $S\ is\ a\ subset\ of\ T$          & $S\ is\ a\ subtype\ of\ T$ \\
 $\aleph_1$          & $U_1$           & $the\ first\ uncountable\ cardinal$ & $the\ first\ large\ universe$        \\ \hline 
\end{tabular}
\caption{Sets vs MLTT} \label{fig:P2}
\end{figure}

We show the type and set comparisons in \autoref{fig:P2}. The basic types are
sometimes simpler to work with because they are expressive enough to capture
logical and set theoretic notions, but this also comes at a cost. The union of
two sets simply gives a predicate over the members of the sets, whereas union
and intersection types are often not considered ``core" in intuitionistic type
theory. The behavior of subtypes and subsets, while related in some ways, also
represents a semantic departure from sets and types. For example, while there
can be a greatest type in some sub-typing schema, there is no notion of a top
set.

We also note that, type theorists often interchange the logical,
set theoretic, and type theoretic vocabularies when describing types. Because 
types were developed to overcome shortcomings of set theory and classical logic,
the lexicons of all three ended up being blended, and in some sense, the type
theorist can substitute certain words that a classical mathematician
wouldn't.  Whereas $p\ implies\ q$ and $function\ from\ X\ to\ Y$ are not to
be mixed, the type theorist may in some sense default to either.
Nonetheless, pragmatically speaking, one would never catch a type theorist
saying $Nat\ implies\ Nat$ when expressing $Nat\ \to\ Nat$.

Terms become even messier, and this can be seen in just a small sample shown in
\autoref{fig:P3}. In simple type theory, one distinguishes between types and
terms at the syntactic level - this disappears when one allows dependency. As
will be seen later, the mixing of terms and types gives MLTT an incredible
expressive power, but undoubtedly introduces difficulties.
In set theory, everything is a set, so there is no distinguishing between
elements of sets and sets even though practically they function very
differently. Mathematicians only use sets because of their flexibility in so
many ways, not because the axioms of set theory make a compelling case for sets
being this kind of atomic form that makes up the mathematical universe. Category
theorists have discovered vast generalizations of sets where elements are
arrows. The comparison with categories and types is much tighter than with
sets. Regardless, mathematicians day-to-day work may not need all this general
infrastructure.

In FOL the proof rules themselves contain the necessary information to encode
the proofs or constructions. The terms in type theory compress and encode the
proof tree derivations - where nodes are displayed during the interactive
type-checking phase in ITPs.

\begin{figure}[H]
\centering
\begin{tabular}{|c|c|c|c|c|} \hline
 Set Theory & MLTT & NL Set Theory & NL MLTT & Logic \\ \hline
 $f(x) := p$ & $\lambda x. p$ & $f\ of\ x\ is\ p$ & $lambda\ x,\ p$ & $\supset-intro$ \\
 $f(p)$ & $f\ p$ & $f\ of\ p$ & $the\ application\ of\ f\ to\ p$ & $modus\ ponens$ \\
 $(x,y)$          & $(x,y)$ & $the\ pair\ of\ x\ and\ y$ & $the\ pair\ of\ x\ and\ y$ &  $\wedge-i$ \\
 $\pi_{1}\ x$      & $\pi_{1}\ x$ & $the\ first\ projection\ of\ x$ & $the\ first\ projection\ of\ x$ & $\wedge-e_1$ \\ \hline
\end{tabular}
\caption{Term syntax in Sets, Logic, and MLTT} \label{fig:P3}
\end{figure}

We don't include all the constructors for type theory here for space, but note some
interesting features:

\begin{itemize}
\item The disjoint union in set theory is actually defined using 
pairs - and therefore it doesn't have elimination forms other than those
for the product. The disjoint union is not common relative to coproducts in more
general categories.
\item $\lambda$ is a constructor for both the dependent and
non-dependent function, so its use in either case will be type-checked by Agda,
\item The projections for a $\Sigma$ type behaves differently from the
elimination principle for $\exists$, and this leads to
incongruities in the natural language presentation.
\end{itemize}

Finally, we should note that there are many linguistic presentations
mathematicians use for logical reasoning, i.e. the use of introduction and
elimination rules. They certainly seem to use linguistic forms more when dealing
with proofs, and symbolic notation for sets, so the investigation of how these
translate into type theory is a source of future work. Whereas propositions make
explicit all the relevant detail, and can be read by non-experts, proofs are
incredibly diverse and will be incomprehensible to those without expertise.

A detailed analysis of this should be done if and when a proper translation
corpus is built to account for some of the ways mathematicians articulate these
rules, as well as when and how mathematicians discuss sets,
symbolically and otherwise. To create translation with ``real" natural language 
is likely not to be very effective or interesting without a lot of evidence about
how mathematicians speak and write.